\section{Real Number Arithmetic}
\label{RealNumberArithmetic}


In this section we list the properties of real number arithmetic.  We will focus on those aspects that are most needed for Math 1410 (things like the algebraic axioms, and working with fractions), and gloss over those that are more calculus related (exponents, roots, etc.). Students wanting more detail than what is provided here might want to consult the resources available on the \textit{Math Basics} page on Moodle. In particular, since this is an \textit{algebra} textbook, we will assume that the reader has encountered the real number system before, and omit a definition, focusing instead on the algebraic properties of real numbers. We begin with the axioms for addition of real numbers.

\mnote{.7}{The set of real numbers is denoted usually denoted $\R$. A careful definition of $\R$ is actually quite complicated, and even most calculus classes choose not to include it. If you want to \textit{really} understand what the real numbers are, you'll need to take a course in Real Analysis, such as Math 3500.}

\medskip

\definition{realnumberaddition}{Properties of Real Number Addition}{

\begin{itemize}

\item  \textbf{Closure:}  For all real numbers $a$ and $b$,  $a+b$ is also a real number.

\item  \textbf{Commutativity:}  For all real numbers $a$ and $b$, $a+b = b+a$.

\item  \textbf{Associativity:}  For all real numbers $a$, $b$ and $c$, $a+(b+c) = (a+b)+c$.

\item  \textbf{Identity:}  There is a real number `$0$' so that for all real numbers $a$, $a+0 = a$.

\item  \textbf{Inverse:}  For all real numbers $a$, there is a real number $-a$ such that $a + (-a) = 0$.

\item \textbf{Definition of Subtraction:}  For all real numbers $a$ and $b$, $a - b = a + (-b)$.

\end{itemize}
}

\medskip

Next, we give real number multiplication a similar treatment.  Recall that we may denote the product of two real numbers $a$ and $b$ a variety of ways:  $ab$, $a \cdot b$, $a(b)$, $(a)(b)$ and so on.  We'll refrain from using $a \times b$ for real number multiplication in this text.

\medskip

\definition{realnumbermultiplication}{Properties of Real Number Multiplication}{

\begin{itemize}

\item  \textbf{Closure:}  For all real numbers $a$ and $b$,  $ab$ is also a real number.

\item  \textbf{Commutativity:}  For all real numbers $a$ and $b$, $ab = ba$.

\item  \textbf{Associativity:}  For all real numbers $a$, $b$ and $c$, $a(bc) = (ab)c$.

\item  \textbf{Identity:}  There is a real number `$1$' so that for all real numbers $a$, $a \cdot 1 = a$.

\item  \textbf{Inverse:}  For all real numbers $a \neq 0$, there is a real number $\dfrac{1}{a}$ such that $a \left(\dfrac{1}{a}\right) = 1$.

\item \textbf{Definition of Division:}  For all real numbers $a$ and $b \neq 0$, $a \div b = \dfrac{a}{b} = a  \left(\dfrac{1}{b}\right)$.
\end{itemize}
}

\medskip

While most students (and some faculty) tend to skip over these properties or give them a cursory glance at best, it is important to realize that the properties stated above are what drive the symbolic manipulation for all of Algebra.  When listing a tally of more than two numbers, $1 + 2 + 3$\label{howtoaddonetwothree} for example, we don't need to specify the order in which those numbers are added. Notice though, try as we might, we can add only two numbers at a time and it is the associative property of addition which assures us that we could organize this sum as $(1+2) + 3$ or $1+(2+3)$.  This brings up a note about `grouping symbols'.  Recall that parentheses and brackets are used in order to specify which operations are to be performed first.  In the absence of such grouping symbols, multiplication (and hence division) is given priority over addition (and hence subtraction). For example, $1 + 2 \cdot 3 = 1+6 = 7$, but $(1+2) \cdot 3 = 3 \cdot 3 = 9$.  As you may recall, we can `distribute' the $3$ across the addition if we really wanted to do the multiplication first:  $(1+2) \cdot 3 = 1\cdot 3 + 2 \cdot 3 = 3 + 6 = 9$. More generally, we have the following.

\medskip

\definition{distributiveproperty}{The Distributive Property and Factoring}{
%\smallskip
For all real numbers $a$, $b$ and $c$:

\begin{itemize}

\item  \textbf{Distributive Property:}   $a(b+c) = ab + ac$ and $(a+b)c = ac + bc$.

\item  \textbf{Factoring:}  $ab+ac = a(b+c)$ and $ac + bc = (a+b)c$.

\end{itemize}
}

\medskip

\noindent {\bf Warning:} A common source of errors for beginning students is the misuse (that is, lack of use) of parentheses. When in doubt, more is better than less: redundant parentheses add clutter, but do not change meaning, whereas writing $2x+1$ when you meant to write $2(x+1)$ is almost guaranteed to cause you to make a mistake. (Even if you're able to proceed correctly in spite of your lack of proper notation, this is the sort of thing that will get you on your grader's bad side, so it's probably best to avoid the problem in the first place.)

\medskip

\mnote{.6}{Taken together, the axioms for addition and multiplication of real numbers, along with the distributive property, tell us that algebraically, the set of real numbers is a \href{https://en.wikipedia.org/wiki/Field_(mathematics)}{\underline{field}}.}

It is worth pointing out that we didn't really need to list the Distributive Property both for $a(b+c)$ (distributing from the left) and $(a+b)c$ (distributing from the right), since the commutative property of multiplication gives us one from the other.  Also, `factoring' really is the same equation as the distributive property, just read from right to left. These are the first of many redundancies in this section, and they exist in this review section for one reason only - in our experience, many students \textit{see} these things differently so we will list them as such.   

\smallskip

It is hard to overstate the importance of the Distributive Property.  For example, in the expression $5(2+x)$, without knowing the value of $x$, we cannot perform the addition inside the parentheses first;  we must rely on the distributive property here to get  $5(2+x) = 5\cdot 2 + 5 \cdot x = 10 + 5x$.  The Distributive Property is also responsible for combining `like terms'.  Why is $3x + 2x = 5x$?  Because  $3x + 2x = (3+2)x = 5x$.  

\smallskip

We continue our review with summaries of other properties of arithmetic, each of which can be derived from the properties listed above.  First up are properties of the additive identity $0$.

\medskip

\theorem{propertiesofzero}{Properties of Zero}{

Suppose $a$ and $b$ are real numbers.

\begin{itemize}

\item  \textbf{Zero Product Property:} $ab = 0$ if and only if $a=0$ or $b=0$ (or both)

\textbf{Note:} This not only says that $0 \cdot a = 0$ for any real number $a$, it also says that the \textit{only} way to get an answer of `$0$' when multiplying two real numbers  is to have one (or both) of the numbers be `$0$' in the first place.

\item  \textbf{Zeros in Fractions:}  If $a \neq 0$, $\dfrac{0}{a} = 0 \cdot \left(\dfrac{1}{a}\right) = 0$.

\textbf{Note:}  The quantity $\dfrac{a}{0}$ is undefined.
\end{itemize}
}

\mnote{.8}{
The Zero Product Property drives most of the equation solving algorithms in Algebra because it allows us to take complicated equations and reduce them to simpler ones.  For example, you may recall that one way to solve  $x^2+x-6=0$ is by factoring the left hand side of this equation to get  $(x-2)(x+3) = 0$.  From here, we apply the Zero Product Property and set each factor equal to zero.  This yields  $x-2=0$ or $x+3=0$ so $x=2$ or $x=-3$.  This type of calculation is key to finding the eigenvalues of a matrix, as we'll see in Section \ref{sec:eigen}.
}


\medskip

We now continue with a review of arithmetic with fractions.


\keyidea{fractionarithmetic}{Properties of Fractions}{
Suppose $a$, $b$, $c$ and $d$ are real numbers.  Assume them to be nonzero whenever necessary; for example,  when they appear in a denominator.

\begin{itemize}

\item  \textbf{Identity Properties:}  $a = \dfrac{a}{1}$ and $\dfrac{a}{a} = 1$.

\item  \textbf{Fraction Equality:}  $\dfrac{a}{b} = \dfrac{c}{d}$ if and only if $ad = bc$. 

\item  \textbf{Multiplication of Fractions:}  $\dfrac{a}{b} \cdot \dfrac{c}{d} = \dfrac{ac}{bd}$. In particular:  $\dfrac{a}{b} \cdot c = \dfrac{a}{b} \cdot \dfrac{c}{1} = \dfrac{ac}{b}$

\mnote{.4}{\textbf{Note:}  A common denominator is \textbf{not} required to \textbf{multiply} or \textbf{divide} fractions!}

\item  \textbf{Division of Fractions:}  $\dfrac{a}{b} \left/ \dfrac{c}{d}\right. = \dfrac{a}{b} \cdot \dfrac{d}{c} = \dfrac{ad}{bc}$. 

In particular: $1 \left/ \dfrac{a}{b}\right. = \dfrac{b}{a}$ and  $\dfrac{a}{b} \left/ c \right. = \dfrac{a}{b} \left/ \dfrac{c}{1}\right.  = \dfrac{a}{b} \cdot \dfrac{1}{c} = \dfrac{a}{bc}$

\mnote{.48}{It's always worth remembering that division is the same as multiplication by the reciprocal. You'd be surprised how often this comes in handy.}

\item  \textbf{Addition and Subtraction of Fractions:}  $\dfrac{a}{b} \pm \dfrac{c}{b} = \dfrac{a \pm c}{b}$.  

\mnote{.36}{\textbf{Note:}  A common denominator \textbf{is} required to \textbf{add or subtract} fractions!}

\item  \textbf{Equivalent Fractions:}  $\dfrac{a}{b} = \dfrac{ad}{bd}$, since $ \dfrac{a}{b} = \dfrac{a}{b} \cdot 1 = \dfrac{a}{b} \cdot \dfrac{d}{d} = \dfrac{ad}{bd}$

\mnote{.28}{\textbf{Note:}  The \textit{only} way to change the denominator is to multiply both it and the numerator by the same nonzero value because we are, in essence, multiplying the fraction by $1$.}

\item  \textbf{`Reducing' Fractions:} $\dfrac{a\cancel{d}}{b\cancel{d}} = \dfrac{a}{b}$, since  $\dfrac{ad}{bd} = \dfrac{a}{b} \cdot \dfrac{d}{d} = \dfrac{a}{b} \cdot 1 = \dfrac{a}{b}$.

In particular, $\dfrac{ab}{b} = a$ since $\dfrac{ab}{b} = \dfrac{ab}{1 \cdot b} =  \dfrac{a \cancel{b}}{1 \cdot \cancel{b}} = \dfrac{a}{1} = a$ and $\dfrac{b-a}{a-b} = \dfrac{(-1)\cancel{(a-b)}}{\cancel{(a-b)}} = -1$.

\mnote{.2}{We reduce fractions by `cancelling' common factors - this is really just reading the previous property `from right to left'.\textbf{Caution:}  We may only cancel common \textbf{factors} from both numerator and denominator.}
\end{itemize}
}

\medskip


Next up is a review of the arithmetic of `negatives'. On page \pageref{realnumberaddition} we first introduced the dash which we all recognize as the `negative' symbol in terms of the additive inverse.  For example, the number $-3$ (read `negative $3$') is defined so that $3 + (-3) = 0$.  We then defined subtraction using the concept of the additive inverse again so that, for example, $5 - 3 = 5 + (-3)$.  



\medskip

\keyidea{propertiesofnegatives}{Properties of Negatives}{
Given real numbers $a$ and $b$ we have the following.  

\begin{itemize}

\item  \textbf{Additive Inverse Properties:}  $-a = (-1)a$ and $-(-a) = a$

\item  \textbf{Products of Negatives:} $(-a)(-b) = ab$. 

\item  \textbf{Negatives and Products:} $-ab = -(ab) = (-a)b = a(-b)$.

\item  \textbf{Negatives and Fractions:} If $b$ is nonzero, $-\dfrac{a}{b} = \dfrac{-a}{b} = \dfrac{a}{-b}$ and $\dfrac{-a}{-b} = \dfrac{a}{b}$.

\item  \textbf{`Distributing' Negatives:}  $-(a+b) = -a-b$ and $-(a-b) = -a + b = b-a$.

\item  \textbf{`Factoring' Negatives:} $-a-b = -(a+b)$ and $b-a = -(a-b)$.

\end{itemize}
}

\mnote{.5}{In this text we do not distinguish typographically between the dashes in the expressions `$5-3$' and `$-3$' even though they are mathematically quite different. In the expression `$5-3$,' the dash is a \textit{binary} operation (that is, an operation requiring \textit{two} numbers) whereas in `$-3$', the dash is a \textit{unary} operation (that is, an operation requiring only one number).  You might ask, `Who cares?'  Your calculator does - that's who!  In the text we can write $-3 - 3 = -6$ but that will not work in your calculator.  Instead you'd need to type $^{-}3 - 3$ to get $-6$ where the first dash comes from the `$+/-$' key.}

\mnote{.8}{
It might be junior high (elementary?) school material, but arithmetic with fractions is one of the most common sources of errors among university students. If you're not comfortable working with fractions, we strongly recommend seeing your instructor (or a tutor) to go over this material until you're completely confident that you understand it. Experience (and even formal educational \href{http://home.isr.umich.edu/releases/fractions-are-the-key-to-math-success-new-study-shows/}{\underline{studies}}) suggest that your success handling fractions corresponds pretty well with your overall success in passing your Mathematics courses.
}

\medskip

An important point here is that when we `distribute' negatives, we do so across addition or subtraction only.  This is because we are really distributing a factor of $-1$ across each of these terms:  $-(a+b) = (-1)(a+b) = (-1)(a) + (-1)(b) = (-a)+(-b) = -a-b$. Negatives do not `distribute' across multiplication:  $- (2 \cdot 3) \neq (-2)\cdot(-3)$. Instead, $-(2\cdot 3) = (-2)\cdot (3) = (2) \cdot (-3) = -6$.  The same sort of thing goes for fractions:  $- \frac{3}{5}$ can be written as $\frac{-3}{5}$ or $\frac{3}{-5}$, but not $\frac{-3}{-5}$.  It's about time we did a few examples to see how these properties work in practice.

\medskip

\example{fractionreview}{Arithmetic with fractions}{
Perform the indicated operations and simplify. By `simplify' here, we mean to have the final answer written in the form $\frac{a}{b}$ where $a$ and $b$ are integers which have no common factors.  Said another way, we want $\frac{a}{b}$ in `lowest terms'.

\begin{multicols}{3}
\begin{enumerate}

\item $\dfrac{1}{4} + \dfrac{6}{7}$\vphantom{$\dfrac{\dfrac{12}{5} - \dfrac{7}{24}}{1 + \left(\dfrac{12}{5}\right) \left(\dfrac{7}{24}\right)}$}
\item $\dfrac{5}{12} - \left(\dfrac{47}{30} - \dfrac{7}{3}\right)$\vphantom{$\dfrac{\dfrac{12}{5} - \dfrac{7}{24}}{1 + \left(\dfrac{12}{5}\right) \left(\dfrac{7}{24}\right)}$}
\item $\dfrac{\dfrac{12}{5} - \dfrac{7}{24}}{1 + \left(\dfrac{12}{5}\right) \left(\dfrac{7}{24}\right)}$ 

\setcounter{HW}{\value{enumi}}
\end{enumerate}
\end{multicols}


\begin{multicols}{2}
\begin{enumerate}
\setcounter{enumi}{\value{HW}}

\item $\dfrac{(2(2)+1)(-3-(-3)) - 5(4-7)}{4-2(3)}$\vphantom{$\left(\dfrac{3}{5} \right) \left(\dfrac{5}{13} \right) - \left(\dfrac{4}{5}\right) \left( - \dfrac{12}{13}\right)$}
\item $\left(\dfrac{3}{5} \right) \left(\dfrac{5}{13} \right) - \left(\dfrac{4}{5}\right) \left( - \dfrac{12}{13}\right)$

\setcounter{HW}{\value{enumi}}
\end{enumerate}
\end{multicols}
}
{
\begin{enumerate}

\item It may seem silly to start with an example this basic but experience has taught us not to take much for granted.  We start by finding the lowest common denominator and then we rewrite the fractions using that new denominator.  Since $4$ and $7$ are {\bf relatively prime},\index{relatively prime} meaning they have no factors in common, the lowest common denominator is $4 \cdot 7 = 28$.\[ \begin{array}{rclr}

\dfrac{1}{4} + \dfrac{6}{7} & = & \dfrac{1}{4} \cdot \dfrac{7}{7} + \dfrac{6}{7} \cdot \dfrac{4}{4} &  \text{Equivalent Fractions} \\ [10pt]
                                           & = & \dfrac{7}{28}  + \dfrac{24}{28} & \text{Multiplication of Fractions}\\ [10pt]
																					 & = & \dfrac{31}{28}                  & \text{Addition of Fractions} \\ \end{array} \]

The result is in lowest terms because $31$ and $28$ are relatively prime so we're done.

%%%%%%%%%%%%%%%%%%%

\mnote{.65}{We could have used $12 \cdot 30 \cdot 3 = 1080$ as our common denominator but then the numerators would become unnecessarily large.  It's best to use the \emph{lowest} common denominator.}

\item  We could begin with the subtraction in parentheses, namely $\frac{47}{30} - \frac{7}{3}$, and then subtract that result from $\frac{5}{12}$.  It's easier, however, to first distribute the negative across the quantity in parentheses and then use the Associative Property to perform all of the addition and subtraction in one step.  The lowest common denominator for all three fractions is $60$.

%\noindent\hskip-50pt
%\begin{minipage}{1.2\linewidth}

\begin{align*}
\dfrac{5}{12} - \left(\dfrac{47}{30} - \dfrac{7}{3}\right) & = \dfrac{5}{12} - \dfrac{47}{30} + \dfrac{7}{3} \quad \tag*{Distribute the Negative}\\ 
& =  \dfrac{5}{12} \cdot \dfrac{5}{5} - \dfrac{47}{30} \cdot \dfrac{2}{2} + \dfrac{7}{3} \cdot \dfrac{20}{20} \quad \tag*{Equivalent Fractions}\\ 
& =  \dfrac{25}{60} - \dfrac{94}{60} + \dfrac{140}{60} \quad \tag*{Multiplication of Fractions} \\
& =  \dfrac{71}{60} \quad \tag*{Addition and Subtraction of Fractions} \\ \end{align*}
%\end{minipage}

\drawexampleline

The numerator and denominator are relatively prime so the fraction is in lowest terms and we have our final answer.

%%%%%%%%%%%%%%%%%%%%%%%%%%%%%%%


\item What we are asked to simplify in this problem is known as a  `complex' or `compound' fraction.  Simply put, we have fractions within a fraction.  The longest division line (also called a `vinculum') acts as a grouping symbol, quite literally dividing the compound fraction into a numerator (containing fractions) and a denominator (which in this case does not contain fractions):

\[
\dfrac{\dfrac{12}{5} - \dfrac{7}{24}}{1 + \left(\dfrac{12}{5}\right) \left(\dfrac{7}{24}\right)} =  \dfrac{\left(\dfrac{12}{5} - \dfrac{7}{24}\right)}{\left(1 + \left(\dfrac{12}{5}\right) \left(\dfrac{7}{24}\right)\right)} 
\] 

The first step to simplifying a compound fraction like this one is to see if you can simplify the little fractions inside it. There are two ways to proceed. One is to simplify the numerator and denominator separately, and then use the fact that division is the same thing as multiplication by the reciprocal, as follows:

\noindent\vskip-10pt\begin{minipage}{\textwidth}
\begin{flalign*}
 \dfrac{\left(\dfrac{12}{5} - \dfrac{7}{24}\right)}{\left(1 + \left(\dfrac{12}{5}\right) \left(\dfrac{7}{24}\right)\right)} & = \dfrac{\left(\dfrac{12}{5}\cdot \dfrac{24}{24} - \dfrac{7}{24}\cdot \dfrac{5}{5}\right)}{\left(1\cdot \dfrac{120}{120} + \left(\dfrac{12}{5}\right) \left(\dfrac{7}{24}\right)\right)} & & \tag*{Equivalent Fractions}\\[5pt]
& =  \dfrac{288/120 - 35/120}{120/120 + 84/120} & & \tag*{Multiplication of fractions} \\[5pt]
& = \dfrac{253/120}{204/120} & & \tag*{Addition and subtraction of fractions} \\[5pt]
& = \dfrac{253}{\cancel{120}}\cdot \dfrac{\cancel{120}}{204} & & \tag*{Division of fractions and cancellation} \\[5pt]
 & =  \dfrac{253}{204} & &
 \end{flalign*}
\end{minipage}
 
 \medskip
 
Since $253 = 11 \cdot 23$ and $204 = 2 \cdot 2 \cdot 3 \cdot 17$ have no common factors our result is in lowest terms which means we are done.


While there is nothing wrong with the above approach, we can also use our Equivalent Fractions property to rid ourselves of the `compound' nature of this fraction straight away.  The idea is to multiply both the numerator and denominator by the lowest common denominator of each of the `smaller' fractions - in this case, $24 \cdot 5 = 120$.

\drawexampleline

\noindent\vskip-10pt\begin{minipage}{\textwidth}
\begin{flalign*}
 \dfrac{\left(\dfrac{12}{5} - \dfrac{7}{24}\right)}{\left(1 + \left(\dfrac{12}{5}\right) \left(\dfrac{7}{24}\right)\right)} & = \dfrac{\left(\dfrac{12}{5} - \dfrac{7}{24}\right) \cdot 120}{\left(1 + \left(\dfrac{12}{5}\right) \left(\dfrac{7}{24}\right)\right) \cdot 120} & & \tag*{Equivalent Fractions}\\[5pt]
& =  \dfrac{\left(\dfrac{12}{5}\right) (120) - \left(\dfrac{7}{24}\right) (120)}{(1)(120) + \left(\dfrac{12}{5}\right) \left(\dfrac{7}{24}\right)(120)} & & \tag*{Distributive Property} \\[5pt]
& =  \dfrac{\dfrac{12 \cdot 120}{5} - \dfrac{7 \cdot 120}{24}}{120 + \dfrac{12 \cdot 7 \cdot 120}{5 \cdot 24}} & & \tag*{Multiply fractions} \\[5pt]
& =  \dfrac{\dfrac{12 \cdot 24 \cdot \cancel{5}}{\cancel{5}} - \dfrac{7 \cdot 5 \cdot \cancel{24}}{\cancel{24}}}{120 + \dfrac{12 \cdot 7 \cdot \cancel{5} \cdot \cancel{24}}{\cancel{5} \cdot \cancel{24}}} & & \tag*{Factor and cancel} \\[5pt]
 & =  \dfrac{(12 \cdot 24) - (7 \cdot 5)}{120 + (12 \cdot 7)} & &\\[5pt] 
 & =  \dfrac{288 - 35}{120 + 84} =  \dfrac{253}{204},& & 
\end{flalign*}
\end{minipage}

\medskip
which is the same as we obtained above.

%%%%%%%%%%%%%%%%%%%%%%%%%%%%%%%%%%%%%%%%

																					
\item  This fraction may look simpler than the one before it, but the negative signs and parentheses mean that we shouldn't get complacent.  Again we note that the division line here acts as a grouping symbol.  That is, 

\[ 
\dfrac{(2(2)+1)(-3-(-3)) - 5(4-7)}{4-2(3)} = \dfrac{\left((2(2)+1)(-3-(-3)) - 5(4-7) \right)}{(4-2(3))}
\]

This means that we should simplify the numerator and denominator first, then perform the division last.  We tend to what's in parentheses first, giving multiplication priority over addition and subtraction.
\begin{align*}
\dfrac{(2(2)+1)(-3-(-3)) - 5(4-7)}{4-2(3)} & =  \dfrac{(4+1)(-3+3)-5(-3)}{4 - 6}   \\
	& =  \dfrac{(5)(0) + 15}{-2}   \\ 
	& =  \dfrac{15}{-2}  \\[5pt] 
	& =  -\dfrac{15}{2}  \tag*{Properties of Negatives}
\end{align*}
Since $15 = 3\cdot 5$ and $2$ have no common factors, we are done.
																			

%%%%%%%%%%%%%%%%%%%%%%%%%%%%%%


\item  In this problem, we have multiplication and subtraction.  Multiplication takes precedence so we perform it first.  Recall that to multiply fractions, we do \textit{not} need to obtain common denominators;  rather, we multiply the corresponding numerators together along with the corresponding denominators.  Like the previous example, we have parentheses and negative signs for added fun!
\begin{align*}
\left(\dfrac{3}{5} \right) \left(\dfrac{5}{13} \right) - \left(\dfrac{4}{5}\right) \left( - \dfrac{12}{13}\right) & =  \dfrac{3 \cdot 5}{5 \cdot 13} - \dfrac{4\cdot (-12)}{5 \cdot 13}  \tag*{Multiply fractions}\\[5pt] 
& =  \dfrac{15}{65} - \dfrac{-48}{65}  \\[5pt]
& =  \dfrac{15}{65} + \dfrac{48}{65}  \tag*{Properties of Negatives}\\[5pt]
& =  \dfrac{15+48}{65}   \tag*{Add numerators} \\[5pt] 
& =  \dfrac{63}{65} 
\end{align*}

Since $64 = 3 \cdot 3 \cdot 7$ and $65 = 5 \cdot 13$ have no common factors, our answer $\dfrac{63}{65}$ is in lowest terms and we are done.
\end{enumerate}
} 

\medskip

Of the issues discussed in the previous set of examples none causes students more trouble than simplifying compound fractions.  We presented two different methods for simplifying them:  one in which we simplified the overall numerator and denominator and then performed the division and one in which we removed the compound nature of the fraction at the very beginning.   We encourage the reader to go back and use both methods on each of the compound fractions presented.  Keep in mind that when a compound fraction is encountered in the rest of the text it will usually be simplified using only one method and we may not choose your favourite method.  Feel free to use the other one in your notes.




\printexercises{exercises/00_02_exercises}
