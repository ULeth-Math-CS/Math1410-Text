\Huge
\noindent {\bf \textsc{Preface}}\\
\large
\emph{A Note to Students, Teachers, and other Readers}
\vspace{1in}
\normalsize

Thank you for reading this short preface. Allow me to share a few key points about the text so that you may better understand what you will find beyond this page.

This text deals with \emph{matrix} algebra, as opposed to \emph{linear} algebra. Without arguing semantics, I view matrix algebra as a subset of linear algebra, focused primarily on basic concepts and solution techniques. There is little formal development of theory and abstract concepts are avoided. This is akin to the master carpenter teaching his apprentice how to use a hammer, saw and plane before teaching how to make a cabinet.

\emph{This book is intended to be read.} Each section starts with ``\emph{AS YOU READ}'' questions that the reader should be able to answer after a careful reading of the section even if all the concepts of the section are not fully understood. I use these questions as a daily reading quiz for my students. The text is written in a conversational manner, hopefully resulting in a text that is easy (and even enjoyable) to read.

Many examples are given to illustrate concepts. When a concept is first learned, I try to demonstrate all the necessary steps so mastery can be obtained. Later, when this concept is now a tool to study another idea, certain steps are glossed over to focus on the new material at hand. I would suggest that technology be employed in a similar fashion.

This text is ``open.'' If it nearly suits your needs as an instructor, but falls short in any way, feel free to make changes. I will readily share the source files (and help you understand them) and you can do with them as you wish. I would find such a process very rewarding on my own end, and I would enjoy seeing this text become better and even eventually grow into a separate linear algebra text. I do ask that the Creative Commons copyright be honored, in that any changes acknowledge this as a source and that it only be used non commercially.

This is the third edition of the {\it Fundamentals of Matrix Algebra} text. I had not intended a third edition, but it proved necessary given the number of errors found in the second edition and the other opportunities found to improve the text. It varies from the first and second editions in mostly minor ways. I hope this edition is ``stable;'' I do not want a fourth edition anytime soon.%Errors have been corrected. %Some style changes have been made, such as changing typefaces and making the Definition, Theorem and Key Idea boxes more uniform. A few exercises have been changed as the computations they once required were onerous to point of obfuscation of the underlying principle to be learned. A few new exercises have been added.

Finally, I welcome any and all feedback. Please contact me with suggestions, corrections, etc.

Sincerely,

Gregory Hartman



