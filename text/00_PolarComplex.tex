\section{The Polar Form of Complex Numbers}

\label{PolarComplex}

In this section, we return to our study of complex numbers begun in Section \ref{CmpNums}.  Recall that a \index{complex number ! definition of}\sword{complex number} is a number of the form $z = a + bi$ where $a$ and $b$ are real numbers and $i$ is the imaginary unit defined by $i = \sqrt{-1}$. The number $a$ is called the \index{complex number ! real part} \sword{real part} \index{real part of a complex number} of $z$, denoted  $\operatorname{Re}(z)$, while the real number $b$ is called the \index{imaginary part of a complex number} \index{complex number ! imaginary part}\sword{imaginary part} of $z$, denoted $\operatorname{Im}(z)$.  

To start off this section, we associate each complex number $z = a+bi$ with the point $(a,b)$ on the coordinate plane.  In this case, the $x$-axis is relabelled as the \index{real axis}\sword{real axis}, which corresponds to the real number line as usual,  and the $y$-axis is relabelled as the \index{imaginary axis}\sword{imaginary axis}, which is demarcated in increments of the imaginary unit $i$.  The plane determined by these two axes is called the \index{complex plane}\sword{complex plane}.

%\mnote{.7}{Saying that $\operatorname{Re}(z)$ and $\operatorname{Im}(z)$ are `well-defined' means that no matter how we express $z$, the number $\operatorname{Re}(z)$ is always the same, and the number $\operatorname{Im}(z)$ is always the same.  In other words, $\operatorname{Re}$ and $\operatorname{Im}$ are \textit{functions} of complex numbers.}

\medskip

\noindent\begin{minipage}{\textwidth}
\begin{center}
\myincludegraphics[width=0.7\textwidth]{figures/PolarComplex-1}
\end{center}
\captionsetup{type=figure}
\caption{The complex plane}\label{fig:complexplane}
\end{minipage}

\medskip

Since the ordered pair $(a,b)$ gives the \textit{rectangular} coordinates associated with the complex number $z = a+bi$, the expression $z=a+bi$ is called the \index{complex number ! rectangular form}\index{rectangular form of a complex number}\sword{rectangular form} of $z$. Of course, we could just as easily associate $z$ with a pair of \textit{polar} coordinates $(r,\theta)$.  Although it is not as straightforward as the definitions of $\operatorname{Re}(z)$ and $\operatorname{Im}(z)$, we can still give $r$ and $\theta$ special names in relation to $z$.

\smallskip

\definition{modulusargumentdefn}{The Modulus and Argument of Complex Numbers}{  Let $z = a+bi$ be a complex number with $a = \operatorname{Re}(z)$ and $b=\operatorname{Im}(z)$.  Let $(r,\theta)$ be a polar representation of the point with rectangular coordinates $(a,b)$ where $r \geq 0$.

\begin{itemize}

\item  The \index{complex number ! modulus ! definition of}\index{modulus of a complex number ! definition of}\sword{modulus} of $z$, denoted $\abs{z}$, is defined by $\abs{z} = r$.

\item  The angle $\theta$ is an \index{complex number ! argument ! definition of}\index{argument ! of a complex number ! definition of}\sword{argument} of $z$. The set of all arguments of $z$ is denoted $\operatorname{arg}(z)$.

\item  If $z \neq 0$ and $-\pi < \theta \leq \pi$, then $\theta$ is the \index{complex number ! principal argument}\index{principal argument of a complex number}\sword{principal argument} of $z$, written $\theta = \operatorname{Arg}(z)$. 

\end{itemize}
}

\smallskip

Some remarks about  Definition \ref{modulusargumentdefn} are in order.  We know from Section \ref{IntroPolar} that every point in the plane has infinitely many polar coordinate representations $(r,\theta)$. Thus, if we want to speak of ``the'' argument, we can't simply choose one value of $\theta$ that works: there are infinitely many other possible answers.

One option, as given above, is to simply include \textit{all} possible answers. This idea becomes important if one wants to do calculus with complex numbers, but doesn't play a role in the algebraic considerations we're interested in. The other, also seen above, is to specify an interval of length $2\pi$ in which the argument must lie. The interval $(-\pi,\pi]$, as given above, is a common choice; another is to use the interval $[0,2\pi)$.

\mnote{.7}{For the purposes of Math 1410, it suffices to speak of ``an'' argument of $z$, and choose whatever interval is convenient for a given problem. The one exception to this is \textbf{online homework}: the easiest way to program a computer to grade the answer for a problem on the polar form of a complex number is to specify an interval for the argument, so be sure to read any online homework problems carefully, and choose $\theta$ to lie in the appropriate interval.}

For this course we won't worry much about distinguishing between ``argument'' and ``principal argument'': any value of $\theta$ that gives the correct complex number will do. While this ambiguity may be annoying at times, we'll see at the end of this section that being able to add an integer multiple of $2\pi$ to the argument without changing the value of $z$ gives us an ingenious method for computing roots of complex numbers.

Let us explore this new terminology with an example.


\medskip

\example{plotmodargex}{Components of a complex number}{  For each of the following complex numbers find $\operatorname{Re}(z)$, $\operatorname{Im}(z)$, $\abs{z}$, $\operatorname{arg}(z)$ and $\operatorname{Arg}(z)$.  Plot $z$ in the complex plane.

\begin{multicols}{2}

\begin{enumerate}

\item  $z = \sqrt{3}-i$

\item  $z = -2+4i$

\item  $z = 3i$

\item  $z = -117$

\end{enumerate}

\end{multicols}}
{
\begin{enumerate}

\item For $z = \sqrt{3} -i = \sqrt{3} + (-1)i$, we have $\operatorname{Re}(z) = \sqrt{3}$ and $\operatorname{Im}(z) = -1$.   To find $\abs{z}$, $\operatorname{arg}(z)$ and $\operatorname{Arg}(z)$, we need to find a polar representation $(r,\theta)$ with $r \geq 0$ for the point $P(\sqrt{3},-1)$ associated with $z$.   We know $r^2 = (\sqrt{3})^2 + (-1)^2 = 4$, so $r = \pm 2$.  Since we require $r \geq 0$, we choose $r =2$, so $\abs{z} = 2$.  Next, we find a corresponding angle $\theta$.  Since $r>0$ and $P$ lies in Quadrant IV, $\theta$ is a Quadrant IV angle.  We know $\tan(\theta) = \frac{-1}{\sqrt{3}} = -\frac{\sqrt{3}}{3}$, so $\theta = -\frac{\pi}{6} + 2\pi k$ for integers $k$.  Hence, $\operatorname{arg}(z) = \left\{-\frac{\pi}{6} + 2\pi k \, | \, \text{$k$ is an integer} \right\}$. Of these values, only  $\theta = -\frac{\pi}{6}$  satisfies the requirement that $-\pi < \theta \leq \pi$, hence $\operatorname{Arg}(z) = -\frac{\pi}{6}$.  

\mnote{.4}{While we can always use the relationship $\tan\theta = \frac{y}{x}$ to determine a value for the argument of $z=x+iy$, it is much easier in practice to ``factor out'' the modulus, and then try to identify what's left over as a point on the unit circle. For $z=\sqrt{3}-i$ we found $\abs{z}=2$, so
\[
z = 2\left(\frac{\sqrt{3}}{2}+i\left(-\frac{1}{2}\right)\right).
\]
If we know our unit circle (or have a copy of it handy), we can immediately spot that the point $\left(\frac{\sqrt{3}}{2},-\frac{1}{2}\right)$ corresponds to the angle $\theta = -\frac{\pi}{6}$.
}


\item The complex number $z = -2+4i$ has  $\operatorname{Re}(z) = -2$,   $\operatorname{Im}(z) = 4$, and is associated with the point $P(-2,4)$.  Our next task is to find a polar representation $(r,\theta)$ for $P$ where $r \geq 0$. Running through the usual calculations gives $r = 2\sqrt{5}$, so $\abs{z} = 2\sqrt{5}$.  To find $\theta$, we get $\tan(\theta) = -2$, and since $r > 0$ and $P$ lies in Quadrant II, we know $\theta$  is a Quadrant II angle.  We find $\theta = \pi + \arctan(-2) + 2\pi k$, or, more succinctly  $\theta = \pi - \arctan(2) + 2\pi k$ for integers $k$.  Hence $\operatorname{arg}(z) = \left\{\pi - \arctan(2) + 2\pi k \, | \, \text{$k$ is an integer}\right\}$.  Only  $\theta = \pi - \arctan(2)$ satisfies the requirement $-\pi < \theta \leq \pi$,  so $\operatorname{Arg}(z) = \pi - \arctan(2)$. 

\item    We rewrite $z = 3i$ as $z = 0+3i$ to find $\operatorname{Re}(z) = 0$ and $\operatorname{Im}(z) = 3$.  The point in the plane which corresponds to $z$ is $(0,3)$ and while we could go through the usual calculations to find the required polar form of this point, we can almost `see' the answer.  The point  $(0,3)$ lies $3$ units away from the origin on the positive $y$-axis.  Hence, $r=\abs{z}=3$ and $\theta = \frac{\pi}{2} + 2\pi k$ for integers $k$. We get $\operatorname{arg}(z) = \left\{ \frac{\pi}{2} + 2\pi k \, | \, \text{$k$ is an integer} \right\}$ and $\operatorname{Arg}(z) = \frac{\pi}{2}$. 

\item As in the previous problem, we write $z = -117 = -117 + 0i$ so $\operatorname{Re}(z) = -117$ and $\operatorname{Im}(z) = 0$. The number $z = -117$ corresponds to the point $(-117,0)$, and this is another instance where  we can determine the polar form `by eye'.  The point $(-117,0)$ is $117$ units away from the origin along the negative $x$-axis.  Hence, $r=\abs{z}=117$ and $\theta = \pi + 2\pi  = (2k+1)\pi k$ for integers $k$. We have  $\operatorname{arg}(z) = \left\{ (2k+1)\pi \, | \, k \text{ is an  integer} \right\}$.  Only one of these values, $\theta = \pi$, just barely lies in the interval $(-\pi, \pi]$ which means and $\operatorname{Arg}(z) =\pi$. We plot $z$ along with the other numbers in this example in Figure \ref{fig:polarcom1} below.

\medskip

\noindent\begin{minipage}{\textwidth}
\begin{center}
\myincludegraphics[width=0.7\textwidth]{figures/PolarComplex-2}
\end{center}
\captionsetup{type=figure}
\caption{Plots of the four complex numbers in Example \ref{plotmodargex}}\label{fig:polarcom1}
\end{minipage}

\end{enumerate}
}

\medskip

Now that we've had some practice computing the modulus and argument of some complex numbers, it is time to explore their properties.  We have the following theorem.

\mnote{.25}{Since the absolute value $\abs{x}$ of a real number $x$ can be viewed as the distance from $x$ to $0$ on the number line, the first property in Theorem \ref{modprops} justifies the notation $\abs{z}$ for modulus.  We leave it to the reader to show that if $z$ is real, then the definition of modulus coincides with absolute value so the notation $\abs{z}$ is unambiguous.}

\smallskip

\theorem{modprops}{Properties of the Modulus}{  Let $z$ and $w$ be complex numbers. \index{complex number ! modulus ! properties of} \index{modulus of a complex number ! properties of}

\begin{itemize}

\item  $\abs{z}$ is the distance from $z$ to $0$ in the complex plane

\item  $\abs{z} \geq 0$ and $\abs{z} = 0$ if and only if $z=0$

\item  $\abs{z} = \sqrt{\operatorname{Re}(z)^2 + \operatorname{Im}(z)^2}$

\item  \textbf{Product Rule:}  $\abs{zw} = \abs{z}\abs{w}$ \index{product rule ! for the modulus of a complex number}

\item  \textbf{Power Rule:}  $\abs{z^{n}} = \abs{z}^{n}$ for all natural numbers, $n$ \index{power rule ! for the modulus of a complex number}

\item  \textbf{Quotient Rule:}  $\abs{\dfrac{z}{w}} = \dfrac{\abs{z}}{\abs{w}}$, provided $w \neq 0$ \index{quotient rule ! for the modulus of a complex number}

\end{itemize}
}

\smallskip

The first property is simply a consequence of the distance formula: $\abs{a+ib} = \sqrt{a^2+b^2}$ is precisely the distance from the point $(a,b)$ to the origin.  For the second property, note that since $\abs{z}$ is a distance, $\abs{z} \geq 0$.  Furthermore,  $\abs{z} = 0$ if and only if the distance from $z$ to $0$ is $0$, and the latter happens if and only if $z = 0$, which is what we were asked to show. 

For the third property, we note that since $a = \operatorname{Re}(z)$ and $b = \operatorname{Im}(z)$, $z = \sqrt{a^2+b^2} = \sqrt{\operatorname{Re}(z)^2 + \operatorname{Im}(z)^2}$.


\smallskip

To prove the product rule, suppose $z = a + bi$ and  $w = c + di$ for real numbers $a$, $b$, $c$ and $d$.  By Definition \ref{def:compmultiplication}, we have $zw = (ac-bd) + (ad+bc)i$. Therefore,

\begin{align*}
\abs{zw} & =  \sqrt{(ac-bd)^2+(ad+bc)^2}  \\
	 & =  \sqrt{a^2c^2 - 2abcd + b^2d^2 + a^2d^2 +2abcd + b^2c^2}  \tag*{Expand} \\
	 & = \sqrt{a^2c^2 + a^2d^2 + b^2c^2 + b^2d^2}  \tag*{Rearrange terms} \\
	 & =  \sqrt{a^2\left(c^2+d^2\right) + b^2\left(c^2+d^2\right)}  \tag*{Factor} \\
	 & =  \sqrt{\left(a^2+b^2\right)\left(c^2+d^2\right)}  \tag*{Factor}\\
	 & = \sqrt{a^2+b^2} \sqrt{c^2+d^2}  \tag*{Product Rule for Radicals} \\
	 & =  \abs{z} \abs{w}  \tag*{Definition of $\abs{z}$ and $\abs{w}$}
\end{align*}
		 
Hence $\abs{zw} = \abs{z} \abs{w}$ as required.  

\smallskip

\mnote{.75}{In case you were not convinced by the argument for the second property in Theorem \ref{modprops}, we can work through the underlying Algebra to see this is true.  We know  $\abs{z} = 0$ if and only if $\sqrt{a^2+b^2} = 0$ if and only if $a^2+b^2 = 0$, which is true if and only if $a = b = 0$.  The latter happens if and only if $z = a + bi =0$.  There.}

The Power Rule essentially follows from repeated application of the Product Rule.
Like the Power Rule, the Quotient Rule can also be established with the help of the Product Rule. We assume $w \neq 0$ (so $\abs{w} \neq 0$) and we get
\begin{align*}
\left\vert \dfrac{z}{w} \right\rvert &  =  \left\lvert (z) \left( \dfrac{1}{w} \right) \right\rvert  \\[3pt] 
							& =  |z| \left\lvert \dfrac{1}{w}\right\rvert \tag*{Product Rule.}
\end{align*}
Hence, the proof really boils down to showing $\left\lvert \dfrac{1}{w} \right\rvert = \dfrac{1}{\abs{w}}$.  This is left as an exercise.

\smallskip

By restating Theorem \ref{polarrectangularconversion} in terms of complex numbers, we get the following characterization of the argument of a complex number in terms of its real and imaginary parts.

\smallskip

\theorem{argprops}{Properties of the Argument}{  Let $z$ be a complex number. \index{complex number ! argument ! properties of} \index{argument ! of a complex number ! properties of}
 
\begin{itemize}

\item  If $\operatorname{Re}(z) \neq 0$ and $\theta \in \operatorname{arg}(z)$, then $\tan(\theta) = \frac{\operatorname{Im}(z)}{\operatorname{Re}(z)}$.

\item  If $\operatorname{Re}(z) = 0$ and  $\operatorname{Im}(z) > 0$, then $\operatorname{arg}(z) = \left\{ \frac{\pi}{2} + 2\pi k \, | \, \text{$k$ is an integer} \right\}$.

\item  If $\operatorname{Re}(z) = 0$ and $\operatorname{Im}(z) < 0$, then $\operatorname{arg}(z) = \left\{ -\frac{\pi}{2} + 2\pi k \, | \, \text{$k$ is an integer} \right\}$.

\item  If $\operatorname{Re}(z) = \operatorname{Im}(z) = 0$, then $z = 0$ and $\operatorname{arg}(z) = (-\infty, \infty)$.

\end{itemize}
}

\smallskip

%To prove Theorem \ref{argprops}, suppose $z = a + bi$ for real numbers $a$ and $b$.  By definition, $a = \operatorname{Re}(z)$ and $b = \operatorname{Im}(z)$, so the point associated with $z$ is $(a,b) = \left(\operatorname{Re}(z), \operatorname{Im}(z)\right)$.  From Section \ref{IntroPolar}, we know that if $(r,\theta)$ is a polar representation for $\left(\operatorname{Re}(z), \operatorname{Im}(z)\right)$, then $\tan(\theta) = \frac{\operatorname{Im}(z)}{\operatorname{Re}(z)}$, provided $\operatorname{Re}(z) \neq 0$.  If $\operatorname{Re}(z) = 0$ and $\operatorname{Im}(z) > 0$, then $z$ lies on the positive imaginary axis.  Since we take $r > 0$,  we have that $\theta$ is coterminal with $\frac{\pi}{2}$, and the result follows.   If $\operatorname{Re}(z) = 0$ and $\operatorname{Im}(z) < 0$, then $z$ lies on the negative imaginary axis, and a similar argument shows $\theta$ is coterminal with $-\frac{\pi}{2}$.  The last property in the theorem was already discussed in the remarks following Definition \ref{modulusargumentdefn}.  

\pagebreak

Our next goal is to completely marry the Geometry and the Algebra of the complex numbers.  To that end,  consider Figure \ref{fig:polarcom2} below.

\medskip

\noindent\begin{minipage}{\textwidth}
\begin{center}
\myincludegraphics[width=0.75\textwidth]{figures/PolarComplex-3}
\end{center}
\captionsetup{type=figure}
\caption{Polar coordinates, $(r, \theta)$ associated with $z = a+bi$ with $r \geq 0$.}\label{fig:polarcom2}
\end{minipage}

\medskip

We know from  Theorem \ref{polarrectangularconversion} that $a = r\cos(\theta)$ and $b = r\sin(\theta)$. Making these substitutions for $a$ and $b$ gives $z = a + bi = r\cos(\theta) + r \sin(\theta) i = r \left[\cos(\theta) + i \sin(\theta)\right]$. The expression `$\cos(\theta) + i\sin(\theta)$' is abbreviated $\operatorname{cis}(\theta)$\index{cis($\theta$)} so we can write  $z = r\operatorname{cis}(\theta)$.	Since  $r = \abs{z}$ and $\theta \in \operatorname{arg}(z)$, we get

\medskip

\definition{polarformcomplex}{A Polar Form of a Complex Number}{ \index{complex number ! polar form ! cis-notation} \index{polar form of a complex number}  Suppose $z$ is a complex number and $\theta \in \operatorname{arg}(z)$.  The expression:  \[\abs{z} \operatorname{cis}(\theta) = \abs{z}\left[ \cos(\theta) + i \sin(\theta)\right]\]

is called a polar form for $z$.
}

\medskip										

Since there are infinitely many choices for $\theta \in \operatorname{arg}(z)$, there infinitely many polar forms for $z$, so we used the indefinite article `a' in Definition \ref{polarformcomplex}.  It is time for an example.

\medskip

\example{polarcomplexex}{Converting between rectangular and polar form}{
\begin{enumerate}

\item Find the rectangular form of the following complex numbers. Find $\operatorname{Re}(z)$ and $\operatorname{Im}(z)$.

\begin{multicols}{2}

\begin{enumerate}

\item  $z = 4 \operatorname{cis}\left(\frac{2\pi}{3}\right)$

\item  $z = 2 \operatorname{cis}\left(-\frac{3\pi}{4}\right)$

\item  $z = 3 \operatorname{cis}(0)$

\item  $z = \operatorname{cis}\left(\frac{\pi}{2}\right)$

\end{enumerate}

\end{multicols}

\item  Use the results from Example \ref{plotmodargex} to find a polar form of the following complex numbers.

\begin{multicols}{2}

\begin{enumerate}

\item  $z = \sqrt{3}-i$

\item  $z = -2+4i$

\item  $z = 3i$

\item  $z = -117$

\end{enumerate}

\end{multicols}



\end{enumerate}\pagebreak}
{
\begin{enumerate}


\item The key to this problem is to write out $\operatorname{cis}(\theta)$ as $\cos(\theta) + i\sin(\theta)$.

\begin{enumerate}

\item By definition, $z =4 \operatorname{cis}\left(\frac{2\pi}{3}\right) = 4\left[\cos\left(\frac{2\pi}{3}\right) + i \sin\left(\frac{2\pi}{3}\right)\right]$. After some simplifying, we get $z = -2 + 2i\sqrt{3}$, so that $\operatorname{Re}(z) = -2$ and $\operatorname{Im}(z) = 2\sqrt{3}$.

\item Expanding, we get $z = 2 \operatorname{cis}\left(-\frac{3\pi}{4}\right) = 2\left[\cos\left(-\frac{3\pi}{4}\right) + i \sin\left(-\frac{3\pi}{4}\right)\right]$. From this, we find $z= -\sqrt{2} - i\sqrt{2}$, so $\operatorname{Re}(z) = -\sqrt{2} = \operatorname{Im}(z)$.

\item  We get  $z = 3 \operatorname{cis}(0) = 3\left[\cos(0) + i\sin(0)\right] = 3$.  Writing $3 = 3 + 0i$, we get $\operatorname{Re}(z) = 3$ and $\operatorname{Im}(z) = 0$, which makes sense seeing as $3$ is a real number.

\item  Lastly, we have  $z = \operatorname{cis}\left(\frac{\pi}{2}\right) = \cos\left(\frac{\pi}{2}\right)+ i\sin\left(\frac{\pi}{2}\right) = i$.  Since $i = 0 + 1i$, we get $\operatorname{Re}(z) = 0$ and $\operatorname{Im}(z) = 1$.  Since $i$ is called the `imaginary unit,'  these answers make perfect sense.

\end{enumerate}

\item  To write a polar form of a complex number $z$, we need two pieces of information:  the modulus $\abs{z}$ and an argument (not necessarily the principal argument) of $z$.   We shamelessly mine our solution to  Example \ref{plotmodargex} to find what we need.

\begin{enumerate}

\item  For $z = \sqrt{3}-i$, $\abs{z} = 2$ and $\theta = -\frac{\pi}{6}$, so $z = 2 \operatorname{cis}\left(-\frac{\pi}{6}\right)$.  We can check our answer by converting it back to rectangular form to see that it simplifies to $z = \sqrt{3} - i$.

\item  For $z = -2+4i$, $\abs{z} = 2\sqrt{5}$ and $\theta = \pi - \arctan(2)$.  Hence, $z = 2\sqrt{5} \operatorname{cis}(\pi - \arctan(2))$.  It is a good exercise to actually show that this polar form reduces to $z=-2+4i$.

\item  For $z = 3i$, $\abs{z} = 3$ and $\theta = \frac{\pi}{2}$.  In this case, $z = 3 \operatorname{cis}\left(\frac{\pi}{2}\right)$.  This can be checked geometrically.  Head out $3$ units from $0$ along the positive real axis. Rotating $\frac{\pi}{2}$ radians counter-clockwise lands you exactly $3$ units above $0$ on the imaginary axis at $z = 3i$.

\item  Last but not least, for $z = -117$, $\abs{z} = 117$ and $\theta = \pi$. We get $z = 117 \operatorname{cis}(\pi)$. As with the previous problem, our answer is easily checked geometrically. 

\end{enumerate}

\end{enumerate}
}

\medskip

The following theorem summarizes the advantages of working with complex numbers in polar form.

\medskip

\theorem{prodquotpolarcomplex}{Products, Powers and Quotients Complex Numbers in Polar Form}{  Suppose $z$ and $w$ are complex numbers with polar forms $z = \abs{z}\operatorname{cis}(\alpha)$ and $w = \abs{w}\operatorname{cis}(\beta)$.  Then

\begin{itemize}

\item  \textbf{Product Rule:} $zw = \abs{z}\abs{w} \operatorname{cis}(\alpha + \beta)$ \index{product rule ! for complex numbers}

\item  \textbf{Power Rule} (\index{DeMoivre's Theorem}\textbf{\href{http://en.wikipedia.org/wiki/Abraham_de_Moivre}{\underline{DeMoivre's Theorem}}}) \textbf{:}  $z^{n} = \abs{z}^{n} \operatorname{cis}(n \theta)$ for every natural number $n$ \index{power rule ! for complex numbers}

\item  \textbf{Quotient Rule:} $\dfrac{z}{w} = \dfrac{\abs{z}}{\abs{w}} \operatorname{cis}(\alpha - \beta)$, provided $\abs{w} \neq 0$ \index{quotient rule ! for complex numbers}

\end{itemize}
}

\medskip

\mnote{.5}{While the notation $\operatorname{cis}(\theta) = \cos(\theta)+ i\sin(\theta)$ is not uncommon, most authors prefer to use Euler's exponential notation. In light of Theorem \ref{prodquotpolarcomplex}, one can make sense of the polar form using \sword{Euler's formula}
\[
e^{i\theta} = \cos(\theta)+i\sin(\theta).
\]
The appearance of the exponential function in this context might seem strange, but note that the three properties in Theorem \ref{prodquotpolarcomplex} can then be understood in terms of laws of exponents. If $z=re^{i\alpha}$ and $w=se^{i\beta}$, we have
\[
zw = (rs)\left(e^{i\alpha}e^{i\beta}\right) = (rs)e^{i(\alpha+\beta)},
\]
\[
z^n = r^n(e^{i\alpha})^n = r^ne^{in\alpha},
\]
and so on. For more details, see Exercise \ref{eulerformulaexercise}.}

The proof of Theorem \ref{prodquotpolarcomplex} requires a healthy mix of definition, arithmetic and identities.  We first start with the product rule.

\begin{align*}
zw & =  \left[|z|\operatorname{cis}(\alpha)\right] \left[|w|\operatorname{cis}(\beta)\right]  \\[3pt]
   & =  |z||w|\left[\cos(\alpha) + i\sin(\alpha)\right]\left[\cos(\beta) + i \sin(\beta)\right]
\end{align*}

We now focus on the quantity in brackets on the right hand side of the equation.

\begin{align*}
\left[\cos(\alpha) + i\sin(\alpha)\right]&\left[\cos(\beta) + i \sin(\beta)\right]\\
 & = \cos(\alpha)\cos(\beta) + i\cos(\alpha)\sin(\beta) \\	
 & \quad\quad + i\sin(\alpha)\cos(\beta) + i^2 \sin(\alpha)\sin(\beta) \\																		 & =  \cos(\alpha)\cos(\beta) +  i^2 \sin(\alpha) \sin(\beta) \tag*{Rearranging terms}  \\
 & \quad\quad + \, i\sin(\alpha)\cos(\beta) +i\cos(\alpha)\sin(\beta)  \\
 & =  \left(\cos(\alpha)\cos(\beta) - \sin(\alpha) \sin(\beta)\right)\tag*{Since $i^2 = -1$}\\
 &  \quad\quad + i\left(\sin(\alpha)\cos(\beta)+ \cos(\alpha)\sin(\beta)\right)  \tag*{Factor out $i$}\\	
 & = \cos(\alpha + \beta) + i \sin(\alpha+\beta)  \tag*{Sum identities} \\
 & = \operatorname{cis}(\alpha + \beta)  \tag*{Definition of `cis'}
\end{align*}

Putting this together with our earlier work, we get $zw = \abs{z} \abs{w} \operatorname{cis}(\alpha + \beta)$, as required.  

\smallskip

As with with the proof of the Power Rule in Theorem \ref{modprops}, the proof of the Power Rule (better known  as DeMoivre's Theorem) amounts to repeated application of the product rule. A formal proof requires the technique of \textit{mathematical induction} (usually first encountered in Math 2000), and is therefore omitted. 
\smallskip

The last property in Theorem \ref{prodquotpolarcomplex} to prove is the quotient rule.   Assuming $\abs{w} \neq 0$ we have

\begin{align*}
\dfrac{z}{w} & =  \dfrac{|z| \operatorname{cis}(\alpha)}{|w| \operatorname{cis}(\beta)}  \\[3pt]
			 & =  \left( \dfrac{|z|}{|w|}\right) \dfrac{\cos(\alpha) + i \sin(\alpha)}{\cos(\beta) + i \sin(\beta)}
\end{align*}
						 
Next, we multiply both the numerator and denominator of the right hand side by $(\cos(\beta) - i \sin(\beta))$ which is the complex conjugate of $(\cos(\beta) + i \sin(\beta))$ to get

\[
\dfrac{z}{w}	 =  \left( \dfrac{|z|}{|w|}\right) \dfrac{\cos(\alpha) + i \sin(\alpha)}{\cos(\beta) + i \sin(\beta)} \cdot \dfrac{\cos(\beta) - i \sin(\beta)}{\cos(\beta) - i \sin(\beta)} 
\]

If we let the numerator be $N = \left[\cos(\alpha) + i \sin(\alpha)\right] \left[\cos(\beta) - i \sin(\beta)\right]$ and simplify we get

\begin{align*}
N & = \left[\cos(\alpha) + i \sin(\alpha)\right] \left[\cos(\beta) - i \sin(\beta)\right]  \\ 
  & =  \cos(\alpha)\cos(\beta)-i\cos(\alpha)\sin(\beta)\\
  &\quad\quad\quad  + i \sin(\alpha)\cos(\beta) - i^2 \sin(\alpha)\sin(\beta)  \tag*{Expand} \\
	& =  \left[\cos(\alpha)\cos(\beta)+\sin(\alpha)\sin(\beta)\right]\\
	&\quad\quad\quad + i\left[\sin(\alpha)\cos(\beta) -\cos(\alpha)\sin(\beta)  \right]  \tag*{Rearrange and Factor} \\
	& = \cos(\alpha - \beta) + i \sin(\alpha - \beta)   \tag*{Difference Identities}  \\
	& =  \operatorname{cis}(\alpha - \beta) \tag*{Definition of `cis'} 
\end{align*}
	
\medskip

If we call the denominator $D$ then we get


\begin{align*}
D & =  \left[\cos(\beta) + i\sin(\beta)\right]\left[\cos(\beta) - i\sin(\beta)\right]  \\ 
  & =  \cos^{2}(\beta) - i\cos(\beta)\sin(\beta)\\
  &\quad\quad\quad  + i\cos(\beta)\sin(\beta) - i^2 \sin^{2}(\beta)  \tag*{Expand}  \\
  & =  \cos^{2}(\beta) - i^2 \sin^{2}(\beta)  \tag*{Simplify}  \\
  & =  \cos^{2}(\beta) + \sin^{2}(\beta)   \tag*{Again, $i^{2} = -1$}\\
  & =  	1   \tag*{Pythagorean Identity}
\end{align*}


\medskip
																										 
Putting it all together, we get

\begin{align*}
\dfrac{z}{w} & =  \left( \dfrac{|z|}{|w|}\right) \dfrac{\cos(\alpha) + i \sin(\alpha)}{\cos(\beta) + i \sin(\beta)} \cdot \dfrac{\cos(\beta) - i \sin(\beta)}{\cos(\beta) - i \sin(\beta)}  \\[3pt]
						 & =  \left( \dfrac{|z|}{|w|}\right) \dfrac{\operatorname{cis}(\alpha - \beta)}{1}    \\[3pt]	 
						 & =   \dfrac{|z|}{|w|} \operatorname{cis}(\alpha - \beta)
\end{align*}
and we are done.  The next example makes good use of Theorem \ref{prodquotpolarcomplex}.

\medskip

\example{polararithmeticex}{Complex arithmetic using the polar form}{ Let $z = 2\sqrt{3} + 2i$ and $w = -1 + i\sqrt{3}$.  Use Theorem \ref{prodquotpolarcomplex} to find the following.

\begin{multicols}{3}

\begin{enumerate}

\item $zw$

\item  $w^5$

\item $\dfrac{z}{w}$

\end{enumerate}

\end{multicols}

Write your final answers in rectangular form.}
{In order to use Theorem \ref{prodquotpolarcomplex}, we need to write $z$ and $w$ in polar form.  For $z=2\sqrt{3} + 2i$, we find 
\[
\abs{z} = \sqrt{(2\sqrt{3})^2 + (2)^2} = \sqrt{16} = 4.
\]
We then have
\[
z = 4\left(\frac{1}{4}(2\sqrt{3}+2i)\right) = 4\left(\frac{\sqrt{3}}{2}+i\frac{1}{2}\right) = 4\operatorname{cis}\left(\frac{\pi}{6}\right),
\]
using the fact that the point $\left(\frac{\sqrt{3}}{2},\frac{1}{2}\right)$ on the unit circle corresponds to the angle $\theta = \frac{\pi}{6}$. For $w = -1 + i\sqrt{3}$, we have $\abs{w} = \sqrt{(-1)^2+(\sqrt{3})^2} = 2$, and thus, using the same method that we did for $z$, we have
\[
w = 2\left(\frac{1}{2}(-1+i\sqrt{3})\right) = 2\left(-\frac{1}{2}+i\frac{\sqrt{3}}{2}\right) = 2\operatorname{cis}\left(\frac{2\pi}{3}\right).
\]
  We can now proceed.

\begin{enumerate}

\item  We get $zw = \left(4 \operatorname{cis}\left(\frac{\pi}{6}\right)\right) \left(2\operatorname{cis}\left(\frac{2\pi}{3}\right)\right) = 8\operatorname{cis}\left(\frac{\pi}{6} + \frac{2\pi}{3}\right) = 8\operatorname{cis}\left(\frac{5\pi}{6}\right) = 8\left[ \cos\left(\frac{5\pi}{6}\right) + i\sin\left(\frac{5\pi}{6}\right) \right]$.  After simplifying, we get $zw = -4\sqrt{3} + 4i$.

\item We use DeMoivre's Theorem which yields 
\[
w^{5} = \left[2\operatorname{cis}\left(\frac{2\pi}{3}\right)\right]^{5} = 2^{5} \operatorname{cis} \left(5\cdot \frac{2\pi}{3}\right) = 32 \operatorname{cis}\left(\frac{10\pi}{3}\right).
\]
Since $\frac{10\pi}{3}$ is coterminal with $\frac{4\pi}{3}$, we get
\[
w^{5} = 32\left[ \cos\left(\frac{4\pi}{3}\right) + i\sin\left(\frac{4\pi}{3}\right) \right] = -16-16i\sqrt{3}.
\]

\item  Last, but not least, we have $\dfrac{z}{w} = \frac{4 \operatorname{cis}\left(\frac{\pi}{6}\right)}{2\operatorname{cis}\left(\frac{2\pi}{3}\right)} = \frac{4}{2} \operatorname{cis}\left(\frac{\pi}{6} - \frac{2\pi}{3}\right) = 2\operatorname{cis}\left(-\frac{\pi}{2}\right)$.  Since the angle $-\frac{\pi}{2}$ lies along the negative $y$-axis, we can `see' the rectangular form by moving out $2$ units along the positive real axis, then rotating $\frac{\pi}{2}$ radians \textit{clockwise} to arrive at the point $2$ units below $0$ on the imaginary axis.  The long and short of it is that $\frac{z}{w} = -2i$. 
\end{enumerate}
}

\medskip

Some remarks are in order.  First, the reader may not be sold on using the polar form of complex numbers to multiply complex numbers -- especially if they aren't given in polar form to begin with. Indeed, a lot of work was needed to convert the numbers $z$ and $w$ in Example \ref{polararithmeticex} into polar form, compute their product, and convert back to rectangular form -- certainly more work than is required to multiply out $zw =  (2\sqrt{3} + 2i)(-1 + i\sqrt{3})$ the old-fashioned way.  However, Theorem \ref{prodquotpolarcomplex} pays huge dividends when computing powers of complex numbers.  Consider how we computed $w^{5}$ above and compare that to using the Binomial Theorem to accomplish the same feat by expanding  $(-1 + i\sqrt{3})^{5}$.  Division is tricky in the best of times, and we saved ourselves a lot of time and effort using Theorem \ref{prodquotpolarcomplex} to find and simplify $\frac{z}{w}$ using their polar forms as opposed to starting with $\frac{2\sqrt{3} + 2i}{-1 + i\sqrt{3}}$, rationalizing the denominator, and so forth.    

\smallskip

There is geometric reason for studying these polar forms and we would be derelict in our duties if we did not mention the Geometry hidden in Theorem \ref{prodquotpolarcomplex}.  Take the product rule, for instance. If $z = \abs{z} \operatorname{cis}(\alpha)$ and $w = \abs{w} \operatorname{cis}(\beta)$, the formula $zw = \abs{z}\abs{w} \operatorname{cis}(\alpha + \beta)$ can be viewed geometrically as a two step process.  The multiplication of $\abs{z}$ by $\abs{w}$ can be interpreted as magnifying the distance $\abs{z}$ from $z$ to $0$, by the factor $\abs{w}$.  (Assuming $\abs{w} > 1$.) Adding the argument of $w$ to the argument of $z$ can be interpreted geometrically as a rotation of $\beta$ radians counter-clockwise. (Assuming $\beta > 0$.) Focusing on $z$ and $w$ from Example \ref{polararithmeticex}, we can arrive at the product $zw$ by plotting $z$, doubling its distance from $0$ (since $\abs{w} = 2$), and rotating $\frac{2\pi}{3}$ radians counter-clockwise. The sequence of diagrams in Figure \ref{fig:polarcomp} below attempts to describe this process geometrically.

\medskip

\noindent\ifthenelse{\isodd{\thepage}}{\hskip-0.9\marginparwidth}{}
\noindent\begin{minipage}{\textwidth+\marginparwidth}
\begin{center}
\begin{tabular}{cc}
\myincludegraphics{figures/PolarComplex-4}
& \myincludegraphics{figures/PolarComplex-5}\\
{\scriptsize Multiplying $z$ by $\abs{w} = 2$}. &
{\scriptsize Rotating counter-clockwise by $\operatorname{Arg}(w) = \frac{2\pi}{3}$ radians.} \\
\end{tabular}
\end{center}
\captionsetup{type=figure}
\caption{Visualizing $zw$ for $z = 4\operatorname{cis}\left(\frac{\pi}{6}\right)$ and $w = 2 \operatorname{cis}\left(\frac{2\pi}{3}\right)$.}\label{fig:polarcomp}
\end{minipage}

\medskip

We may also visualize division similarly. Here, the formula  $\frac{z}{w} = \frac{\abs{z}}{\abs{w}} \operatorname{cis}(\alpha - \beta)$ may be interpreted as shrinking (again, assuming $\abs{w} > 1$) the distance from $0$ to $z$ by the factor $\abs{w}$, followed up by a \textit{clockwise}  rotation (again, assuming $\beta > 0$) of $\beta$ radians.  In the case of $z$ and $w$ from Example \ref{polararithmeticex}, we arrive at $\frac{z}{w}$ by first halving the distance from $0$ to $z$, then rotating clockwise $\frac{2\pi}{3}$ radians.

\medskip
%\noindent\ifthenelse{\isodd{\thepage}}{}{\hskip-\marginparwidth}
\noindent\begin{minipage}{\textwidth}
\begin{center}
\begin{tabular}{cc}
\myincludegraphics{figures/PolarComplex-6} & 
\myincludegraphics{figures/PolarComplex-7}\\
{\scriptsize Dividing $z$ by $\abs{w} = 2$}. &
{\scriptsize Rotating clockwise by $\operatorname{Arg}(w) = \frac{2\pi}{3}$ radians.} \\
\end{tabular}
\end{center}
\captionsetup{type=figure}
\caption{Visualizing $\dfrac{z}{w}$ for $z = 4\operatorname{cis}\left(\frac{\pi}{6}\right)$ and $w = 2 \operatorname{cis}\left(\frac{2\pi}{3}\right)$.}
\end{minipage}

\medskip

Our last goal of the section is to reverse DeMoivre's Theorem to extract roots of complex numbers.

\smallskip

\definition{nthrootcomplex}{Complex $n^{\textrm{th}}$ roots}{ Let $z$ and $w$ be complex numbers.  If there is a natural number $n$ such that $w^{n} = z$, then $w$ is an $n^{\textrm th}$ \sword{root} of $z$. \index{$n^{\textrm{th}}$ root ! of a complex number} \index{complex number ! $n^{\textrm{th}}$ root} 
}

\smallskip

\mnote{.3}{Recall that when taking an \textit{even} root of a positive \textit{real} number, there are two possible values: a positive root and a negative root. The principal root is taken to be the positive value. On the other hand, for \textit{odd} roots of real numbers, there is only ever one possible value.}

Unlike with real numbers, we do not specify one particular \textit{prinicpal} $n^{\text{th}}$ root, hence the use of the indefinite article `an' as in `an $n^{\text{th}}$ root of $z$'.  Using this definition, both $4$ and $-4$ are square roots of $16$, while $\sqrt{16}$ means the principal square root of $16$ as in $\sqrt{16}= 4$.  Suppose we wish to find all complex third (cube) roots of $8$.  Algebraically, we are trying to solve $w^{3} = 8$.  We know that there is only one \textit{real} solution to this equation, namely $w = \sqrt[3]{8} = 2$, but if we take the time to rewrite this equation as $w^3 - 8 = 0$ and factor, we get $(w-2)\left(w^2 + 2w + 4\right) = 0$.  The quadratic factor gives two more cube roots $w = -1 \pm i \sqrt{3}$, for a total of three cube roots of $8$. In accordance with Theorem \ref{complexfactorization}, since the degree of $p(w) = w^3 -8$ is three, there are three complex zeros, counting multiplicity.  Since we have found three distinct zeros, we know these are all of the zeros, so there are exactly three distinct cube roots of $8$.  Let us now solve this same problem using the machinery developed in this section.  To do so, we express $z = 8$ in polar form. Since $z=8$ lies $8$ units away on the positive real axis, we get $z = 8 \operatorname{cis}(0)$.  If we let $w = \abs{w} \operatorname{cis}(\alpha)$ be a polar form of $w$, the equation $w^3 = 8$ becomes
\begin{align*}
w^3 & =  8  \\
\left(|w| \operatorname{cis}(\alpha)\right)^3 & = 8 \operatorname{cis}(0)  \\
|w|^3 \operatorname{cis}(3\alpha) & =  8 \operatorname{cis}(0)  \tag*{DeMoivre's Theorem} \\
\end{align*}

The complex number on the left hand side of the equation corresponds to the point with polar coordinates $\left(\abs{w}^3, 3\alpha\right)$,   while the complex number on the right hand side corresponds to the point with polar coordinates $(8,0)$.  Since $\abs{w} \geq 0$, so is $\abs{w}^3$, which means  $\left(\abs{w}^3, 3\alpha\right)$ and $(8,0)$ are two polar representations corresponding to the same complex number, both with positive $r$ values.  From Section \ref{IntroPolar}, we know $\abs{w}^3 = 8$ and $3\alpha = 0 + 2\pi k$ for integers $k$.  Since $\abs{w}$ is a real number, we solve $\abs{w}^3 = 8$ by extracting the principal cube root to get  $\abs{w} = \sqrt[3]{8} = 2$.  As for $\alpha$, we get  $\alpha = \frac{2\pi k}{3}$ for integers $k$.  This produces three distinct points with polar coordinates corresponding to $k = 0$, $1$ and $2$: specifically $(2,0)$, $\left(2, \frac{2\pi}{3}\right)$ and $\left(2, \frac{4\pi}{3}\right)$.  These correspond to the complex numbers  $w_{0} = 2 \operatorname{cis}(0)$, $w_{1} = 2 \operatorname{cis}\left(\frac{2\pi}{3}\right)$ and $w_{2} = 2 \operatorname{cis}\left(\frac{4\pi}{3}\right)$, respectively.  Writing these out in rectangular form yields $w_{0} = 2$, $w_{1} = -1 + i\sqrt{3}$ and $w_{2} = -1-i\sqrt{3}$. While this process seems a tad more involved than our previous factoring approach, this procedure can be generalized to find, for example, all of the fifth roots of $32$. If we start with a generic complex number in polar form $z = \abs{z} \operatorname{cis}(\theta)$ and solve $w^{n} = z$ in the same manner as above, we arrive at the following theorem.

\smallskip

\theorem{nthrootscomplexthm}{The $n^{\textrm{th}}$ roots of a complex number}{ Let $z \neq 0$ be a complex number with polar form $z = r\operatorname{cis}(\theta)$. For each natural number $n$, $z$ has $n$ distinct $n^{\text{th}}$ roots, which we denote by $w_{0}$, $w_{1}$, \ldots, $w_{n-1}$, and they are given by the formula \index{$n^{\textrm{th}}$ root ! of a complex number} \index{complex number ! $n^{\textrm{th}}$ root} 

\[
 w_{k} = \sqrt[n]{r}\operatorname{cis}\left(\frac{\theta}{n} + \frac{2\pi}{n} k \right) 
\]
}

\smallskip

The proof of Theorem \ref{nthrootscomplexthm} breaks into to two parts:  first, showing that each $w_{k}$ is an $n^{\text{th}}$ root, and second, showing that the set  $\left\{ w_{k} \, | \, k = 0, 1, \ldots, (n-1)\right\}$ consists of $n$ different  complex numbers.  To show $w_{k}$ is an  $n^{\text{th}}$ root of $z$, we use DeMoivre's Theorem to show $\left(w_{k}\right)^n = z$.

\begin{align*}
\left(w_{k}\right)^n & =  \left(\sqrt[n]{r}\operatorname{cis}\left(\frac{\theta}{n} + \frac{2\pi}{n} k \right)\right)^n  \\ 																	 & =  \left(\sqrt[n]{r}\right)^{n} \operatorname{cis}\left( n \cdot \left[\frac{\theta}{n} + \frac{2\pi}{n} k \right] \right)  \tag*{DeMoivre's Theorem}\\ 													 & =  r \operatorname{cis}\left(\theta + 2\pi k\right)
\end{align*}
																	 
Since $k$ is a whole number, $\cos(\theta + 2\pi k) = \cos(\theta)$ and $\sin(\theta + 2\pi k) = \sin(\theta)$. Hence, it follows that $\operatorname{cis}(\theta + 2\pi k) = \operatorname{cis}(\theta)$, so $\left(w_{k}\right)^n = r \operatorname{cis}(\theta) = z$, as required.  To show that the formula in Theorem  \ref{nthrootscomplexthm} generates $n$ distinct numbers, we assume $n \geq 2$ (or else there is nothing to prove) and note that the modulus of each of the  $w_{k}$  is the same, namely $\sqrt[n]{r}$.  Therefore, the only way any two of these polar forms correspond to  the same number is if their arguments are coterminal -- that is, if the arguments differ by an integer multiple of $2\pi$. Suppose $k$ and $j$ are whole numbers between $0$ and $(n-1)$, inclusive,  with $k \neq j$.  Since $k$ and $j$ are different, let's assume for the sake of argument that  $k > j$.  Then $\left( \frac{\theta}{n} + \frac{2\pi}{n} k \right)  - \left( \frac{\theta}{n} + \frac{2\pi}{n} j \right) = 2\pi \left(\frac{k-j}{n}\right)$. For this to be an integer multiple of $2\pi$, $(k-j)$ must be a multiple of $n$.  But because of the restrictions on $k$ and $j$, $0 < k - j \leq n-1$. (Think this through.)  Hence, $(k-j)$ is a positive number less  than $n$, so it cannot be a multiple of $n$.  As a result, $w_{k}$ and $w_{j}$ are different complex numbers, and we are done.  By Theorem  \ref{complexfactorization}, we know there at most $n$ distinct solutions to $w^{n} = z$, and we have just found all of them.   We illustrate Theorem \ref{nthrootscomplexthm} in the next example.

\medskip

\example{nthrootscomplexex}{Finding complex roots}{  Use Theorem \ref{nthrootscomplexthm} to find the following:

\begin{enumerate}

\item  both square roots of $z = -2  + 2i\sqrt{3}$

\item  \label{fourthrootsneg16} the four fourth roots of $z = -16$

\item  \label{halfanglecuberoot} the three cube roots of $z = \sqrt{2} + i \sqrt{2}$

\item  \label{calculatorfifthroot} the five fifth roots of $z = 1$.

\end{enumerate}}
{
\begin{enumerate}

\item  We start by writing $z= - 2 + 2i\sqrt{3} = 4 \operatorname{cis}\left(\frac{2\pi}{3}\right)$.  To use Theorem \ref{nthrootscomplexthm}, we identify $r =4$,  $\theta = \frac{2\pi}{3}$ and $n=2$.  We know that $z$ has two square roots, and in keeping with the notation in Theorem \ref{nthrootscomplexthm}, we'll call them  $w_{0}$ and $w_{1}$.  We get $w_{0} = \sqrt{4} \operatorname{cis}\left(\frac{(2\pi/3)}{2} + \frac{2\pi}{2} (0)\right) = 2\operatorname{cis}\left(\frac{\pi}{3}\right)$ and $w_{1} = \sqrt{4} \operatorname{cis}\left(\frac{(2\pi/3)}{2} + \frac{2\pi}{2} (1)\right) = 2\operatorname{cis}\left(\frac{4\pi}{3}\right)$.  In rectangular form, the two square roots of $z$ are $w_{0} = 1+i\sqrt{3}$ and $w_{1} = -1-i\sqrt{3}$.  We can check our answers by squaring them and showing that we get $z= -2 + 2i\sqrt{3}$. We've plotted the position of the two square roots along the circle $r=2$ in Figure \ref{fig:polarroot1}.

\mfigure{.25}{The two square roots of $z=-2+2\sqrt{3}i$}{fig:polarroot1}{figures/PolarComplex01}

\item  Proceeding as above, we get $z = -16 = 16 \operatorname{cis}(\pi)$.  With $r = 16$, $\theta = \pi$ and $n = 4$, we get the four fourth roots of $z$ to be  
\begin{align*}
w_{0} &= \sqrt[4]{16} \operatorname{cis}\left(\frac{\pi}{4} + \frac{2\pi}{4} (0)\right) = 2\operatorname{cis}\left(\frac{\pi}{4}\right)\\
w_{1} &= \sqrt[4]{16} \operatorname{cis}\left(\frac{\pi}{4} + \frac{2\pi}{4} (1)\right) = 2\operatorname{cis}\left(\frac{3\pi}{4}\right)\\
w_{2} &= \sqrt[4]{16} \operatorname{cis}\left(\frac{\pi}{4} + \frac{2\pi}{4} (2)\right) = 2\operatorname{cis}\left(\frac{5\pi}{4}\right)\\
w_{3} &= \sqrt[4]{16} \operatorname{cis}\left(\frac{\pi}{4} + \frac{2\pi}{4} (3)\right) = 2\operatorname{cis}\left(\frac{7\pi}{4}\right).
\end{align*}
Converting these to rectangular form gives $w_{0} = \sqrt{2} + i\sqrt{2}$,  $w_{1} = -\sqrt{2} + i\sqrt{2}$,  $w_{2} = -\sqrt{2} - i\sqrt{2}$ and  $w_{3} = \sqrt{2} - i\sqrt{2}$. We've plotted the four roots in Figure \ref{fig:polarroot2}. Note how the roots are placed symmetrically about the circle $r=2$.

\mfigure{.8}{The four fourth roots of $z=-16$}{fig:polarroot2}{figures/PolarComplex02}

\item  For $z = \sqrt{2} + i \sqrt{2}$, we have $z = 2\operatorname{cis}\left(\frac{\pi}{4}\right)$.  With $r = 2$, $\theta = \frac{\pi}{4}$ and $n =3$ the usual computations yield $w_{0} = \sqrt[3]{2} \operatorname{cis}\left(\frac{\pi}{12}\right)$,  $w_{1} = \sqrt[3]{2} \operatorname{cis}\left(\frac{9\pi}{12}\right) = \sqrt[3]{2} \operatorname{cis}\left(\frac{3\pi}{4}\right) $ and  $w_{2} = \sqrt[3]{2} \operatorname{cis}\left(\frac{17\pi}{12}\right)$.  If we were to  convert these to rectangular form, we would need to use either the Sum and Difference Identities or the Half-Angle Identities to evaluate $w_{0}$ and  $w_{2}$.  Since we are not explicitly told to do so, we leave this as a good, but messy, exercise, and plot the points in Figure \ref{fig:polarroot3}.

\mfigure{.55}{The three third roots of $z=\sqrt{2}+i\sqrt{2}$}{fig:polarroot3}{figures/PolarComplex03}

\item  To find the five fifth roots of $1$, we write $1 = 1 \operatorname{cis}(0)$.  We have $r = 1$, $\theta = 0$ and $n = 5$. Since $\sqrt[5]{1} = 1$, the roots are  $w_{0} = \operatorname{cis}(0) = 1$, $w_{1} = \operatorname{cis}\left(\frac{2\pi}{5}\right)$, $w_{2} = \operatorname{cis}\left(\frac{4\pi}{5}\right)$, $w_{3} = \operatorname{cis}\left(\frac{6\pi}{5}\right)$ and $w_{4} = \operatorname{cis}\left(\frac{8\pi}{5}\right)$.  The situation here is even graver than in the previous example, since we have not developed any identities to help us determine the cosine or sine of $\frac{2\pi}{5}$.  At this stage, we could approximate our answers using a calculator, and we leave this as an exercise. Once more, we plot the roots, which in this case all lie on the unit circle.

\mfigure{.3}{The five fifth roots of $1$}{fig:polarroot4}{figures/PolarComplex04}

\end{enumerate}
}

\medskip

Notice the geometric interpretation given in Figures \ref{fig:polarroot1}-\ref{fig:polarroot4}.  Essentially,  Theorem \ref{nthrootscomplexthm} says that to find the $n^{\text{th}}$ roots of a complex number,  we first take the $n^{\text{th}}$ root of the modulus and divide the argument by $n$.  This gives the first root  $w_{0}$. Each successive root is found by adding  $\frac{2\pi}{n}$ to the argument, which amounts to rotating $w_{0}$ by $\frac{2\pi}{n}$ radians.  This results in $n$ roots, spaced equally around the complex plane.  

We have only glimpsed at the beauty of the complex numbers in this section.  The complex plane is without a doubt one of the most important mathematical constructs ever devised.  Coupled with Calculus, it is the venue for incredibly important Science and Engineering applications.  For now, the following exercises will have to suffice.

\printexercises{exercises/08_05_exercises}
