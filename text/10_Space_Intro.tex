This chapter introduces a new mathematical object, the \sword{vector}. Defined in Section \ref{sec:vector_intro}, we will see that vectors provide a powerful language for describing quantities that have magnitude and direction aspects. A simple example of such a quantity is force: when applying a force, one is generally interested in how much force is applied (i.e., the magnitude of the force) and the direction in which the force was applied. Vectors will play an important role in many of the subsequent chapters in this text. 

This chapter begins with moving our mathematics out of the plane and into ``space.'' That is, we begin to think mathematically not only in two dimensions, but in three. With this foundation, we can explore vectors both in the plane and in space. 

\section{Introduction to Cartesian Coordinates in Space}\label{sec:space_coord}

Up to this point in this text we have considered mathematics in a 2--dimensional world. We have plotted graphs on the $x$-$y$ plane using rectangular and polar coordinates and found the area of regions in the plane. We have considered properties of \textit{solid} objects, such as volume and surface area, but only by first defining a curve in the plane and then rotating it out of the plane.

While there is wonderful mathematics to explore in ``2D,'' we live in a ``3D'' world and eventually we will want to apply mathematics involving this third dimension. In this section we introduce Cartesian coordinates in space and explore basic surfaces. This will lay a foundation for much of what we do in the remainder of the text.\\

Each point $P$ in space can be represented with an ordered triple, $P=(a,b,c)$, where $a$, $b$ and $c$ represent the relative position of $P$ along the $x$-, $y$- and $z$-axes, respectively. Each axis is perpendicular to the other two.

Visualizing points in space on paper can be problematic, as we are trying to represent a 3-dimensional concept on a 2--dimensional medium. We cannot draw three lines representing the three axes in which each line is perpendicular to the other two. Despite this issue, standard conventions exist for plotting shapes in space that we will discuss that are more than adequate.

One convention is that the axes must conform to the \sword{right hand rule}. This rule states that when the index finger of the right hand is extended in the direction of the positive $x$-axis, and the middle finger (bent ``inward'' so it is perpendicular to the palm) points along the positive $y$-axis, then the extended thumb will point in the direction of the positive $z$-axis. (It may take some thought to verify this, but this system is inherently different from the one created by using the ``left hand rule.'')\index{right hand rule!of Cartesian coordinates}

As long as the coordinate axes are positioned so that they follow this rule, it does not matter how the axes are drawn on paper. There are two popular methods that we briefly discuss.
\mfigurethree{width=150pt,3Dmenu,activate=onclick,deactivate=onclick,
3Droll=0,
3Dortho=0.0045,
3Dc2c=11 5 2.8,
3Dcoo=0 0 0,
3Droo=200,
3Dlights=Headlamp,add3Djscript=asylabels.js}{}{.78}{Plotting the point $P=(2,1,3)$ in space.}%
{fig:cartcoord1}{figures/figcartcoord1}%
%{\mfigure{.4}{Plotting the point $P=(2,1,3)$ in space.}{fig:cartcoord1}{figures/figcartcoord1}


In Figure \ref{fig:cartcoord1} we see the point $P=(2,1,3)$ plotted on a set of axes. The basic convention here is that the $x$-$y$ plane is drawn in its standard way, with the $z$-axis down to the left. The perspective  is that the paper represents the $x$-$y$ plane and the positive $z$ axis is coming up, off the page. This method is preferred by many engineers. Because it can be hard to tell where a single point lies in relation to all the axes, dashed lines have been added to let one see how far along each axis the point lies.

One can also consider the $x$-$y$ plane as being a horizontal plane in, say, a room, where the positive $z$-axis is pointing up. When one steps back and looks at this room, one might draw the axes as shown in Figure \ref{fig:cartcoord2}. The same point $P$ is drawn, again with dashed lines. This point of view is preferred by most mathematicians, and is the convention adopted by this text.
\mfigurethree{width=150pt,3Dmenu,activate=onclick,deactivate=onclick,
3Droll=0,
3Dortho=0.004,
3Dc2c=0.6666666865348816 0.6666666865348816 0.3333333730697632,
3Dcoo=16.7497615814209 8.84995174407959 40.36832046508789,
3Droo=129.79413580025474,
3Dlights=Headlamp,add3Djscript=asylabels.js}{scale=1.5,trim=5mm 5mm 3mm 5mm,clip=true}{.5}{Plotting the point $P=(2,1,3)$ in space with a perspective used in this text.}%
{fig:cartcoord2}{figures/figcartcoord2}%
%{\mtable{.8}{Plotting the point $P=(2,1,3)$ in space with a perspective used in this text.}{fig:cartcoord2}{\myincludegraphics[scale=1.5,trim=5mm 5mm 3mm 5mm,clip=true]{figures/figcartcoord2}}
\\

\noindent\textbf{\large Measuring Distances}\\

It is of critical importance to know how to measure distances between points in space. The formula for doing so is based on measuring distance in the plane, and is known (in both contexts) as the Euclidean measure of distance.

\definition{def:space_distance}{Distance In Space}
{Let $P=(x_1,y_1,z_1)$ and $Q = (x_2,y_2,z_2)$ be points in space. The distance $D$ between $P$ and $Q$ is \index{distance!between points in space}
$$D = \sqrt{(x_2-x_1)^2+(y_2-y_1)^2+(z_2-z_1)^2}.$$
}

We refer to the line segment that connects points $P$ and $Q$ in space as $\overline{PQ}$, and refer to the length of this segment as $||\overline{PQ}||$. The above distance formula allows us to compute the length of this segment.\\

\example{ex_space1}{Length of a line segment}{
Let $P = (1,4,-1)$ and let $Q = (2,1,1)$. Draw the line segment $\overline{PQ}$ and find its length.}
{The points $P$ and $Q$ are plotted in Figure \ref{fig:space1}; no special consideration need be made to draw the line segment connecting these two points; simply connect them with a straight line. One \textit{cannot} actually measure this line on the page and deduce anything meaningful; its true length must be measured analytically. Applying Definition \ref{def:space_distance}, we have
%\ifthenelse{\boolean{in_threeD}}{%
%\mfigurethree[width=150pt,3Dmenu,activate=onclick,deactivate=onclick,
%3Droll=0,
%3Dortho=0.0044,
%3Dc2c=0.6474442481994629 0.6759132146835327 0.35207563638687134,
%3Dcoo=32.306640625 39.99113082885742 5.668694019317627,
%3Droo=113.20770262248418,
%3Dlights=Headlamp,add3Djscript=asylabels.js]{.4}{Plotting points $P$ and $Q$ in Example \ref{ex_space1}.}%
%{fig:space1}{figures/figspace1_3D}}%
%{\mfigure[scale=1.25,trim=1mm 5mm 2mm 2mm,clip=true]{.4}{Plotting points $P$ and $Q$ in Example \ref{ex_space1}.}{fig:space1}{figures/figspace1}}
$$||\overline{PQ}|| = \sqrt{(2-1)^2+(1-4)^2+(1-(-1))^2} = \sqrt{14}\approx 3.74.$$
\mfigurethree{width=150pt,3Dmenu,activate=onclick,deactivate=onclick,
3Droll=0,
3Dortho=0.0044,
3Dc2c=0.6474442481994629 0.6759132146835327 0.35207563638687134,
3Dcoo=32.306640625 39.99113082885742 5.668694019317627,
3Droo=113.20770262248418,
3Dlights=Headlamp,add3Djscript=asylabels.js}{scale=1.25,trim=1mm 5mm 2mm 2mm,clip=true}{.77}{Plotting points $P$ and $Q$ in Example \ref{ex_space1}.}%
{fig:space1}{figures/figspace1}
\vskip-1.5\baselineskip
}\\

\noindent\textbf{\large Spheres}\\

\index{sphere}Just as a circle is the set of all points in the \textit{plane} equidistant from a given point (its center), a sphere is the set of all points in \textit{space} that are equidistant from a given point. Definition \ref{def:space_distance} allows us to write an equation of the sphere.

We start with a point $C = (a,b,c)$ which is to be the center of a sphere with radius $r$. If a point $P=(x,y,z)$ lies on the sphere, then $P$ is $r$ units from $C$; that is, 
$$||\overline{PC}|| = \sqrt{(x-a)^2+(y-b)^2+(z-c)^2} = r.$$
Squaring both sides, we get the standard equation of a sphere in space with center at $C=(a,b,c)$ with radius $r$, as given in the following Key Idea.

\keyidea{idea:sphere}{Standard Equation of a Sphere in Space}
{The standard equation of the sphere with radius $r$, centered at $C=(a,b,c)$, is
$$(x-a)^2+(y-b)^2+(z-c)^2=r^2.$$
}

\example{ex_space2}{Equation of a sphere}{
Find the center and radius of the sphere defined by $x^2+2x+y^2-4y+z^2-6z=2.$
}
{To determine the center and radius, we must put the equation in standard form. This requires us to complete the square (three times).
\begin{align*}
x^2+2x+y^2-4y+z^2-6z&=2 \\
(x^2+2x+1) + (y^2-4y+4)+ (z^2-6z+9) - 14 &= 2\\
(x+1)^2 + (y-2)^2 + (z-3)^2 &= 16
\end{align*}
The sphere is centred at $(-1,2,3)$ and has a radius of 4.
}\\

The equation of a sphere is an example of an implicit function defining a surface in space. In the case of a sphere, the variables $x$, $y$ and $z$ are all used. We now consider situations where surfaces are defined where one or two of these variables are absent.\\

\noindent\textbf{\large Introduction to Planes in Space}\\

The coordinate axes naturally define three planes (shown in Figure \ref{fig:coordplanes}), the \textbf{coordinate planes}: the $x$-$y$ plane, the $y$-$z$ plane and the $x$-$z$ plane. The $x$-$y$ plane is characterized as the set of all points in space where the $z$-value is 0. %(Likewise, the $x$-$z$ plane is all points where the $y$-value is 0.) 
\index{planes!coordinate plane}\index{planes!introduction}
This, in fact, gives us an equation that describes this plane: $z=0$. Likewise, the $x$-$z$ plane is all points where the $y$-value is 0, characterized by $y=0$.\\

%\mtable{.67}{The coordinate planes.}{fig:coordplanes}{%
%\begin{tabular}{c}
%\myincludegraphicsthree{width=100pt,3Dmenu,activate=onclick,deactivate=onclick,
%3Droll=0,
%3Dortho=0.004,
%3Dc2c=4 4 2,
%3Dcoo=0 0 0,
%3Droo=150,
%3Dlights=Headlamp,add3Djscript=asylabels.js}{scale=1.25,trim=4mm 5mm 4mm 5mm,clip=true}{figures/figspacexy}\\[-0pt]
%%\myincludegraphics[scale=1.25,trim=4mm 5mm 4mm 5mm,clip=true]{figures/figspacexy}\\[-10pt]
%the $x$-$y$ plane\\[5pt]
%\myincludegraphicsthree{width=100pt,3Dmenu,activate=onclick,deactivate=onclick,
%3Droll=0,
%3Dortho=0.004,
%3Dc2c=4 4 2,
%3Dcoo=0 0 0,
%3Droo=150,
%3Dlights=Headlamp,add3Djscript=asylabels.js}{scale=1.25,trim=4mm 5mm 4mm 5mm,clip=true}{figures/figspaceyz}\\[-0pt]
%the $y$-$z$ plane\\[5pt]
%\myincludegraphicsthree{width=100pt,3Dmenu,activate=onclick,deactivate=onclick,
%3Droll=0,
%3Dortho=0.004,
%3Dc2c=4 4 2,
%3Dcoo=0 0 0,
%3Droo=150,
%3Dlights=Headlamp,add3Djscript=asylabels.js}{scale=1.25,trim=4mm 5mm 4mm 5mm,clip=true}{figures/figspacexz}\\[-0pt]
%the $x$-$z$ plane
%\end{tabular} 
%}
\noindent\begin{minipage}{\textwidth}
\begin{tabular}{ccc}
\myincludegraphicsthree{width=95pt,3Dmenu,activate=onclick,deactivate=onclick,
3Droll=0,
3Dortho=0.004,
3Dc2c=4 4 2,
3Dcoo=0 0 0,
3Droo=150,
3Dlights=Headlamp,add3Djscript=asylabels.js}{scale=1.25,trim=4mm 5mm 4mm 5mm,clip=true}{figures/figspacexy}&
\myincludegraphicsthree{width=95pt,3Dmenu,activate=onclick,deactivate=onclick,
3Droll=0,
3Dortho=0.004,
3Dc2c=4 4 2,
3Dcoo=0 0 0,
3Droo=150,
3Dlights=Headlamp,add3Djscript=asylabels.js}{scale=1.25,trim=4mm 5mm 4mm 5mm,clip=true}{figures/figspaceyz}&
\myincludegraphicsthree{width=95pt,3Dmenu,activate=onclick,deactivate=onclick,
3Droll=0,
3Dortho=0.004,
3Dc2c=4 4 2,
3Dcoo=0 0 0,
3Droo=150,
3Dlights=Headlamp,add3Djscript=asylabels.js}{scale=1.25,trim=4mm 5mm 4mm 5mm,clip=true}{figures/figspacexz}\\
%\myincludegraphics[scale=1.25,trim=4mm 5mm 4mm 5mm,clip=true]{figures/figspacexy}\\[-10pt]
the $x$-$y$ plane &
the $y$-$z$ plane &
the $x$-$z$ plane
\end{tabular}
\captionsetup{type=figure}%
\caption{The coordinate planes.}\label{fig:coordplanes}
\end{minipage}
\\

The equation $x=2$ describes all points in space where the $x$-value is 2. This is a plane, parallel to the $y$-$z$ coordinate plane, shown in Figure \ref{fig:space2}.\\

\mfigurethree{width=100pt,3Dmenu,activate=onclick,deactivate=onclick,
3Droll=0,
3Dortho=0.0044,
3Dc2c=4 4 2,
3Dcoo=0 0 0,
3Droo=150,
3Dlights=Headlamp,add3Djscript=asylabels.js}{scale=1.25,trim=5mm 2mm 5mm 5mm,clip=true}{.6}{The plane $x=2$.}{fig:space2}{figures/figspace2}
% figure is printed below in next example.

\example{ex_space3}{Regions defined by planes}{
Sketch the region defined by the inequalities $-1\leq y\leq 2$.}
{The region is all points between the planes $y=-1$ and $y=2$. These planes are sketched in Figure \ref{fig:space3}, which are parallel to the $x$-$z$ plane. Thus the region extends infinitely in the $x$ and $z$ directions, and is bounded by planes in the $y$ direction.
\mfigurethree{width=125pt,3Dmenu,activate=onclick,deactivate=onclick,
3Droll=0,
3Dortho=0.0044,
3Dc2c=4 2.5 2,
3Dcoo=0 0 0,
3Droo=150,
3Dlights=Headlamp,add3Djscript=asylabels.js}{scale=1.25,trim=5mm 2mm 5mm 2mm,clip=true}{.35}{Sketching the boundaries of a region in Example \ref{ex_space3}.}{fig:space3}{figures/figspace3}
}\\



This section has introduced points in space and shown how equations can define regions in space. The next sections explore \emph{vectors}, an important mathematical object that we'll use to explore curves in space.

\printexercises{exercises/10_01_exercises}