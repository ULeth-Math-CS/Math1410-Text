\section{Complex Numbers}
\label{CmpNums}

We now move on to the study of the set of \sword{complex numbers}\index{numbers ! complex}\index{complex numbers}.  As you may recall, the complex numbers fill an algebraic gap left by the real numbers.  There is no real number $x$ with $x^2 = -1$, since for any real number $x^2 \geq 0$.  However, we could formally extract square roots and write $x = \pm \sqrt{-1}$.  We build the complex numbers by relabelling the quantity $\sqrt{-1}$ as $i$, the unfortunately misnamed \index{imaginary unit, $i$}\index{complex number ! imaginary unit, $i$}\sword{imaginary unit}.  The number $i$, while not a real number, is defined so that it plays along well with real numbers and acts very much like any other radical expression.  For instance, $3(2i) = 6i$, $7i-3i = 4i$, $(2-7i) + (3 + 4i) = 5-3i$, and so forth.  The key properties which distinguish $i$ from the real numbers are listed below.

\mnote{.7}{Historically, the lack of solutions to the equation $x^2=-1$ had nothing to do with the development of the complex numbers. Until the 19th century, equations such as $x^2=-1$ would have been considered in the context of the analytic geometry of Descartes. The lack of solutions simply indicated that the graph $y=x^2$ did not intersect the line $y=-1$. The more remarkable case was that of \textit{cubic} equations, of the form $x^3=ax+b$. In this case a \textbf{real} solution is \textit{guaranteed}, but there are cases where one needs \textbf{complex} numbers to find it! For details, see the excellent book \textit{Visual Complex Analysis}, by Tristan Needham.}

\medskip

\definition{idefn}{The imaginary unit}{
The imaginary unit $i$ satisfies the two following properties:

\begin{enumerate}

\item  $i^2 = -1$

\item  If $c$ is a real number with $c \geq 0$ then $\sqrt{-c} = i \sqrt{c}$

\end{enumerate}
}

\medskip

\mnote{.55}{Note the use of the indefinite article `a'.  Whatever beast is chosen to be $i$, $-i$ is the other square root of $-1$.}

\mnote{.45}{Some Technical Mathematics textbooks label the imaginary unit `$j$', usually to avoid confusion with the use of the letter $i$ to denote electric current.  While it carries the adjective `imaginary', these numbers have essential real-world implications.  For example, every electronic device owes its existence to the study of `imaginary' numbers.}

Property 1 in Definition \ref{idefn} establishes that $i$ does act as a square root of $-1$, and property 2 establishes what we mean by the `principal square root' of a negative real number.  In property 2, it is important to remember the restriction on $c$.  For example, it is perfectly acceptable to say  $\sqrt{-4} = i \sqrt{4} = i(2) = 2i$. However, $\sqrt{-(-4)} \neq i \sqrt{-4}$, otherwise, we'd get\[ 2 = \sqrt{4} = \sqrt{-(-4)} = i \sqrt{-4} = i (2i) = 2i^2 = 2(-1) = -2,\] which is unacceptable. The moral of this story is that the general properties of radicals do not apply for even roots of negative quantities.  With Definition \ref{idefn} in place, we are now in position to define the \sword{complex numbers}.

\medskip

\definition{complexdefn}{Complex number}{
\index{complex number ! definition of}
A \sword{complex number} is a number of the form $a+bi$, where $a$ and $b$ are real numbers and $i$ is the imaginary unit.  The set of complex numbers is denoted $\C$.
}

\medskip

Complex numbers include things you'd normally expect, like $3+2i$ and $\frac{2}{5} - i\sqrt{3}$.  However, don't forget that $a$ or $b$ could be zero, which means numbers like $3i$ and $6$ are also complex numbers.  In other words, don't forget that the complex numbers \textit{include} the real numbers, so $0$ and $\pi - \sqrt{21}$ are both considered complex numbers. We want to study the arithmetic of complex numbers, but before we can do so, we first need to make sure we understand what it means for two complex numbers to be equal.

\smallskip

\definition{def:compequality}{Equality of complex numbers}{
\index{complex numbers ! equality} Let $z=a+ib$ and $w=c+id$ be two complex numbers. We say that $z$ and $w$ are \sword{equal}, and write $z=w$, if and only if $a=c$ and $b=d$.}

\smallskip

The arithmetic of complex numbers is as you would expect. The definitions of addition and multiplication are as follows:

\smallskip

\definition{def:compaddition}{Addition of complex numbers}{
\index{complex numbers ! addition} Given two complex numbers $z=a+ib$ and $w=c+id$, we define their \sword{sum} to be the complex number given by
\[
z+w = (a+c)+i(b+d).
\]}

\smallskip

\definition{def:compmultiplication}{Multiplication of complex numbers}{
\index{complex numbers ! multiplication} Give two complex numbers $z=a+ib$ and $w=c+id$, we define their \sword{product} to be the complex number
\[
zw = (ac-bd) + i(ad+bc).
\]}

\smallskip

Addition of complex numbers is defined by simply adding the corresponding parts. The definition of multiplication looks complicated, but it's simply an application of the ``F.O.I.L.'' rule for multiplying binomials, where we have to account for the fact that $i^2=-1$.
 As long as we remember the two properties in Definition \ref{idefn}, we can treat expressions involving $i=\sqrt{-1}$ as we would with any other radical.
Let's work through an example to see how this works.

\medskip

%\label{complexnumberarithmetic}

\example{complexzeroex1}{Arithmetic with complex numbers}{
Perform the indicated operations.


\begin{multicols}{2}
\begin{enumerate}

\item  $(1-2i) - (3+4i)$
\item  $(1-2i)(3+4i)$ 
\item  $\dfrac{1-2i}{3-4i}$
\item  $\sqrt{-3} \sqrt{-12}$
\item  $\sqrt{(-3)(-12)}$
\item  $(x-[1+2i])(x-[1-2i])$

\end{enumerate}
\end{multicols}
}
{
\begin{enumerate}

\item  Subtraction is simply a variation on addition: We distribute the minus sign across the second complex number and combine like terms:
\begin{align*}
 (1-2i) - (3+4i) & =   1-2i-3-4i  \tag*{Distribute} \\
                 & =   -2 - 6i  \tag*{Gather like terms} \\
\end{align*}
Technically, we'd have to rewrite our answer  $-2-6i$ as $(-2) + (-6)i$ to be (in the strictest sense) `in the form $a+bi$'. That being said, even pedants have their limits, and we'll consider $-2-6i$ good enough.

\item  The key to multiplying complex numbers is to forget about Definition \ref{def:compmultiplication} above, and simply treat this as a product of two binomials. Using the Distributive Property (a.k.a. F.O.I.L.), we get
\begin{align*}
  (1-2i)(3+4i)  & =  (1)(3) + (1)(4i) - (2i)(3) - (2i)(4i)  \tag*{F.O.I.L.} \\
	              & =  3+4i-6i-8i^2  \\
								& =  3 - 2i - 8(-1)  \tag*{$i^2=-1$} \\
								& =  3 - 2i + 8  \\
								& =  11 - 2i 
\end{align*}

\item  How in the world are we supposed to simplify $\frac{1-2i}{3-4i}$?  Well, we deal with the denominator $3-4i$ as we would any other denominator containing two terms, one of which is a square root: we then multiply both numerator and denominator by $3+4i$, the (complex) conjugate of $3 - 4i$.  Doing so produces
\begin{align*}
 \dfrac{1-2i}{3-4i} & =  \dfrac{(1-2i)(3+4i)}{(3-4i)(3+4i)}  \tag*{Equivalent Fractions} \\[5pt]
                    & =    \dfrac{3 + 4i - 6i - 8i^2}{9 - 16i^2}  \tag*{F.O.I.L.}\\[5pt]
					& =  \dfrac{3 - 2i - 8(-1)}{9  - 16(-1)}  \tag*{$i^2 = -1$}\\[5pt]
					& =  \dfrac{11 - 2i}{25} =  \dfrac{11}{25} - \dfrac{2}{25} \, i 
\end{align*}
										

\item  We use property 2 of Definition \ref{idefn} first, then apply the rules of radicals applicable to real numbers to get $\sqrt{-3} \sqrt{-12} = \left(i \sqrt{3}\right) \left(i \sqrt{12}\right) = i^2 \sqrt{3\cdot 12} = -\sqrt{36} = -6$.

\item  We adhere to the order of operations here and perform the multiplication before the radical to get  $\sqrt{(-3)(-12)} = \sqrt{36} = 6$. 

\item  We can brute force multiply using the distributive property and see that 
\begin{align*}
(x-[1+2i])(x-[1-2i]) & =  x^2 -x[1-2i]-x[1+2i]+[1-2i][1+2i]  \\
					 & =  x^2-x+2ix-x-2ix+1-2i+2i-4i^2   \\
					 & =  x^2-2x + 1-4(-1) \\
					 & =  x^2 -2x +5 
\end{align*}
\end{enumerate}
\vskip-2\baselineskip}

\medskip



In the previous example, we used the idea of a `conjugate' to divide two complex numbers.   More generally, the \index{complex number ! complex conjugate ! definition of}\index{conjugate ! complex conjugate ! definition of}\sword{complex conjugate} of a complex number $a+bi$ is the number $a-bi$.  The notation commonly used for complex conjugation is a `bar':  $\overline{a+bi} = a-bi$. For example, 
\[
\overline{3+2i} = 3-2i \quad \text{ and } \quad \overline{3-2i} = 3+2i.
\]
To find $\overline{6}$, we note that $\overline{6} = \overline{6+0i}= 6 - 0i = 6$, so $\overline{6} = 6$. Similarly, $\overline{4i} = -4i$, since $\overline{4i} = \overline{0 + 4i} = 0 - 4i =  -4i$.  The properties of the conjugate are summarized in the following theorem.

%\pagebreak

\theorem{conjugateprops}{Properties of the Complex Conjugate}{
\index{complex number ! conjugate ! properties of}\index{conjugate of a complex number ! properties of}
Let $z$ and $w$ be complex numbers. 

\begin{itemize}

\item  $\overline{\overline{z}} = z$

\item  $ \overline{z+w} = \overline{z} + \overline{w}$

\item  $ \overline{zw} = \overline{z} \, \overline{w}$

\item  $\overline{z^{n}} = \left(\overline{z}\right)^n$, for any natural number $n$

\item  $z$ is a real number if and only if $\overline{z} = z$.

\end{itemize}
}

\medskip

\mnote{.7}{You may recall using conjugates to rationalize expressions involving square roots. For example, we have
\begin{align*}
\frac{3}{\sqrt{2}+1} &= \frac{3(\sqrt{2}-1)}{(\sqrt{2}+1)(\sqrt{2}-1)}\\ &= 3\sqrt{3}-3.
\end{align*}

The key observation is that multiplying by the conjugate sets up a difference of squares: the terms involving the radical cancel out. In some ways, multiplication by a complex conjugate is even more convenient than with real radicals: since
\begin{align*}
(a+ib)(a-ib) &= a^2-iab+iab-i^2b^2\\
&=a^2+b^2,
\end{align*}
the product $z\overline{z}$ is never negative, and only vanishes if $z=0$.}

Essentially, Theorem \ref{conjugateprops} says that complex conjugation works well with addition, multiplication and powers.  The proofs of these properties can best be achieved by writing out $z = a+bi$ and $w = c+di$ for real numbers $a$, $b$, $c$ and $d$.   Next, we compute the left and right sides of each equation and verify that they are the same.  

\smallskip

Verifying the first property is a very quick exercise.  To prove the second property, we compare $\overline{z+w}$ with $\overline{z} + \overline{w}$.  We have $\overline{z} + \overline{w} = \overline{a+bi} + \overline{c+di}  = a-bi + c-di$.  To find $\overline{z+w}$, we first compute 
\begin{align*}
z+w &= (a+bi) + (c+di)\\
& = (a+c)+(b+d)i
\end{align*}
 so 
\begin{align*}
 \overline{z+w} &= \overline{(a+c)+(b+d)i}\\
  & = (a+c) - (b+d)i\\
  & = a - bi + c - di\\
  & = \overline{z} + \overline{w}
\end{align*}
 
 As such, we have established  $\overline{z+w} = \overline{z}+\overline{w}$. The proof for multiplication works similarly: we have
 \begin{align*}
 \overline{zw}& = \overline{(ac-bd)+i(ad+bc)}\\
  & = (ac-bd)-i(ad+bc)\\
  & = (ac-(-b)(-d))+i(a(-d)+b(-c))\\
  & = (a-ib)(c-id)\\
  & = \overline{z}\,\overline{w}.
 \end{align*} 
   The proof that the conjugate works well with powers can be viewed as a repeated application of the product rule.   The last property is a characterization of real numbers.  If $z$ is real, then $z = a + 0i$, so $\overline{z} = a - 0i = a = z$.  On the other hand, if $z=\overline{z}$, then $a+bi = a - bi$ which means $b=-b$ so $b=0$.  Hence, $z = a +0i = a$ and is real.

%\mnote{.25}{Proof by Mathematical Induction usually isn't encountered until Math 2000. It provides a way of formally proving statements that are claimed to hold true for all natural numbers.}

\medskip

It is worth noting that although the arithmetic of complex numbers seems, at first impression, to be very different and strange compared to the arithmetic of real numbers, it actually satisfies all the same properties, as outlined in the following theorem.

%\mnote{.15}{The formal definition of a \sword{field} is usually not encountered until a second course in abstract algebra, such as Math 4500. For now, you should think of a field as any number system where the rules of arithmetic behave exactly as you expect them to.}

\smallskip

\theorem{thm:comparithprops}{Properties of Complex Arithmetic}{
The addition and multiplication of complex numbers satisfy the following properties:
\begin{itemize}
\item \textbf{Closure under addition:} For any complex numbers $z$ and $w$, $z+w$ is a complex number.
\item \textbf{Commutativity of addition:} For any complex numbers $z$ and $w$, $z+w = w+z$.
\item \textbf{Associativity of addition:} For any complex numbers $z_1, z_2, z_3$, $z_1+(z_2+z_3) = (z_1+z_2)+z_3$.
\item \textbf{Additive identity:} There exists a complex number $0$ such that $z+0 = 0+z=z$ for every complex number $z$.
\item \textbf{Additive inverses:} For every complex number $z$ there exists a complex number $-z$ such that $z+(-z)=-z+z=0$.
\item \textbf{Closure under multiplication:} For any complex numbers $z$ and $w$, $zw$ is a complex number.
\item \textbf{Commutativity of multiplication:} For any complex numbers $z$ and $w$, $zw=wz$.
\item \textbf{Assiciativity of multiplication:} For any complex numbers $z_1, z_2, z_3$, $z_1(z_2z_3) = (z_1z_2)z_3$
\item \textbf{Multiplicative identity:} There exists a complex number $1$ such that $1\cdot z = z\cdot 1 = z$ for every complex number $z$.
\item \textbf{Multiplicative inverses:} For every complex number $z\neq 0$, there exists a complex number $z^{-1}$ such that $zz^{-1}=z^{-1}z=1$.
\item \textbf{Distributive property:} For all complex numbers $z_1, z_2, z_3$, we have $z_1(z_2+z_3) = z_1z_2+z_1z_3$.
\end{itemize}
}

\smallskip

We leave the proof of Theorem \ref{thm:comparithprops} as a long (but straightforward) exercises. Working through the proof is a good way to confirm for yourself that you understand the corresponding rules for real number arithmetic from Section \ref{RealNumberArithmetic}, and how the properties for complex arithmetic are inherited from their real counterparts.


We now consider the problem of solving quadratic equations. Consider  $x^2-2x+5 = 0$. The discriminant $b^2 - 4ac = -16$ is negative, so we know from the quadratic formula that there are no \textit{real} solutions, since the Quadratic Formula would involve the term $\sqrt{-16}$.  Complex numbers, however, are built just for such situations, so we can go ahead and apply the Quadratic Formula to get:
 \[
  x = \dfrac{-(-2) \pm \sqrt{(-2)^2-4(1)(5)}}{2(1)} = \dfrac{2 \pm \sqrt{-16}}{2} = \dfrac{2 \pm 4i}{2} = 1 \pm 2i.
\]  



\pagebreak

\example{complexsolnsreviewex}{Finding complex solutions}{
Find the complex solutions to the following equations.

\begin{multicols}{3}
\begin{enumerate}

\mnote{.5}{We're assuming some prior familiarity on the part of the reader where quadratic equations are concerned. If you're a bit rusty when it comes to finding \textit{real} solutions to quadratic equations (and in particular, the quadratic formula), you may want to check out the review materials available on the ``Math Basics'' Moodle page.}

\item  $\dfrac{2x}{x+1} = x+3$

\item $2t^4 = 9t^2 + 5$

\item  $z^3 + 1 = 0$

\end{enumerate}
\end{multicols}
}
{
\begin{enumerate}

%\enlargethispage{20pt}

\item  Clearing fractions yields a quadratic equation so we collect all terms on one side and apply the Quadratic Formula.
\begin{align*}
\dfrac{2x}{x+1} & =   x+3  \\
2x & =  (x+3)(x+1)  \tag*{Clear denominators} \\
2x & =  x^2 + x + 3x + 3  \tag*{F.O.I.L.} \\
2x & =  x^2 + 4x + 3  \tag*{Gather like terms} \\
 0 & =  x^2 + 2x + 3  \tag*{Subtract $2x$}
\end{align*}
From here, we apply the Quadratic Formula 
\begin{align*}
x  & =   \dfrac{-2 \pm \sqrt{2^2 - 4(1)(3)}}{2(1)}   \tag*{Quadratic Formula}\\
    & =   \dfrac{-2 \pm \sqrt{-8}}{2}  \tag*{Simplify}\\
	& =   \dfrac{-2 \pm i \sqrt{8}}{2}  \tag*{Definition of $i$}\\
	& =   \dfrac{-2 \pm i 2\sqrt{2}}{2}  \tag*{Product Rule for Radicals}\\
	& =  \dfrac{\cancel{2}(-1 \pm i\sqrt{2})}{\cancel{2}} \tag*{Factor and reduce}\\
	& =  -1 \pm i \sqrt{2} 
\end{align*}	
We get two answers: $x = -1 + i\sqrt{2}$ and its conjugate $x = -1 - i\sqrt{2}$.  Checking both of these answers reviews all of the salient points about complex number arithmetic and is therefore strongly encouraged.

\item  Since we have three terms, and the exponent on one term (`$4$' on $t^4$) is exactly twice the exponent on the other (`$2$' on $t^2$), we have a Quadratic in Disguise.  We proceed accordingly.
\begin{align*}
2t^4 & =  9t^2 + 5  \\
2t^4 - 9t^2 - 5 & =  0  \tag*{Subtract $9t^2$ and $5$} \\
(2t^2 + 1)(t^2 - 5) & =  0  \tag*{Factor} \\
2t^2 + 1 = 0 & \quad \text{ or } \quad  t^2 = 5  \tag*{Zero Product Property}
\end{align*}
From $2t^2 + 1 = 0$ we get $2t^2 = -1$, or $t^2 = -\frac{1}{2}$.  We extract square roots as follows: 
\[
 t = \pm \sqrt{-\dfrac{1}{2}} = \pm i \sqrt{\dfrac{1}{2}} = \pm i \dfrac{\sqrt{1}}{\sqrt{2}} = \pm i \dfrac{1}{\sqrt{2}} = \pm \dfrac{i \sqrt{2}}{2},
\]
where we have rationalized the denominator per convention.  From $t^2 = 5$, we get $t = \pm \sqrt{5}$. In total, we have four complex solutions - two real: $t = \pm \sqrt{5}$ and two non-real: $t = \pm \frac{i \sqrt{2}}{2}$.

\mnote{.3}{Remember, all real numbers are complex numbers, so `complex solutions' means both real and non-real answers.} 

\item To find  the \textit{real} solutions to  $z^3 + 1 = 0$, we can subtract the $1$ from both sides and extract cube roots: $z^3 = -1$, so $z  = \sqrt[3]{-1} = -1$.  It turns out there are two more non-real complex number solutions to this equation.  To get at these, we factor:
\begin{align*}
z ^ 3 + 1 & =  0  \\
(z + 1)(z^2 - z + 1) & =  0  \tag*{Factor (Sum of Two Cubes)} \\
z + 1 = 0 & \quad \text{ or } \quad  z^2 - z + 1 = 0
\end{align*}
From $z+1 = 0$, we get our real solution $z = -1$.  From $z^2 -z + 1 = 0$, we apply the Quadratic Formula to get: 
\[
z = \dfrac{-(-1) \pm \sqrt{(-1)^2 - 4(1)(1)}}{2(1)} = \dfrac{1 \pm \sqrt{-3}}{2} = \dfrac{1 \pm i\sqrt{3}}{2}
\]
Thus we get \textit{three} solutions to $z^3 + 1 = 0$ - one real: $z = -1$ and two non-real: $z =  \frac{1 \pm i\sqrt{3}}{2}$.  As always, the reader is encouraged to test their algebraic mettle and check these solutions. 
		
\end{enumerate}
}

\mnote{.55}{In Section \ref{PolarComplex} we will develop a much easier method for solving the equation $z^3+1=0$ using polar coordinates.}

\medskip

It is no coincidence that the non-real solutions to the equations in Example \ref{complexsolnsreviewex} appear in  complex conjugate pairs. Any time we use the Quadratic Formula to solve an equation with \underline{real} coefficients, the answers will form a  complex conjugate pair owing to the $\pm$ in the Quadratic Formula.  This is stated formally in the following theorem.

\smallskip

\theorem{discriminanttheoremcomplexversion}{Discriminant Theorem}{
Given a Quadratic Equation $AX^2 + BX + C = 0$, where $A$, $B$ and $C$ are real numbers, let $D = B^2 - 4AC$ be the discriminant.

\begin{itemize}

\item  If $D > 0$, there are two distinct real number solutions to the equation. 

\item  If $D = 0$, there is one (repeated) real number solution.  

\textbf{Note:}  `Repeated' here comes from the fact that `both' solutions $\frac{-B \pm 0}{2A}$ reduce to $-\frac{B}{2A}$.

\item  If $D < 0$, there are two non-real solutions which form a complex conjugate pair.

\end{itemize}
}

\pagebreak

Theorem \ref{discriminanttheoremcomplexversion} tells us that if ever we obtain non-real zeros to a quadratic function with \underline{real} coefficients, the zeros  will be a complex conjugate pair. (Do you see why?)  Next, we note that in Example \ref{complexzeroex1}, part 6, we found $(x-[1+2i])(x-[1-2i])=x^2-2x+5$.  This demonstrates that the factor theorem holds even for non-real zeros, i.e,  $x=1+2i$ is a zero of $f(x)=x^2-2x+5$, and, sure enough, $(x-[1+2i])$ is a factor of $f(x)$.  It turns out that polynomial division works the same way for all complex numbers, real and non-real alike, so the Factor and Remainder Theorems hold as well.  But how do we know if a general polynomial has any complex zeros at all?  We have many examples of polynomials with no real zeros.  Can there be polynomials with no zeros whatsoever?  The answer to that last question is ``No.'' and the theorem which provides that answer is \index{Fundamental Theorem of Algebra} The Fundamental Theorem of Algebra.

\medskip

\theorem{ftoa}{The Fundamental Theorem of Algebra}{\index{Fundamental Theorem of Algebra} \index{theorem ! Fundamental Theorem of Algebra}  Suppose $f$ is a polynomial function with complex number coefficients of degree $n \geq 1$, then $f$ has at least one complex zero.
}

\medskip

The Fundamental Theorem of Algebra is an example of an `existence' theorem in Mathematics.  It  guarantees the existence of at least one zero, but gives us no algorithm to use in finding it.  The authors are fully aware that the full impact and profound nature of the Fundamental Theorem of Algebra  is lost on most students, and that's fine.  It took mathematicians literally hundreds of years to prove the theorem in its full generality, and some of that history can be found by looking up the Fundamental Theorem on \href{http://en.wikipedia.org/wiki/Fundamental_theorem_of_algebra}{\underline{Wikipedia}}.  Note that the Fundamental Theorem of Algebra  applies to not only polynomial functions with real coefficients, but to those with complex number coefficients as well.  

\mnote{.65}{The Fundamental Theorem of Algebra has since been proved many times, using many different methods, by many mathematicians. There are probably very few, if any, results in mathematics with the variety of proofs this result has. Unfortunately, none of the proofs can be understood within the realm of this text, but if the reader is sufficiently interested, a collection of proofs can be found at \href{http://www.cut-the-knot.org/fta/analytic.shtml}{\underline{www.cut-the-knot.org/fta/analytic.shtml}}.}

\enlargethispage{2\baselineskip}

\smallskip

Suppose  $f$ is a polynomial of degree $n \geq 1$.  The Fundamental Theorem of Algebra guarantees us at least one complex zero, $z_{1}$, and as such, the Factor Theorem guarantees that $f(x)$ factors as $f(x) = \left(x - z_{1}\right) q_{1}(x)$ for a polynomial function $q_{1}$,  of degree exactly $n-1$.  If $n-1 \geq 1$, then the Fundamental Theorem of Algebra guarantees a complex zero of $q_{1}$ as well, say $z_{2}$, so then the Factor Theorem gives us $q_{1}(x) = \left(x - z_{2}\right) q_{2}(x)$, and hence $f(x) = \left(x - z_{1}\right) \left(x - z_{2}\right) q_{2}(x)$.  We can continue this process exactly $n$ times, at which point our quotient polynomial $q_{n}$ has degree $0$ so it's a constant.  This argument gives us the following factorization theorem.

\smallskip

\theorem{complexfactorization}{Complex Factorization Theorem}{ Suppose $f$ is a polynomial function with complex number coefficients.  If the degree of $f$ is $n$ and $n \geq 1$, then  $f$ has exactly $n$ complex zeros, counting multiplicity.  If $z_{1}$, $z_{2}$, \ldots, $z_{k}$ are the distinct zeros of $f$, with multiplicities $m_{1}$, $m_{2}$, \ldots, $m_{k}$, respectively, then $f(x) = a\left(x - z_{1}  \right)^{m_{1}}\left(x - z_{2}  \right)^{m_{2}} \cdots \left(x - z_{k}  \right)^{m_{k}}$. \index{Complex Factorization Theorem}\index{factorization ! over the complex numbers}
}

\smallskip

To complete our study of the arithmetic of complex numbers, we should discuss powers and roots. Computing powers can be done using the form $z=x+iy$, but it quickly becomes unpleasant (try computing $(4+3i)^7$, for example). Roots, on the other hand, are nearly impossible. Luckily for us, there is a better way: using the \textit{polar form} of complex numbers. Before we get to this discussion, however, we need to pause to introduce the \textit{polar coordinate system} for the Cartesian coordinate system.

\printexercises{exercises/00_04_exercises}