%%%%
% Additional packages to support the book, not part of the APEX style.
%%%%
\usepackage{fancyhdr}
\usepackage{eso-pic}
\usepackage{lipsum}
\usepackage{units}
%\usepackage{supertabular,longtable,hhline}

%\usepackage{pgfplots}
%\pgfplotsset{width=\marginparwidth+1pt,compat=1.3}
\usepackage[font=small]{caption}
%,justification=centering

%%%%
%% These are low level LaTeX commands that determine the 
%% look of the Chapter and Section headings.
%% Note the use in the chapter part of an external file
%% that contains graphics for each chapter start.
%%%%

%%%%
%% Commands for the header, utilizing the fancyhdr
%% (fancy header) package
%%%%

\pagestyle{fancy}
\fancyhead{}
%\fancyfoot{}
\renewcommand{\chaptermark}[1]{\markboth{\chaptername\ \thechapter\ \ \ \ {#1}}{}}
\renewcommand{\sectionmark}[1]{\markright{\thesection\ \ \ \  #1}}
\fancyhf{}         %Clears all header and footer fields, in preparation.
\renewcommand{\headrulewidth}{0pt}
\renewcommand{\footrulewidth}{0pt}

\ifthenelse{\boolean{longpage}}%
{}% end header of longpage
{% begin header/footer of not longpage
\fancyhfoffset[LE,RO]{\marginparwidth+\marginparsep}
%\fancyfoot[LE,RO]{\textbf{\thepage}} %Displays the page number in bold in the header,

\fancyfoot{}
\fancyfoot[LE]{\textbf{\thepage}}
\fancyfoot[RO]{\textbf{\thepage}}

%\fancyfoot[LE]{\begin{minipage}{\textwidth}%
%\noindent\hskip\marginparwidth\hskip\marginparsep\hskip-4pt\rule{\textwidth}{.4pt}
%\vskip.2\baselineskip
%\noindent\hskip\marginparwidth\hskip\marginparsep\hskip-4pt%
%Notes:
%\vskip 1.5in\textbf{\thepage}
%\end{minipage}} 

%\fancyfoot[RO]{\begin{minipage}{\textwidth+\marginparwidth+\marginparsep}%
%\rule{\textwidth-\marginparwidth-\marginparsep}{.4pt}
%\vskip.2\baselineskip
%Notes:
%\vskip 1.5in
%\hfill\textbf{\thepage}
%\end{minipage}}       


\fancyhead[LE]{\nouppercase{\leftmark}}
      %Displays the upper-level (chapter) information---
      % as determined above---in non-upper case in the header, 
      %to the right on even pages.
\fancyhead[RO]{\rightmark}
			%Displays the lower-level (section) information---as
      % determined above---in the header, to the left on odd pages.
      
}% End the ifthenelse{longpage}


\ifthenelse{\boolean{longpage}}% for changing the header
{% longpage
\newcommand{\exerciseheader}{}
\newcommand{\regularheader}{}
\newcommand{\endmatheader}{}
}% i.e, the above does nothing
{% now for a real change
\newcommand{\exerciseheader}{%
				\fancyhfoffset[LE,RO]{32pt}%
				\fancyfoot[LE]{\textbf{\thepage}}% 
				\fancyfoot[RO]{\textbf{\thepage}}
				\fancyhead{}% 
}

\newcommand{\regularheader}{%
\fancyhead[LE]{\nouppercase{\leftmark}}%
\fancyhead[RO]{\rightmark}%
%\fancyfoot[LE]{}
%\fancyfoot[RO]{}
\fancyfoot{}
\fancyfoot[LE]{\textbf{\thepage}}
\fancyfoot[RO]{\textbf{\thepage}}
%\fancyfoot[LE]{\begin{minipage}{\textwidth}%
%\noindent\hskip\marginparwidth\hskip\marginparsep\hskip-4pt\rule{\textwidth}{.4pt}
%\vskip.2\baselineskip
%\noindent\hskip\marginparwidth\hskip\marginparsep\hskip-4pt%
%Notes:
%\vskip 1.5in\textbf{\thepage}
%\end{minipage}} 

%\fancyfoot[RO]{\begin{minipage}{\textwidth+\marginparwidth+\marginparsep}%
%\rule{\textwidth-\marginparwidth-\marginparsep}{.4pt}
%\vskip.2\baselineskip
%Notes:
%\vskip 1.5in
%\hfill\textbf{\thepage}
%\end{minipage}}

\fancyhfoffset[LE,RO]{\marginparsep+\marginparwidth}
}

\newcommand{\qendheader}{%
		\fancyhead{}%
		\fancyfoot{}%
		}
}% ends if/then/else exercise/regular headers


%
%  Defining what the chapter titles look like
%

\newdimen\titleheight

\makeatletter
\def\@makechapterhead#1{%
  {\parindent \z@ \raggedright \reset@font
    {\Huge \thechapter: \scshape \textsc #1}
    \par\vskip 10\p@
    \hrule height 1pt
    \vskip 10\p@
  }}
%%  
%%%\makeatletter
%%\def\@makesectionhead#1{%
%%	 {\reset@font\LARGE\itshape\bfseries\strut #1 \thechapter.\thesection \ #1
%%	 }}

%%%%
%%
%%  For figures in the margin
%%
%%%%

%%\usepackage{wrapfig}
%%
%%\setlength{\marginparsep}{10pt}
%%\setlength{\wrapoverhang}{\marginparwidth}
%%\addtolength{\wrapoverhang}{\marginparsep}
%%
%%%\newenvironment{mfigure}[1]{%
%%%    \wrapfigure{#1}{0pt}%
%%%    \begin{minipage}{\marginparwidth}%
%%%    \centering%
%%%}{%
%%%    \end{minipage}%
%%%    \endwrapfigure%
%%%}


\ifthenelse{\boolean{longpage}}%
		{\newcommand{\mfigure}[5][]{%
		\vskip \baselineskip%
			\noindent\begin{minipage}{\textwidth}
  			\centering
  			\myincludegraphics{#5}
  			#1%
  			\captionsetup{type=figure}
	  		\caption{#3}\label{#4}
  		\end{minipage}
		}}
		{\newcommand{\mfigure}[5][]{%
		\ifthenelse{\isodd{\thepage}}%
			{\AddToShipoutPicture*{\put(427,\LenToUnit{#2\paperheight}){%
			\begin{minipage}{\marginparwidth}%
				\centering%
	  		\myincludegraphics[#1]{#5}%
  			\captionsetup{type=figure}%
  			%#1%
	  		\caption{#3}\label{#4}%
  		\end{minipage}}}}%
			{\AddToShipoutPicture*{\put(36,\LenToUnit{#2\paperheight}){%
			\begin{minipage}{\marginparwidth}%
			  \centering%
  			\myincludegraphics[#1]{#5}%
  			\captionsetup{type=figure}%
  			%#1%
	  		\caption{#3}\label{#4}%
  		\end{minipage}}}}%
		}}%

\ifthenelse{\boolean{longpage}}%
		{\newcommand{\mtable}[4]{
					\vskip \baselineskip%
					\noindent\begin{minipage}{\textwidth}
  					\centering\small
  					#4
  					\captionsetup{type=figure}
	  				\caption{#2}\label{#3}
  				\end{minipage}
  				\vskip\baselineskip}
  	}
		{\newcommand{\mtable}[4]{%
			\ifthenelse{\isodd{\thepage}}%
				{\AddToShipoutPicture*{\put(427,\LenToUnit{#1\paperheight}){%
					\begin{minipage}{\marginparwidth}%
  					\centering\small%
  					#4%
  					\captionsetup{type=figure}%
	  				\caption{#2}\label{#3}%
  				\end{minipage}}}}%
				{\AddToShipoutPicture*{\put(36,\LenToUnit{#1\paperheight}){%
					\begin{minipage}{\marginparwidth}%
 						 \centering\small%
 						 #4%
						  \captionsetup{type=figure}%
	  					\caption{#2}\label{#3}%
 					 \end{minipage}}}}%
		}}%

\ifthenelse{\boolean{longpage}}%
		{\newcommand{\mnote}[2]{%
					\hskip 25pt%
					\begin{minipage}{\marginparwidth}%
					\small%
					#2%
					\end{minipage}%
					\vskip\baselineskip%
		}%
		}%
		{\newcommand{\mnote}[2]{%
			\ifthenelse{\isodd{\thepage}}%
				{\AddToShipoutPicture*{\put(427,\LenToUnit{#1\paperheight}){%
					\begin{minipage}{\marginparwidth}
  					\small
  					#2
  				\end{minipage}}}}%
				{\AddToShipoutPicture*{\put(36,\LenToUnit{#1\paperheight}){%
					\begin{minipage}{\marginparwidth}
 						 \small
 						 #2
					\end{minipage}}}}
		}}

%\ifthenelse{\boolean{longpage}}%
		%{}
		%{\newcommand{\mfigurethree}[5][]{%
		%\ifthenelse{\isodd{\thepage}}%
			%{\AddToShipoutPicture*{\put(427,\LenToUnit{#2\paperheight}){%
			%\begin{minipage}{\marginparwidth}%
				%\centering%
	  		%\includemedia[#1]{\includegraphics{#5}}{#5.prc}%
  			%\captionsetup{type=figure}%
  			%%#1%
	  		%\caption{#3}\label{#4}%
  		%\end{minipage}}}}%
			%{\AddToShipoutPicture*{\put(36,\LenToUnit{#2\paperheight}){%
			%\begin{minipage}{\marginparwidth}%
			  %\centering%
  			%\includemedia[#1]{\includegraphics{#5}}{#5.prc}%
  			%\captionsetup{type=figure}%
  			%%#1%
	  		%\caption{#3}\label{#4}%
  		%\end{minipage}}}}%
		%}}%
		
		\newcommand{\mfigurethree}[6]{%
		\ifthenelse{\isodd{\thepage}}%
			{\AddToShipoutPicture*{\put(427,\LenToUnit{#3\paperheight}){%
			\begin{minipage}{\marginparwidth}%
				\centering%
				\ifthenelse{\boolean{in_threeD}}{% in 3D
	  		\includemedia[#1]{\includegraphics{#6_3D\colornamesuffix}}{#6_3D.prc}}%now not in 3D
				{\myincludegraphics[#2]{#6}}%
  			\captionsetup{type=figure}%
  			%#1%
	  		\caption{#4}\label{#5}%
  		\end{minipage}}}}%
			{\AddToShipoutPicture*{\put(36,\LenToUnit{#3\paperheight}){%
			\begin{minipage}{\marginparwidth}%
			  \centering%
  			\ifthenelse{\boolean{in_threeD}}{% in 3D
	  		\includemedia[#1]{\includegraphics{#6_3D\colornamesuffix}}{#6_3D.prc}}%now not in 3D
				{\myincludegraphics[#2]{#6}}%
  			\captionsetup{type=figure}%
  			%#1%
	  		\caption{#4}\label{#5}%
  		\end{minipage}}}}%
		}

%\AddToShipoutPicture{\ifthenelse{\ifodd{\thepage}}%
%{\put(10,10){\begin{tikzpicture}\draw (0,0) circle (5pt);\end{tikzpicture}}}%
%{\put(100,100){\begin{tikzpicture}\draw (0,0) circle (5pt);\end{tikzpicture}}}}%
%

%\AddToShipoutPictureBG{%
% \AtPageLowerLeft{\hspace{1cm}A small logo: \rule{2cm}{3cm}}}
% 
%}{\AddToShipoutPicture*{\put(36,\LenToUnit{#2\paperheight})}}%
%{%
%\begin{minipage}{\marginparwidth}
%  \centering
%  \includegraphics{#1}
%  \captionof{figure}{#3}
%  \end{minipage}}}
%\AddToShipoutPicture*{\put(\LenToUnit{.5\paperwidth},\LenToUnit{.75\paperheight}){%
%  \begin{minipage}{100pt}
%  \centering
%  \includegraphics{test-figure2.pdf}
%  \captionof{figure}{Test caption}
%  \end{minipage}}}


%\newenvironment{mfigurefile}[2]{%
%\begin{tikzpicture}[remember picture,overlay]%
%\ifthenelse{\isodd{\thepage}}{\node [xshift=-36pt-.5\marginparwidth,yshift=#2\paperheight] at (current page.south east) }{\node [xshift=36pt+.5\marginparwidth,yshift=#2\paperheight] at (current page.south west) }%
%{\input{#1}};}%
%{\end{tikzpicture}%
%}

%\DeclareCaptionType{mytype}[Typename][List of mytype]


    
%%%%
%% End margin figure 
%%%%    
  
\newcommand{\noin}{\noindent}
\newcommand{\ds}{\displaystyle}
\newcommand{\vs}[1]{\vskip #1in}
\newcommand{\tbf}[1]{\textbf{#1}}
\newcommand{\dds}{\displaystyle}
\newcommand{\bmx}[1]{\left[\hskip -3pt\begin{array}{#1} }
\newcommand{\emx}{\end{array}\hskip -3pt\right]}
\newcommand{\bdt}[1]{\left| \begin{array}{#1} }
\newcommand{\edt}{\end{array} \right|}

\newcommand{\btz}{\begin{center}\begin{tikzpicture}}
\newcommand{\etz}{\end{tikzpicture}\end{center}}

\newcommand{\vx}[1][]{\ensuremath{\vec{x_{#1}}}}
\newcommand{\vxp}{\ensuremath{\vec{x_p}}}
\newcommand{\vu}{\ensuremath{\vec{u}}}
\renewcommand{\vv}{\ensuremath{\vec{v}}}
\newcommand{\vy}{\ensuremath{\vec{y}}}
\newcommand{\vz}{\ensuremath{\vec{z}}}
\newcommand{\vb}{\ensuremath{\vec{b}}}
\newcommand{\vw}{\ensuremath{\vec{w}}}
\newcommand{\veone}{\ensuremath{\vec{e_1}}}
\newcommand{\vetwo}{\ensuremath{\vec{e_2}}}
\newcommand{\vethree}{\ensuremath{\vec{e_3}}}
\newcommand{\vei}{\ensuremath{\vec{e_i}}}
\newcommand{\ven}[1]{\ensuremath{\vec{e_{#1}}}}
\newcommand{\rr}[1]{\ensuremath{\mathbb{R}^{#1}}}
\newcommand{\zero}{\ensuremath{\vec{\text{\it 0}}}}
\newcommand{\vect}[1]{\ensuremath{\vec{#1}}}
\newcommand{\arref}{\ensuremath{\overrightarrow{\text{rref}}}}
\newcommand{\vectt}[2]{\ensuremath{\bmx{c}#1\\#2\emx}}
\newcommand{\vecttt}[3]{\ensuremath{\bmx{c}#1\\#2\\#3\emx}}

\newcommand{\ttmm}{$M$}
\newcommand{\mm}{\texttt{M}}
\newcommand{\ii}[1]{\ensuremath{I_{#1}}}
\newcommand{\tta}{\ensuremath{A}}
\newcommand{\ttb}{\ensuremath{B}}
\newcommand{\ttc}{\ensuremath{C}}
\newcommand{\ttd}{\ensuremath{D}}
\newcommand{\ttm}{\ensuremath{M}}
\newcommand{\ttx}{\ensuremath{X}}
\newcommand{\tti}{\ensuremath{I}}
\newcommand{\tty}{\ensuremath{Y}}
\newcommand{\ttp}{\ensuremath{P}}
\newcommand{\ttat}{\ensuremath{A^T}}
\newcommand{\ttbt}{\ensuremath{B^T}}
\newcommand{\ttct}{\ensuremath{C^T}}
\newcommand{\ttdt}{\ensuremath{D^T}}
\newcommand{\ttmt}{\ensuremath{M^T}}
\newcommand{\ttxt}{\ensuremath{X^T}}
\newcommand{\ttit}{\ensuremath{I^T}}
\newcommand{\ttyt}{\ensuremath{Y^T}}
\newcommand{\ttai}{\ensuremath{A^{-1}}}
\newcommand{\ttbi}{\ensuremath{B^{-1}}}
\newcommand{\ttxi}{\ensuremath{X^{-1}}}
\newcommand{\ttpi}{\ensuremath{P^{-1}}}
\newcommand{\ttaxb}{\ensuremath{\tta\vx=\vb}}
\newcommand{\ttaxo}{\ensuremath{\tta\vx=\zero}}
\newcommand{\eyetwo}{\ensuremath{\bmx{cc}1&0\\0&1\emx}}
\newcommand{\eyethree}{\ensuremath{\bmx{ccc}1&0&0\\0&1&0\\0&0&1\emx}}
\newcommand{\eyefour}{\ensuremath{\bmx{cccc}1&0&0&0\\0&1&0&0\\0&0&1&0\\0&0&0&1\emx}}
\newcommand{\ma}{\ensuremath{A}}
\newcommand{\mb}{\ensuremath{B}}
\newcommand{\mc}{\ensuremath{C}}
\newcommand{\md}{\ensuremath{D}}
\newcommand{\tto}{\textbf{0}}
\renewcommand{\det}[1]{\text{det}\ensuremath{\left(#1\right)}}
%\newcommand{\tr}{$^\text{\tt T}$}
\newcommand{\lda}{\ensuremath{\lambda}}
\newcommand{\rref}{reduced row echelon form}
\newcommand{\Rref}{Reduced row echelon form}
\newcommand{\el}{eigenvalue}
\newcommand{\ev}{eigenvector}
\newcommand{\realn}{\ensuremath{\mathbb{R}^n}}
\newcommand{\realm}{\ensuremath{\mathbb{R}^m}}
\newcommand{\realnm}{\ensuremath{\mathbb{R}^n\rightarrow\mathbb{R}^m}}
\newcommand{\real}[1]{\ensuremath{\mathbb{R}^{#1}}}
\newcommand{\rrr}[2]{\ensuremath{\mathbb{R}^{#1}\rightarrow\mathbb{R}^{#2}}}
\newcommand{\TT}{\ensuremath{[\, T\, ]}}  
    
\newcommand{\len}[1]{\left\lVert #1\right\rVert}
\newcommand{\fp}{\ensuremath{f\,'}}
\newcommand{\fpp}{\ensuremath{f\,''}}

\newcommand{\Fp}{\ensuremath{F\primeskip'}}
\newcommand{\Fpp}{\ensuremath{F\primeskip''}}

\newcommand{\yp}{\ensuremath{y\primeskip'}}
\newcommand{\gp}{\ensuremath{g\primeskip'}}

\newcommand{\dx}{\ensuremath{\Delta x}}
\newcommand{\dy}{\ensuremath{\Delta y}}
%\newcommand{\dz}{\ensuremath{\Delta z}}
\newcommand{\ddz}{\ensuremath{\Delta z}}

\newcommand{\thet}{\ensuremath{\theta}}
\newcommand{\norm}[1]{\ensuremath{||\ #1\ ||}}
\newcommand{\vnorm}[1]{\ensuremath{\norm{\vec #1}}}
\newcommand{\snorm}[1]{\ensuremath{\left|\left|\ #1\ \right|\right|}}
\newcommand{\la}{\left\langle}
\newcommand{\ra}{\right\rangle}
\newcommand{\dotp}[2]{\ensuremath{\vec #1 \cdot \vec #2}}
\newcommand{\proj}[2]{\ensuremath{\text{proj}_{\,\vec #2}{\,\vec #1}}}
\newcommand{\crossp}[2]{\ensuremath{\vec #1 \times \vec #2}}
\newcommand{\veci}{\ensuremath{\vec i}}
\newcommand{\vecj}{\ensuremath{\vec j}}
\newcommand{\veck}{\ensuremath{\vec k}}
\newcommand{\vecu}{\ensuremath{\vec u}}
\newcommand{\vecv}{\ensuremath{\vec v}}
\newcommand{\vecw}{\ensuremath{\vec w}}
\newcommand{\vecx}{\ensuremath{\vec x}}
\newcommand{\vecy}{\ensuremath{\vec y}}
\newcommand{\vrp}{\ensuremath{\vec r\, '}}
\newcommand{\vsp}{\ensuremath{\vec s\, '}}
\newcommand{\vrt}{\ensuremath{\vec r(t)}}
\newcommand{\vst}{\ensuremath{\vec s(t)}}
\newcommand{\vvt}{\ensuremath{\vec v(t)}}
\newcommand{\vat}{\ensuremath{\vec a(t)}}
\newcommand{\px}{\ensuremath{\partial x}}
\newcommand{\py}{\ensuremath{\partial y}}
\newcommand{\pz}{\ensuremath{\partial z}}
\newcommand{\pf}{\ensuremath{\partial f}}
\newcommand{\underlinespace}{\underline{\phantom{xxxxxx}}}

\newcommand{\mathN}{\ensuremath{\mathbb{N}}}

\newcommand{\zerooverzero}{\ensuremath{\ds \raisebox{8pt}{\text{``\ }}\frac{0}{0}\raisebox{8pt}{\text{\ ''}}}}

\newenvironment{amatrix}[1]{%
  \left[\begin{array}{@{}*{#1}{c}|c@{}}
}{%
  \end{array}\right]
}
\newenvironment{aamatrix}[1]{%
  \left[\begin{array}{@{}*{#1}{c}|*{#1}{c}@{}}
}{%
  \end{array}\right]
}

\newcommand{\bam}{\begin{amatrix}}
\newcommand{\eam}{\end{amatrix}}
\newcommand{\baam}{\begin{aamatrix}}
\newcommand{\eaam}{\end{aamatrix}}
\newcommand{\bbm}{\begin{bmatrix}}
\newcommand{\ebm}{\end{bmatrix}}
\newcommand{\bvm}{\begin{vmatrix}}
\newcommand{\evm}{\end{vmatrix}}


\newcommand{\myrule}{\rule[-4pt]{0pt}{13pt}}
\newcommand{\mmrule}{\rule[-10pt]{0pt}{15pt}}
\newcommand{\myds}{\ds\mmrule}
\newcommand{\deriv}[2]{\ensuremath{\myds\frac{d}{dx}\left(#1\right)=#2}}
\newcommand{\myint}[2]{\ensuremath{\myds\int #1\ dx=} \ensuremath{\ds #2}}

\DeclareMathOperator{\sech}{sech}
\DeclareMathOperator{\csch}{csch}

\newcommand{\sword}[1]{\textbf{#1}}

\newcommand{\primeskip}{\hskip.75pt}

%%%% Begin Header TikZ

%  Some TiKZ  shortcuts to help make drawing 3D vectors faster.
%

\newcommand{\plotlinecolor}{blue}

%
% Draw x and y tick marks
%
\newcommand{\drawxticks}[1]
{\foreach \x in {#1}
		{\draw  (\x,-.1)--(\x,.1);
			};
}
\newcommand{\drawyticks}[1]
{\foreach \x in {#1}
		{\draw  (-.1,\x)--(.1,\x);
			};
}

\newcommand{\drawxlines}[3]
{\draw[<->] (#1,0) -- (#2,0) node [right] {$x$};
\foreach \x in {#3}
		{\draw  (\x,-.1)--(\x,.1);
			};
}

\newcommand{\drawylines}[3]
{\draw[<->] (0,#1) -- (0,#2) node [above] {$y$};
\foreach \x in {#3}
		{\draw  (-.1,\x)--(.1,\x);
			};
}

\newcommand{\drawxlabels}[1]
{\foreach \x in {#1}
		{\draw  (\x,-.1) node [below] {\scriptsize $\x$};
		};
}

\newcommand{\drawylabels}[1]
{\foreach \x in {#1}
		{\draw  (-.1,\x) node [left] {\scriptsize $\x$};
		};
}



\newcommand{\drawvect}[7]{\draw [#4] (0,0,0) -- (#1,0,0) -- (#1,#2,0) -- (#1,#2,#3);
  \draw [#5] (0,0,0) -- (#1,#2,#3) node [#6] {#7};}
\newcommand{\drawjustvect}[7]{\draw [#5] (0,0,0) -- (#1,#2,#3) node [#6] {#7};}

\newcommand{\drawxaxis}[4]{\draw [->] (#1,0,0) -- (#2,0,0) node [below left] {$x$};
\foreach \x in {#3,...,#4}
 {\draw (\x,-.1,0)--(\x,.1,0); } }
\newcommand{\drawyaxis}[4]{\draw [->] (0,#1,0) -- (0,#2,0) node [right] {$y$};
\foreach \y in {#3,...,#4}
 {\draw (0,\y,-.1)--(0,\y,.1); }; }
\newcommand{\drawzaxis}[4]{\draw [->] (0,0,#1) -- (0,0,#2) node [above] {$z$};
\foreach \z in {#3,...,#4}
 {\draw (0,-.1,\z)--(0,.1,\z); }; }
%
% 
% Draws the VMI Spider in TikZ
%
\newcommand{\vmispider}[1][]
{\begin{scope}[#1]
	\draw  (-2,2) -- (0,-1) -- (2,2);
	\draw  (-1,-1) -- (-1,2) -- (0,1) -- (1,2) -- (1,-1);
	\draw  (-1,2.5) -- (1,2.5);
	\draw  (-1,-1.5) -- (1,-1.5);
	\draw  (0,-1.5) -- (0,2.5);
\end{scope}
}
%
% Draws the unit square, easy to transform
%
\newcommand{\unitsquare}[1][]
{\begin{scope}
	\draw [#1] (0,0) node (A) {} -- (1,0) node (B) {} -- (1,1) node (C) {} -- (0,1) node (D) {} -- cycle;
	\draw [->,>=latex,#1,ultra thin] (0,.25)--(1,.25);
	\draw [->,>=latex,#1,ultra thin] (0,.5)--(1,.5);
	\draw [->,>=latex,#1,ultra thin] (0,.75)--(1,.75);
	\filldraw [black]  (A) circle (2pt);%black
	\filldraw [fill=white,thick]  (B) ++(-2pt,-2pt) rectangle ++(4pt,4pt);%red
	\filldraw [fill=white,thick]  (C) circle (2pt);%blue
	\filldraw [fill=white,thick]  (D) ++(-2.5pt,-2.5pt) -- ++(5pt,0pt) -- ++(-2.5pt,5pt) -- cycle;%green
	\end{scope}
}
%
% Draws unit square for cover image
%
\newcommand{\unitsquarecover}[1][]
{\begin{scope}
	\draw [#1] (0,0) node (A) {} -- (1,0) node (B) {} -- (1,1) node (C) {} -- (0,1) node (D) {} -- cycle;
	\draw [->,>=latex,#1,thin] (0,.25)--(1,.25);
	\draw [->,>=latex,#1, thin] (0,.5)--(1,.5);
	\draw [->,>=latex,#1, thin] (0,.75)--(1,.75);
	\filldraw [black]  (A) circle (2pt);%black
	\draw [#1,ultra thick]  (B) ++(-2pt,-2pt) rectangle ++(4pt,4pt);%red
	\draw [ultra thick]  (C) circle (2pt);%blue
	\draw [#1,ultra thick]  (D) ++(-2.5pt,-2.5pt) -- ++(5pt,0pt) -- ++(-2.5pt,5pt) -- cycle;%green
	\end{scope}
}
%
% Draws unit square without the arrows in the middle.
%
\newcommand{\unitsquarewithoutarrows}[1][]
{\begin{scope}
	\draw [#1] (0,0) node (A) {} -- (1,0) node (B) {} -- (1,1) node (C) {} -- (0,1) node (D) {} -- cycle;
	\filldraw [black]  (A) circle (2pt);%black
	\filldraw [fill=white,thick]  (B) ++(-2pt,-2pt) rectangle ++(4pt,4pt);%red
	\filldraw [fill=white,thick]  (C) circle (2pt);%blue
	\filldraw [fill=white,thick]  (D) ++(-2.5pt,-2.5pt) -- ++(5pt,0pt) -- ++(-2.5pt,5pt) -- cycle;%green
	\end{scope}
}



%% draw a box of margin width size to see if figure is properly contained within
\newcommand{\marginsizebox}{\draw (0,0)--(\marginparwidth,0)--(\marginparwidth,3)--(0,3)--cycle;}

%%%%
%%%%

\newcommand{\asyouread}[1]{\begin{tikzpicture}
\ifthenelse{\boolean{in_color}}{\node [preaction={fill=black,opacity=.5,transform canvas={xshift=1mm,yshift=-1mm}}, right color=blue!80!black!30, left color=blue!80] at (0,0) {\textcolor{white}{\textsf{\textit{AS YOU READ $\ldots$}}}};}
{\node [preaction={fill=black,opacity=.5,transform canvas={xshift=1mm,yshift=-1mm}}, right color=black!30, left color=black!10] at (0,0) {\textcolor{white}{\textsf{\textit{AS YOU READ $\ldots$}}}};}
\end{tikzpicture}
\begin{enumerate}
#1
\end{enumerate}
\vskip 20pt}

%%%%
%%  A new figure environment, trying to fix the float problem.
%%
%%%%

\newcounter{myfigurecounter}[chapter]
\renewcommand\themyfigurecounter{\thechapter.\arabic{myfigurecounter}}
\newenvironment{myfigure}{\refstepcounter{myfigurecounter}}{}
\newcommand{\mycaption}[1]{%
\begin{center}%
\vskip -1.5\baselineskip
\begin{tikzpicture}%
\draw (0,0) node [text width=\textwidth,align=center] {Figure \themyfigurecounter: #1};%
\end{tikzpicture}%
\end{center}%
}