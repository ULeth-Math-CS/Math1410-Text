{Explain why the following is true: ``If the coefficient of the $x^2$ term in the equation of an ellipse in standard form is smaller than the coefficient of the $y^2$ term, then the ellipse has a horizontal major axis.''
}
{With the equation $\frac{(x-h)^2}{a^2}+\frac{(y-k)^2}{b^2}=1$, the ellipse has a horizontal major axis if $a>b$. But the coefficient of the $x^2$ term is $1/a^2$ (not $a^2$), so if $1/a^2<1/b^2$, then $a>b$ and the major axis is horizontal.
}
