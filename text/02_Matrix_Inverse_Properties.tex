\section{Properties of the Matrix Inverse}\label{sec:inverse_prop}

\asyouread{
\item What does it mean to say that two statements are ``equivalent?''
\item T/F: If \tta\ is not invertible, then \ttaxo\ could have no solutions.
\item T/F: If \tta\ is not invertible, then \ttaxb\ could have infinitely many solutions.
\item What is the inverse of the inverse of \tta?
\item T/F: Solving \ttaxb\ using Gaussian elimination is faster than using the inverse of \tta.
}

We ended the previous section by stating that invertible matrices are important. Since they are, in this section we study invertible matrices in two ways. First, we look at ways to tell whether or not a matrix is invertible, and second, we study properties of invertible matrices (that is, how they interact with other matrix operations). 

In the last section we stated Theorem \ref{thm:rank_and_inverse}, in which we listed several properties of a matrix that equivalent to that matrix being invertible, including the fact that an $n\times n$ matrix must have rank $n$ in order to be invertible.

We begin this section by collecting additional properties that are equivalent to matrix a matrix being invertible. Some of these results were established in the previous section, but we state them again here for ease of reference.

%Because of this importance, we will begin to collect ways in which we know a matrix \tta\ is invertible. We'll put these ideas together in the Invertible Matrix Theorem, which we'll be adding on to throughout the rest of the text.

\smallskip

\theorem{thm:IMT}{Invertible Matrix Theorem}{\index{Invertible Matrix Theorem}\index{inverse!properties}\index{inverse!Invertible Matrix Theorem}
Let \tta\ be an $n\times n$ matrix. The following statements are equivalent.
\begin{list}{(\alph{IMTcount})}{}\usecounter{IMTcount}

\item \label{imt:def} \tta\ is invertible. \addtocounter{IMTcount}{\value{IMTcount_temp}}
\item	\label{imt:axo} The equation \ttaxo\ has exactly one solution (namely, $\vx = \zero$).
\item	\label{imt:rref} The \rref\ of \tta\ is \tti.
\item	\label{imt:axb} The equation \ttaxb\ has exactly one solution for every $n\times 1$ vector \vb.
\item	\label{imt:ac} There exists a matrix \ttc\ such that $\tta\ttc = \tti$.
\item \label{imt:ba} There exists a matrix \ttb\ such that $\ttb\tta = \tti$.
\end{list}
\setcounter{IMTcount_temp}{\value{IMTcount}}
\addtocounter{IMTcount_temp}{-1}
}

\smallskip

\mnote{.4}{\textbf{Note:} Theorem \ref{thm:IMT} gives us several different ways of saying what is essentially the same thing (logically speaking). Theorems like this are very useful, since it's often the case that, in a given situation, one of the conditions can easily be checked, allowing us to immediately obtain information that might be difficult (or impossible) to verify directly.}

\mnote{.2}{\textbf{Note:} If we know that \tta\ is invertible, then we already know that there is a matrix \ttb\ where $\ttb\tta=\tti$. That is part of the definition of invertible. Part of the significance of this Theorem is that as soon as we find a matrix $B$ such that $AB=I$, we immediately know that $A$ is invertible, and that $BA=I$ as well. We made this claim in Theorem \ref{thm:inverse1}, but did not provide a proof at the time.}
%%
%%  Make notes here about what equivalent means, how each is new or signifcant
%%

Note that the theorem uses the phrase ``the following statements are {\em equivalent.}'' Recall that when two or more statements are equivalent, it means that the truth of any one of them implies that the rest are also true; if any one of the statements is false, then they are all false. So, for example, if we determined that the equation \ttaxo\ had exactly one solution (and \tta\ was an $n\times n$ matrix) then we would know that \tta\ was invertible, that \ttaxb\ had only one solution, that the \rref\ of \tta\ was \tti, etc.

Let's see exactly \textit{why} all of these statements are equivalent. What we will do is establish the following chain of logic:
\[
(\text{a})\Rightarrow (\text{b})\Rightarrow (\text{c})\Rightarrow (\text{d})\Rightarrow (\text{e})\Rightarrow (\text{f})\Rightarrow (\text{a}).
\]
Since this chain of implications circles back on itself, each of the statements implies the others. (For example, to show (e) implies (c), we start at (e) and circle around through (a) until we get to (c).
	\begin{itemize}
	\item	(a) $\Rightarrow$ (b): Suppose \ttaxo\ for some vector $\vx$. Since $A$ is invertible, we can multiply both sides of this equation by $A^{-1}$, giving us
	\[
	\vx = I\vx = (A^{-1}A)\vx = A^{-1}(A\vx) = A^{-1}\zero = \zero.
	\]
Thus, the only possible solution to the system is $\vx = \zero$.
	\item	(b) $\Rightarrow$ (c): Suppose the only solution to \ttaxo\ is $\vx = \zero$. Why must the \rref\ of $A$ be equal to $I$? If we let $R$ denote the \rref\ of $A$, we know that to solve \ttaxo, we form the augmented matrix $\bmx{c|c}A&\zero\emx$ and reduce:
	\[
	\bmx{c|c}A&\zero\emx \quad \arref \quad \bmx{c|c}R&\zero\emx
	\]
If the $R\neq I$, then $R$ has a row of zeros, and thus, so does $\bmx{c|c}R&\zero\emx$, in which case the system \ttaxo\ would have at least one parameter, and thus infinitely many solutions. Since we're assuming that \ttaxo\ has the unique solution $\vx=\zero$, it follows that $R$ cannot have a row of zeros, and thus $R=I$.
	\item	(c) $\Rightarrow$ (d): Suppose that the \rref\ of $A$ is equal to $I$. It follows that when solving the system \ttaxb, we would have
	\[
	\bmx{c|c} A&\vb\emx \quad \arref \quad \bmx{c|c}I& \vec{c}\emx
	\]
for some column vector $\vec{c}$, and thus $\vx = \vec{c}$ is the unique solution to \ttaxb.


	\item (d) $\Rightarrow$ (e): Suppose that $\ttaxb$ has a unique solution for every column vector $\vb$. Then in particular, we have a unique solution $\vec{c}_j$ to the systems $A\vx=\ven{j}$ for $j=1,\ldots, n$, where $\ven{1}, \ven{2}, \ldots, \ven{n}$ are the standard basis vectors in $\R^n$ (and also the columns of $I_n$). If we let 
	\[
	C = \bbm \vec{c}_1 & \vec{c}_2 &  \cdots & \vec{c}_n\ebm,
	\]
then
	\[
	AC = \bbm A\vec{c}_1 & A\vec{c}_2 &  \cdots & A\vec{c}_n\ebm = \bbm \ven{1} & \ven{2} & \cdots & \ven{n}\ebm = I_n.
	\]
	\item (e) $\Rightarrow$ (f): Suppose that $AC=I_n$ for some $n\times n$ matrix $C$. We claim that Property \ref{imt:axo} holds for the matrix $C$. To see this, note that since $AC = I_n$, if $C\vx = \zero$, then 
	\[
	\vx = I_n(\vx) = (AC)\vx = A(C\vx) = A\zero = \zero.
	\]
Since Property \ref{imt:axo} holds for the matrix $C$, it follows that Properties \ref{imt:rref}, \ref{imt:axb}, and \ref{imt:ac} do as well, by what we've proven so far. Thus, there exists a matrix $D$ such that $CD = I_n$. We can complete our proof by showing that $D=A$, and this is the case since (recalling that we've assumed $AC=I_n$)
	\[
	A = A(I_n) = A(CD) = (AC)D = I_nD = D.
	\]
Thus, (f) holds. \label{ax_xa_proof} (Note that this argument finally establishes the truth of Theorem \ref{thm:inverse1}.)

	\item (f) $\Rightarrow$ (a): Suppose that $BA=I_n$ for some matrix $B$. Using the same argument we just gave, it then follows that $AB=I_n$, and if $AB=BA=I_n$, then by definition $A$ is invertible and $B=A^{-1}$.
	
	

\mnote{.7}{Think about how we, up to this point, determined the solution to \ttaxb. We set up the augmented matrix $\bmx{c|c} \tta & \vb \emx$ and put it into \rref. We know that getting the identity matrix on the left means that we had a unique solution (and not getting the identity means we either have no solution or infinitely many solutions). So getting \tti\ on the left means having a unique solution; having \tti\ on the left means that the \rref\ of \tta\ is \tti, which we know is the same as \tta\ being invertible.}
	\end{itemize}
So we came up with a list of statements that are all {\em equivalent} to the statement ``\tta\ is invertible.'' Again, if we know that if any one of them is true (or false), then they are all true (or all false).


Theorem \ref{thm:IMT} states formally that if \tta\ is invertible, then \ttaxb\ has exactly one solution, namely $\ttai\vb$. What if \tta\ is not invertible? What are the possibilities for solutions to \ttaxb? 

We know that \ttaxb\ \textit{cannot} have exactly one solution; if it did, then by our theorem it would be invertible. Recalling that linear equations have either one solution, infinitely many solutions, or no solution, we are left with the latter options when \tta\ is not invertible. This idea is important and so we'll state it again as a Key Idea.

\smallskip

\keyidea{idea:solutions_invert}{Solutions to \ttaxb\ and the Invertibility of \tta}{
Consider the system of linear equations \ttaxb. 
\begin{enumerate}
\item	If \tta\ is invertible, then \ttaxb\ has exactly one solution, namely $\ttai\vb$.
\item	If \tta\ is not invertible, then \ttaxb\ has either infinitely many solutions or no solution.
\end{enumerate}
}

\smallskip

In Theorem \ref{thm:IMT} we've come up with a list of ways in which we can tell whether or not a matrix is invertible. At the same time, we have come up with a list of properties of invertible matrices -- things we know that are true about them. (For instance, if we know that \tta\ is invertible, then we know that \ttaxb\ has only one solution.) 

We now go on to discover other properties of invertible matrices. Specifically, we want to find out how invertibility interacts with other matrix operations. For instance, if we know that \tta\ and \ttb\ are invertible, what is the inverse of $\tta+\ttb$? What is the inverse of $\tta\ttb$? What is ``the inverse of the inverse?'' We'll explore these questions through an example.\\

%In this section, we have defined the inverse of a matrix, shown how to compute the inverse, and have given reasons why the inverse is important. Knowing now that the inverse is important, we should explore how it relates to other matrix operations. For instance, if we know that \tta\ and \ttb\ are invertible, what is the inverse of $\tta+\ttb$? What is the inverse of $\tta\ttb$? What is ``the inverse of the inverse?'' We'll explore these questions through an example.

\medskip

\example{ex_inv_prop}{Exploring properties of the inverse}{Let 
\[
\tta = \bmx{cc}3&2\\0&1\emx \text{ and } \ttb = \bmx{cc}-2 & 0\\1&1\emx.
\]
Find:

\begin{minipage}[t]{.3\linewidth}
\begin{enumerate}
\item		$\ttai$
\item		$\ttbi$
\end{enumerate}
\end{minipage}\begin{minipage}[t]{.3\linewidth}
\begin{enumerate}
\addtocounter{enumi}{2}
\item		$(\tta\ttb)^{-1}$
\item		$(\ttai)^{-1}$
\end{enumerate}
\end{minipage}
\begin{minipage}[t]{.3\linewidth}
\begin{enumerate}
\addtocounter{enumi}{4}
\item	$(\tta+\ttb)^{-1}$
\item	$(5\tta)^{-1}$
\end{enumerate}
\end{minipage}\\

\noindent In addition, try to find connections between each of the above.
}
{\begin{enumerate}
\item	Computing \ttai\ is straightforward; we'll use Theorem \ref{thm:2by2}. 
	
\[
\ttai = \frac{1}{3}\bmx{cc}1&-2\\0&3\emx = \bmx{cc} 1/3 & -2/3\\ 0 & 1\emx
\]
	
\item	We compute \ttbi\ in the same way as above.

\[
  \ttbi = \frac{1}{-2}\bmx{cc}1&0\\-1&-2\emx = \bmx{cc} -1/2 & 0 \\ 1/2 & 1\emx
\]
  
\item	To compute $(\tta\ttb)^{-1}$, we first compute \tta\ttb:

\[
\tta\ttb = \bmx{cc}3&2\\0&1\emx\bmx{cc}-2 & 0\\1&1\emx = \bmx{cc} -4&2\\1&1\emx
\]
We now apply Theorem \ref{thm:2by2} to find $(\tta\ttb)^{-1}$.
\[
(\tta\ttb)^{-1} = \frac{1}{-6}\bmx{cc}1&-2\\-1&-4\emx = \bmx{cc} -1/6 & 1/3\\1/6 & 2/3\emx
\]
	
\item	To compute $(\ttai)^{-1}$, we simply apply Theorem \ref{thm:2by2} to \ttai:
\[
(\ttai)^{-1} = \frac{1}{1/3}\bmx{cc}1 & 2/3\\0 & 1/3\emx = \bmx{cc} 3 & 2\\0 & 1\emx.
\]
  
\item	To compute $(\tta + \ttb)^{-1}$, we first compute $\tta+\ttb$ then apply Theorem \ref{thm:2by2}:
\[
\tta+\ttb = \bmx{cc}3&2\\0&1\emx+\bmx{cc}-2 & 0\\1&1\emx = \bmx{cc}1&2\\1&2\emx.
\]
We notice immediately that the two rows of $A+B$ are the same! Subtracting Row 1 from Row 2 would produce a row of zeros, so $A+B$ has rank $1<2$, and therefore cannot be invertible.
  
\item		To compute $(5\tta)^{-1}$, we compute 5\tta\ and then apply Theorem \ref{thm:2by2}.
\[
(5\tta)^{-1} = \left(\bmx{cc} 15 & 10\\0&5\emx\right)^{-1} = \frac{1}{75}\bmx{cc}5&-10\\0&15\emx = \bmx{cc} 1/15 & -2/15\\ 0 & 1/5\emx
\]
  
\end{enumerate}
\vs{.1}


We now look for connections between \ttai, \ttbi, $(\tta\ttb)^{-1}$, $(\ttai)^{-1}$ and $(\tta+\ttb)^{-1}$. 
%\drawexampleline{ex_inv_prop}
\drawexampleline%{ex_inv_prop}

\begin{enumerate}
\addtocounter{enumi}{2}
\item	 Is there some sort of relationship between $(\tta\ttb)^{-1}$ and \ttai\ and \ttbi? A first guess that seems plausible is $(\tta\ttb)^{-1} = \ttai\ttbi$. Is this true? Using our work from above, we have 
\[		
\ttai\ttbi = \bmx{cc} 1/3 & -2/3\\ 0 & 1\emx\bmx{cc} -1/2 & 0 \\ 1/2 & 1\emx = \bmx{cc} -1/2 & -2/3 \\ 1/2 & 1\emx.
\]
Obviously, this is not equal to $(\tta\ttb)^{-1}$. Before we do some further guessing, let's think about what the inverse of \tta\ttb\ is supposed to do. The inverse -- let's call it \ttc\ -- is supposed to be a matrix such that 
\[
(\tta\ttb)\ttc = \ttc(\tta\ttb) = \tti.
\]
%


%		
In examining the expression $(\tta\ttb)\ttc$, we see that we want \ttb\ to somehow ``cancel'' with \ttc. What ``cancels'' \ttb? An obvious answer is \ttbi. This gives us a thought: perhaps we got the order of \ttai\ and \ttbi\ wrong before. After all, we were hoping to find that 
\[
\tta\ttb\ttai\ttbi \overset{\text{?}}{=} \tti,
\]
but algebraically speaking, it is hard to cancel out these terms.(Recall that matrix multiplication is not commutative: $\tta\ttb\neq \ttb\tta$ in general.)
		However, switching the order of \ttai\ and \ttbi\ gives us some hope. Is $(\tta\ttb)^{-1} = \ttbi\ttai$? Let's see.
		
\begin{align*}
	(\tta\ttb)(\ttbi\ttai) &= \tta(\ttb\ttbi)\ttai \textsf{\hskip 20pt  \small(regrouping by the associative property)}\\ 
						&= \tta\tti\ttai \tag*{($\ttb\ttbi = \tti$)}\\
						&= \tta\ttai \tag*{($\tta\tti = \tta$)}\\
						&= \tti \tag*{$\tta\ttai=\tti$)}
\end{align*}

Since $(AB)(B^{-1}A^{-1})=I_n$, we know immediately from Theorem \ref{thm:IMT} that $(\tta\ttb)^{-1} = \ttbi\ttai$. Note also that our argument above was completely general, so this result holds true for \textit{any} pair of $n\times n$ matrices $A$ and $B$. Let's confirm this with our example matrices. 
\[   
 \ttbi\ttai = \bmx{cc} -1/2 & 0 \\ 1/2 & 1\emx\bmx{cc} 1/3 & -2/3\\ 0 & 1\emx = \bmx{cc} -1/6 & 1/3\\ 1/6 & 2/3 \emx = (\tta\ttb)^{-1}.
\]
It worked!
   
\item		Is there some sort of connection between $(\ttai)^{-1}$ and \tta? The answer is pretty obvious: they are equal. The ``inverse of the inverse'' returns one to the original matrix.

%\drawexampleline%{ex_inv_prop}

\item		Is there some sort of relationship between $(\tta+\ttb)^{-1}$, \ttai\ and \ttbi? Certainly, if we were forced to make a guess without working any examples, we would guess that 
\[
(\tta+\ttb)^{-1} \overset{\text{?}}{=} \ttai+\ttbi.
\]
However, we saw that in our example, the matrix $(\tta+\ttb)$ isn't even invertible. This pretty much kills any hope of a connection.

\item		Is there a connection between $(5\tta)^{-1}$ and \ttai? Consider:
	\begin{align*} 
   (5\tta)^{-1} &= \bmx{cc} 1/15 & -2/15\\ 0 & 1/5\emx \\
				&= \frac{1}{5} \bmx{cc} 1/3 & -2/3\\0&1/5\emx \\
				&= \frac 15 \ttai
	\end{align*}
\enlargethispage{2\baselineskip}	
	Yes, there is a connection!
\end{enumerate}
} 

\pagebreak

Let's summarize the results of this example. If \tta\ and \ttb\ are both invertible matrices, then so is their product, \tta\ttb. We demonstrated this with our example, and there is more to be said. Let's suppose that \tta\ and \ttb\ are $n\times n$ matrices, but we don't yet know if they are invertible. If \tta\ttb\ is invertible, then each of \tta\ and \ttb\ are; if \tta\ttb\ is not invertible, then either \tta\ or \ttb\ is not invertible.

\mnote{.55}{A natural question to ask at this point is whether or not the product of two non-invertible matrices could ever be invertible. The answer, not surprisingly, is ``No.'' To see this, suppose $AB$ is invertible. Then we know that the only solution to the equation $(AB)\vx = \zero$ is $\vx = \zero$. But if $B$ is not invertible, then the equation $B\vx=\zero$ \textit{does} have a non-trivial solution $\vx \neq \zero$, and then
\[
(AB)\vx = A(B\vx) = A\zero = \zero,
\]
implying that $\vx$ is a non-zero solution to $(AB)\vx = \zero$, and thus $AB$ cannot be invertible.}

In short, invertibility ``works well'' with matrix multiplication. However, we saw that it doesn't work well with matrix addition. Knowing that \tta\ and \ttb\ are invertible does not help us find the inverse of $(\tta+\ttb)$; in fact, the latter matrix may not even be invertible.

\mnote{.8}{The fact that invertibility works well with matrix multiplication should not come as a surprise. After all, saying that \tta\ is invertible makes a statement about the mulitiplicative properties of \tta. It says that I can multiply \tta\ with a special matrix to get \tti. Invertibility, in and of itself, says nothing about matrix addition, therefore we should not be too surprised that it doesn't work well with it.}

Let's do one more example, then we'll summarize the results of this section in a theorem.\\

\medskip

\example{ex_diagonal_inverse}{Computing the inverse of a diagonal matrix}{Find the inverse of $\tta = \bmx{ccc} 2&0&0\\0&3&0\\0&0&-7\emx$.}
{We'll find \ttai\ using Key Idea \ref{idea:finding_inverses}.
\[
\bmx{cccccc} 2&0&0&1&0&0\\0&3&0&0&1&0\\0&0&-7&0&0&1\emx \quad \overrightarrow{\text{rref}} \quad
\bmx{cccccc} 1&0&0&1/2&0&0\\0&1&0&0&1/3&0\\0&0&1&0&0&-1/7\emx
\]
Therefore 
\[
\ttai = \bmx{ccc} 1/2 & 0 & 0 \\ 0 & 1/3 & 0 \\ 0 & 0 & -1/7\emx.
\]
\ }

\medskip

The matrix \tta\ in the previous example is a \textit{diagonal} matrix: the only nonzero entries of \tta\ lie on the \textit{diagonal}.  The relationship between \tta\ and \ttai\ in the above example seems pretty strong, and it holds true in general. We'll state this and summarize the results of this section with the following theorem.

\mnote{.3}{We still haven't formally defined \textit{diagonal}, but the definition is rather visual so we risk it. \index{diagonal} See Definition \ref{def:diagonal} on page \pageref{def:diagonal} for more details.}

\smallskip

\theorem{thm:inv_prop}{Properties of Invertible Matrices}{\index{inverse!properties}
Let \tta\ and \ttb\ be $n\times n$ invertible matrices. Then:
\begin{enumerate}
\item	 	\tta\ttb\ is invertible; $(\tta\ttb)^{-1} = \ttbi\ttai$.
\item	 	\ttai\ is invertible; $(\ttai)^{-1} = \tta$.
\item		$n\tta$ is invertible for any nonzero scalar $n$; $(n\tta)^{-1} = \frac 1n \ttai$.
\item		If \tta\ is a diagonal matrix, with diagonal entries $d_1, d_2, \cdots , d_n$, where none of the diagonal entries are 0, then \ttai\ exists and is a diagonal matrix. Furthermore, the diagonal entries of \ttai\ are $1/d_1, 1/d_2, \cdots, 1/d_n$.
\end{enumerate}

Furthermore, 
\begin{enumerate}
\item	If a product \tta\ttb\ is not invertible, then \tta\ or \ttb\ is not invertible.
\item	If \tta\ or \ttb\ are not invertible, then \tta\ttb\ is not invertible.
\end{enumerate}
}

\pagebreak

We end this section with a comment about solving systems of equations ``in real life.'' Solving a system \ttaxb\ by computing \ttai\vb\ seems pretty slick, so it would make sense that this is the way it is normally done. However, in practice, this is rarely done. There are two main reasons why this is the case.

\mnote{.7}{Yes, real people do solve linear equations in real life. Not just mathematicians, but economists, engineers, and scientists of all flavours regularly need to solve linear equations, and the matrices they use are often \textit{huge.}

Most people see matrices at work without thinking about it. Digital pictures are simply ``rectangular arrays'' of numbers representing colours -- they are matrices of colours. Many of the standard image processing operations involve matrix operations. The author's wife has a ``7 megapixel'' camera which creates pictures that are $3072\times 2304$ in size, giving over 7 million pixels, and that isn't even considered a ``large'' picture these days.}

First, computing \ttai\ and \ttai\vb\ is ``expensive'' in the sense that it takes up a lot of computing time. Certainly, our calculators have no trouble dealing with the $3 \times 3$ cases we often consider in this textbook, but in real life the matrices being considered are very large (as in, hundreds of thousand rows and columns). Computing \ttai\ alone is rather impractical, and we waste a lot of time if we come to find out that \ttai\ does not exist. Even if we already know what \ttai\ is, computing $\ttai\vb$ is computationally expensive -- Gaussian elimination is faster.

Secondly, computing \ttai\ using the method we've described often gives rise to numerical roundoff errors. Even though computers often do computations with an accuracy to more than 8 decimal places, after thousands of computations, rounding off can cause big errors. (A ``small'' $1,000 \times 1,000$ matrix has $1,000,000$ entries! That's a lot of places to have roundoff errors accumulate!) It is not unheard of to have a computer compute \ttai\ for a large matrix, and then immediately have it compute \tta\ttai\ and \textit{not} get the identity matrix. (The result is usually very close, with the numbers on the diagonal close to 1 and the other entries near 0. But it isn't exactly the identity matrix.)

Therefore, in real life, solutions to \ttaxb\ are usually found using the methods we learned in Section \ref{sec:vector_solutions}. It turns out that even with all of our advances in mathematics, it is hard to beat the basic method that Gauss introduced a long time ago.

\printexercises{exercises/02_06_exercises}
