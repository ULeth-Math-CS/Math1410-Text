\thispagestyle{empty}
\Huge
\noindent {\bf \textsc{Preface}}\\
\normalsize

This is a textbook on Linear Algebra, designed to be compatible with the curriculum for the course Math 1410 at the University of Lethbridge. As this is a first-year course taken primarily by non-majors, the focus of the text (and the course) is primarily computational, and as a result, much of the theory that common to many first courses in linear algebra is not included.

The book does, however, contain a few topics not always found in a linear algebra course (but required for Math 1410), including vector geometry and complex numbers.

The book is \textit{free}, in every sense of the word. There is no cost to the student, unless one wants a hard copy, in which case the only cost is that of having it printed. The text is also free in less tangible, but equally important ways. The book is licensed under a Creative Commons Public License, which allows you to share the book with others, and even make and distribute copies, as long as this is not done for financial gain. The book is also \textit{open source}: all of the code and figures used to generate the text (over 2,000 files in all) are available online and can be downloaded and edited by anyone wishing to create their own version of the book.

The book has a few quirks compared to other texts. There is an extensive treatment of the arithmetic of complex numbers, since this is required in the Math 1410 course. The treatment of systems of equations also appears quite late in the textbook. In my experience, when the course begins with systems of equations, many students tune out early, either because the material is too easy, or too boring, and aren't able to get back up to speed once the level of difficulty picks up. Having some challenging material early on provides some incentive to do the homework and seek help from the beginning.

Since Math 1410 does not align well with the standard linear algebra curriculum found in many American universities, it was not possible to work with any one pre-existing open textbook. Instead, this book is an amalgamation of three texts, as specified on the copyright page, along with a substantial chunk of content that I've added myself. As a result, you'll find that there are slight differences in the format of some chapters.

The treatment of standard topics like basis and dimension is quite limited, due to the nature of the course. Arguments are provided in support of most theorems in the text, but there aren't too many formal proofs. Some topics that are sometimes taught in Math 1410 also receive a limited treatment here. Eigenvalues and eigenvectors are covered, but diagonalization is not. (Update for the Spring 2017 edition: a section on diagonalization has been added.) The coverage of orthogonality is limited to the shortest distance applications in the chapter on vector geometry. In some cases these omissions are due to time constraints, both in the preparation of the textbook and in the delivery of the material in the classroom. In some cases (such as the Gram-Schmidt procedure and orthogonal diagonalization), there was a concious decision to omit material that more rightly belongs in a second course in linear algebra.

With any luck, the book will continue to evolve as I and other instructors use it. The book is a work in progress, and is bound to have some failings. If you encounter any errors as you use the book (or if you don't like how something is presented, or if you think more exercises of a certain type are needed, etc.) please let me know, and I'll do my best to make the changes. 



\pagebreak

\noindent\textbf{\large Changes in the Spring 2018 edition}

For Spring 2018, the course is being taught by Jana Archibald, who asked that we re-institute the chapter on linear transformations. She also contributed a (previously missing) set of problems for the section on elementary matrices.

\bigskip

\noindent\textbf{\large Changes in the Fall 2017 edition}

For Fall 2017 I've reorganized the textbook based on the input of Habiba Kadiri, since she will be teaching Math 1410 this semester. The main change is to bring the material on systems of equations a bit earlier in the textbook, placing it prior to the chapter on matrices. A few adjustments in writing were needed to accommodate this change. Some of the theoretical content has been omitted from the chapter on vectors in $\mathbb{R}^n$. The material on linear transformations has been moved to the end of the textbook as an optional chapter, to be covered if time permits. I've moved the material on null space and column space into its own section at the end of this chapter. This move includes relocating or reproducing the relevant examples from the section on vector solutions to linear systems. (Time constraints kept me from creating new examples, so there are some examples in Section \ref{sec:vector_solutions} that have been re-written to remove the references to null space and column space, and the same examples reappear in their original context at the end of the book.)

I have also written a new section, on elementary matrices. In the previous edition the only mention of this topic was in a marginal note.

\bigskip

\noindent\textbf{\large Changes in the Spring 2017 edition}

During the inaugural Fall 2016 run for this textbook, I came across a number of errors and typos that have been corrected. Most of these were formatting errors, but there were a few incorrect answers in the back, and one case in Section 6.1 where I somehow used the columns from the wrong matrix in a column space example!

In Chapter 2 I've added formal definitions for addition and multiplication of complex numbers, and the presentation in Section 2.2 has been streamlined. Since the polar coordinate representation of a complex number never has $r<0$, I've removed the material (needed for calculus) involving polar coordinates with $r<0$, as well as the material (also needed for calculus) on converting equations of curves in the plane from rectangular to polar coordinates. Replacing the ``cis'' notation with Euler's exponential notation remains on the to-do list.

In Chapter 3, I've added the standard definition of parallel vectors for linear algebra, and changed the original definition to a theorem. In Chapter 4, I've added additional examples and exercises on span and linear independence in Section 4.2, and two additional examples in Section 4.5 on working with the definition of a linear transformation.

Chapters 5-7 remain largely unchanged, aside from error corrections. In Chapter 8, I've added a new section, on diagonalization. This section introduces the main ideas, such as similar matrices, algebraic and geometric multiplicity, and eigenspaces. There is no treatment of orthogonal diagonalization, since I still consider this topic to be a better fit for Math 3410 than Math 1410.

\pagebreak


\noindent\textbf{\large Acknowledgements}\\

First and foremost, I need to thank the authors of the textbooks that provide the source material for this text. Without their hard work, and willingness to make their books (and the source code) freely available, it would not have been possible to create an affordable textbook for this course. You can find the original textbooks at their websites:

\bigskip


\href{http://www.stitz-zeager.com}{www.stitz-zeager.com}, for the \textit{Precalculus} textbook, by Stitz and Zeager, 

\bigskip

\href{http://www.vmi.edu/academics/departments/applied-mathematics/affordable-textbooks-apex/}{http://www.vmi.edu/academics/departments/applied-mathematics/affordable-textbooks-apex/}, for the \textit{Fundamentals of Matrix Algebra} textbook, by Gregory Hartman, and

\bigskip

\href{http://www.apexcalculus.com}{apexcalculus.com}, for the \apex\ \textit{Calculus} textbook, by Hartman et al.

I would also like to thank Habiba Kadiri for her input on the organization of the textbook for the Fall 2017 edition, and Jana Archibald for her assistance with editing the text to resolve some of the continuity problems arising from the reorganization.

\vspace{1in}

\begin{raggedright}
Sean Fitzpatrick\\
Department of Mathematics and Computer Science\\
University of Lethbridge\\
August 18th, 2017.
\end{raggedright}




