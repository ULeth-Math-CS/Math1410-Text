\section{Solving Matrix Equations $\tta\ttx=\ttb$}\label{sec:solve_axb}

\asyouread{
\item	T/F: To solve the matrix equation $\tta\ttx=\ttb$, put the matrix $\bmx{cc} \tta & \ttx \emx$ into \rref\ and interpret the result properly.
\item	T/F: The first column of a matrix product \tta\ttb\ is \tta\ times the first column of \ttb.
\item	Give two reasons why one might solve for the columns of \ttx\ in the equation \tta\ttx=\ttb\ separately.
}

We concluded the last chapter with a discussion about solving numerical equations like $ax=b$ for $x$. We have seen how to solve equations of the form \ttaxb\ by identifying them as systems of linear equations. In this section we will learn how to solve the general matrix equation $\tta\ttx=\ttb$ for \ttx.

We will start by considering the best case scenario when solving \ttaxb; that is, when \tta\ is square and we have exactly one solution. For instance, suppose we want to solve \ttaxb\ where 
\[
\tta = \bmx{cc}1&1\\2&1\emx \quad \text{and} \quad \vb = \bmx{c} 0\\1\emx.
\]
We know how to solve this; put the appropriate matrix into \rref\ and interpret the result. 
\[
\bmx{cc|c}1&1&0\\2&1&1\emx \quad \arref \quad \bmx{cc|c}1&0&1\\0&1&-1\emx
\]
We read from this that 
\[
\vx = \bmx{c}1\\-1\emx.
\]

Written in a more general form,  we found our solution by forming the augmented matrix 
\[
\bmx{c|c}\tta & \vb\emx
\]
and interpreting its \rref: 
\[
\bmx{c|c}\tta & \vb\emx \quad \arref \quad \bmx{c|c}\tti & \vx \emx
\]
Notice that when the \rref\ of \tta\ is the identity matrix \tti\ we have exactly one solution. This, again, is the best case scenario.

We apply the same general technique to solving the matrix equation $\tta\ttx=\ttb$ for \ttx. We'll assume that \tta\ is a square matrix (\ttb\ need not be) and we'll form the augmented matrix 
\[
\bmx{c|c}\tta&\ttb\emx.
\]
Putting this matrix into \rref\ will give us \ttx, much like we found \vx\ before.
\[
\bmx{c|c}\tta&\ttb\emx \quad \arref \quad \bmx{c|c}\tti & \ttx\emx
\]

As long as the \rref\ of \tta\ is the identity matrix, this technique works great. After a few examples, we'll discuss why this technique works, and we'll also talk %\footnote{we'll just talk; we won't actually answer anything}
  just a little bit about what happens when the \rref\ of \tta\ is not the identity matrix.

First, some examples.

\medskip

\example{ex_axb_1}{Solving a matrix equation}{Solve the matrix equation $\tta\ttx=\ttb$ where 
\[
\tta = \bmx{cc} 1&-1\\ 5&3\emx\quad \text{and} \quad \ttb = \bmx{ccc} -8&-13&1\\32&-17&21\emx.
\]}
{%
To solve $\tta\ttx=\ttb$ for \ttx, we form the proper augmented matrix, put it into \rref, and interpret the result.

\[
\bmx{cc|ccc}1&-1&-8&-13&1\\5&3&32&-17&21\emx\quad \arref \quad 
\bmx{cc|ccc}1&0&1&-7&3\\0&1&9&6&2\emx
\]

We read from the \rref\ of the matrix that 
\[
\ttx = \bmx{ccc}1&-7&3\\9&6&2\emx.
\]
We can easily check to see if our answer is correct by multiplying $\tta\ttx$.}

\medskip

\example{ex_axb_2}{Another matrix equation}{Solve the matrix equation $\tta\ttx = \ttb$ where 
\[
\tta = \bmx{ccc}1&0&2\\0&-1&-2\\2&-1&0\emx \quad\text{and}\quad \ttb = \bmx{cc}-1&2\\2&-6\\2&-4\emx.
\]}
{To solve, let's again form the augmented matrix 
\[
\bmx{c|c} \tta & \ttb\emx,
\]
put it into \rref, and interpret the result.

%\enlargethispage{2\baselineskip}

\[
\bmx{ccc|cc}1&0&2&-1&2\\0&-1&-2&2&-6\\2&-1&0&2&-4\emx\quad\arref\quad \bmx{ccc|cc}1&0&0&1&0\\0&1&0&0&4\\0&0&1&-1&1\emx
\]

We see from this that 
\[
\ttx = \bmx{cc}1&0\\0&4\\-1&1\emx.
\]
\vskip -\baselineskip}

\medskip

%As we have done many times in the past, we need to learn another new concept before we can directly tackle the problem at hand. Since we are solving a problem that involves matrix multiplication, it will be useful to better understand matrix multiplication.

Why does this work? To see the answer, let's define five matrices. 
\[
\tta = \bmx{cc}1&2\\3&4\emx, \ \ \vu =\bmx{c}1\\1\emx, \ \ \vv =\bmx{c}-1\\1\emx, \ \ \vw = \bmx{c} 5\\6\emx \ \ \text{and} \ \ \ttx = \bmx{ccc}1&-1&5\\1&1&6\emx
\]

Notice that \vu, \vv\ and \vw\ are the first, second and third columns of \ttx, respectively. Now consider this list of matrix products: $\tta\vu$, $\tta\vv$, $\tta\vw$ and $\tta\ttx$.

\begin{center}
\begin{minipage}{100pt}
\begin{align*}
	\tta\vu &= \bmx{cc}1&2\\3&4\emx\bmx{c}1\\1\emx\\
			&= \bmx{c}3\\7\emx
\end{align*}
					
\begin{align*}
	\tta\vw	&= \bmx{cc}1&2\\3&4\emx\bmx{c}5\\6\emx\\
			&= \bmx{c}17\\39\emx
\end{align*}
\end{minipage}
\begin{minipage}{20pt}
\
\end{minipage}
\begin{minipage}{100pt}
\begin{align*}
	\tta\vv &= \bmx{cc}1&2\\3&4\emx\bmx{c}-1\\1\emx\\
			&= \bmx{c}1\\1\emx
\end{align*}
					
\begin{align*}
	\tta\ttx	&= \bmx{cc}1&2\\3&4\emx\bmx{ccc}1&-1&5\\1&1&6\emx\\
				&= \bmx{ccc}3&1&17\\7&1&39\emx
\end{align*}
\end{minipage}
\end{center}

So again note that the columns of \ttx\ are \vu, \vv\ and \vw; that is, we can write 
\[
\ttx = \bmx{ccc}\vu & \vv&\vw \emx.
\]
Notice also that the columns of $\tta\ttx$ are $\tta\vu$, $\tta\vv$ and \tta\vw, respectively. Thus we can write 
\begin{align*}
\tta\ttx &= \tta\bmx{ccc}\vu & \vv&\vw \emx\\
		 &=\bmx{ccc} \tta\vu & \tta\vv & \tta\vw \emx\\
		 &=\bmx{ccc} \bmx{c}3\\7\emx & \bmx{c} 1\\1\emx & \bmx{c}17\\39\emx \emx \\
		 &=\bmx{ccc}3&1&17\\7&1&39\emx
\end{align*}
This is exactly the same sort of thing we did in Section \ref{sec:solving_systems} when we had several vectors and we wanted to determine whether or not each of them belonged to a given span. Rather than perform the same set of row operations for each vector separately, we can do them all together. (See the discussion following Example \ref{ex_span1}.)

Thus, we are once again making use of the following fact:
%, which we first observed in Section \ref{sec:lin_trans} (see the margin note on Page \pageref{note:ABvect}):
\begin{quote}
 The columns of a matrix product \tta\ttx\  are \tta\ times the columns of \ttx.
\end{quote}

How does this help us solve the matrix equation $\tta\ttx=\ttb$ for \ttx? Assume that \tta\ is a square matrix (that forces \ttx\ and \ttb\ to be the same size). We'll let $\vect{x_1},\vect{x_2}, \cdots \vect{x_n}$ denote the columns of the (unknown) matrix \ttx, and we'll let $\vect{b_1},\vect{b_2}, \cdots \vect{b_n}$ denote the columns of \ttb. We want to solve $\tta\ttx = \ttb$ for \ttx. That is, we want \ttx\ where 
\begin{align} 
\tta\ttx &= \ttb \notag\\
\tta\bmx{cccc}\vect{x_1}&\vect{x_2}& \cdots & \vect{x_n} \emx & = \bmx{cccc}\vect{b_1}&\vect{b_2}& \cdots & \vect{b_n} \emx \label{eq:AXB}\\
\bmx{cccc}\tta\vect{x_1}&\tta\vect{x_2}& \cdots & \tta\vect{x_n} \emx & = \bmx{cccc}\vect{b_1}&\vect{b_2}& \cdots & \vect{b_n} \emx \notag
\end{align}

If the matrix on the left hand side is equal to the matrix on the right, then their respective columns must be equal. This means we need to solve $n$ equations:
\begin{align*}
\tta\vect{x_1} & = \vect{b_1}\\
\tta\vect{x_2} & = \vect{b_2}\\
\vdots\		\		& = \ \vdots \\
\tta\vect{x_n} &= \vect{b_n}
\end{align*}

We already know how to do this; this is what we learned in the previous section. Let's do this in a concrete example. In our above work we defined matrices \tta\ and \ttx, and looked at the product \tta\ttx. Let's call the product \ttb; that is, set \ttb = \tta\ttx. Now, let's pretend that we don't know what \ttx\ is, and let's try to find the matrix \ttx\ that satisfies the equation $\tta\ttx=\ttb$. As a refresher, recall that 
\[
\tta = \bmx{cc}1&2\\3&4\emx \quad \text{and} \quad \ttb = \bmx{ccc}3&1&17\\7&1&39\emx.
\]

Since \tta\ is a $2\times 2$ matrix and \ttb\ is a $2\times 3$ matrix, what dimensions must \ttx\ be in the equation $\tta\ttx=\ttb$? The number of rows of \ttx\ must match the number of columns of \tta; the number of columns of \ttx\ must match the number of columns of \ttb. Therefore we know that \ttx\ must be a $2\times 3$ matrix.

We'll call the three columns of \ttx\ $\vect{x_1}$, $\vect{x_2}$ and $\vect{x_3}$. Our previous explanation tells us that if $\tta\ttx = \ttb$, then:
\begin{align*}
\tta\ttx & = \ttb\\
\tta\bmx{ccc}\vect{x_1} & \vect{x_2} & \vect{x_3}\emx & = \bmx{ccc}3&1&17\\7&1&39\emx \\
\bmx{ccc}\tta\vect{x_1} & \tta\vect{x_2} & \tta\vect{x_3}\emx & =\bmx{ccc}3&1&17\\7&1&39\emx. 
\end{align*}
Hence
\begin{align*}
\tta\vect{x_1} & = \bmx{c}3\\7\emx\\
\tta\vect{x_2} & = \bmx{c}1\\1\emx\\
\tta\vect{x_3} & = \bmx{c}17\\39\emx
\end{align*}

To find \vect{x_1}, we form the proper augmented matrix and put it into \rref\ and interpret the results. 
\[
\bmx{cc|c}1&2&3\\3&4&7\emx \quad \arref \quad \bmx{cc|c}1&0&1\\0&1&1\emx
\]
This shows us that 
\[
\vect{x_1} = \bmx{c}1\\1\emx.
\]

To find \vect{x_2}, we again form an augmented matrix and interpret its \rref. 
\[
\bmx{cc|c}1&2&1\\3&4&1\emx \quad \arref \quad \bmx{cc|c}1&0&-1\\0&1&1\emx
\]
Thus 
\[
\vect{x_2} = \bmx{c}-1\\1\emx
\]
which matches with what we already knew from above.

Before continuing on in this manner to find \vect{x_3}, we should stop and think. If the matrix vector equation \ttaxb\ is consistent, then the steps involved in putting 
\[
\bmx{c|c}\tta & \vb\emx
\]
into \rref\ depend only on \tta; it does not matter what \vb\ is. So when we put the two matrices 
\[
\bmx{cc|c}1&2&3\\3&4&7\emx \quad \text{and} \quad \bmx{cc|c}1&2&1\\3&4&1\emx
\]
from above into \rref, we performed exactly the same steps! (In fact, those steps are: $-3R_1+R_2\rightarrow R_2$; $-\frac12R_2\rightarrow R_2$; $-2R_2+R_1\rightarrow R_1$.)

This is just as we noted after Example \ref{ex_span1}. Instead of solving for each column of \ttx\ separately, performing the same steps to put the necessary matrices into \rref\ three different times, why don't we just do it all at once? (Unless you enjoy doing unnecessary work.) Instead of individually putting 
\[
\bmx{cc|c}1&2&3\\3&4&7\emx, \quad \bmx{cc|c}1&2&1\\3&4&1\emx \quad \text{and} \quad \bmx{cc|c}1&2&17\\3&4&39\emx
\]
into \rref, let's just put 
\[
\bmx{cc|ccc}1&2&3&1&17\\3&4&7&1&39\emx
\]
into \rref.
\[
\bmx{cc|ccc}1&2&3&1&17\\3&4&7&1&39\emx \quad \arref \quad \bmx{cc|ccc}1&0&1&-1&5\\0&1&1&1&6\emx
\]

By looking at the last three columns, we see \ttx: 
\[
\ttx = \bmx{ccc}1&-1&5\\1&1&6\emx.
\]

%We'll return to theory in a moment. First, a few examples.

In each of the examples we've considered so far, the \rref\ $R$ of the matrix $A$ was equal to the $n\times n$ identity matrix: $R=I$. It follows from Definition \ref{def:rank} in Section \ref{sec:vector_solutions} that for an $n\times n$ matrix $A$, we have $R=I$ if and only if the rank of $A$ is equal to $n$. At this point we should recall Theorem \ref{thm:rank_and_sols} from Section \ref{sec:vector_solutions}, and the discussion that followed. One of the things Theorem \ref{thm:rank_and_sols} tells us is that if $A$ is an $n\times n$ matrix and $\operatorname{rank}(A)=n$, then the equation $A\vec{x}=\vec{b}$ is \sword{guaranteed} to have a unique solution, no matter what the vector $\vec{b}$ is.

%\mnote{.5}{\textbf{Note:} Recall the following argument for why the equation \ttaxb\ is guaranteed a unique solution if $A$ has rank $n$. 
%If we assume $\operatorname{rank}(A) = \dim\operatorname{col}(A) = n$, then the range of the linear transformation $T(\vx)=A\vx$ is all of $\R^n$, so for every $\vb \in \R^n$, we can find a vector $\vx$ such that $A\vx = \vb$. The solution is unique since Theorem \ref{thm:fund_thm_lin_maps} in Section \ref{sec:lin_trans} (the ``Rank-Nullity Theorem'') tells us that
%\[
%\dim\operatorname{null}(A)+\dim\operatorname{col}(A) = n,
%\]
%so the null space of $A$ must be zero-dimensional, and therefore equal to the trivial subspace $\{0\}$. By Theorem \ref{thm:onetoonenull}, this tells us that the vector $\vx$ such that \ttaxb\ is unique.}

But what if $\operatorname{rank}(A)<n$? In this case, the \rref\ of $A$ is an $n\times n$ matrix $R$ with at least row of zeros on the bottom. Our experience with solving systems of the form \ttaxb\ tells us that in this case, the matrix equation $AX=B$ may have infinitely many solutions, or no solution at all. Let us consider an example.

\medskip

\example{ex_AXB_subrank}{Solving $AX=B$ when $\operatorname{rank}(A)<n$}{
Solve the matrix equations $AX=B$ and $AX=C$, where
\[
A = \bbm 1&0&-3\\-2&3&4\\0&6&-4\ebm, B = \bbm 1&2&0\\1&1&3\\6&3&0\ebm, \text{ and } C = \bbm 4&-2&-3\\-6&5&7\\4&2&2\ebm.
\]
}
{We proceed as in the previous examples. For the equation $AX=B$, we have
\[
\bmx{c|c}A&B\emx = \begin{aamatrix}{3}1&0&-3&1&2&0\\-2&3&4&1&1&3\\0&6&-4&6&3&0\end{aamatrix} \quad \to \quad
\begin{aamatrix}{3}1&0&-3&1&2&0\\0&3&-2&3&5&3\\0&0&0&0&-7&-6\end{aamatrix}.
\]
We stopped before reaching the \rref, but there's no reason to continue: we already have a row of zeros on the left-hand side of the augmented matrix, and two non-zero entries in that same row, on the right. What this tells us is that it will be impossible to solve for the second and third columns of $X$; thus, there is no solution in this case.

For the equation $AX=C$, we have
\[
\bmx{c|c}A&C\emx = \begin{aamatrix}{3}1&0&-3&4&-2&-3\\-2&3&4&-6&5&7\\0&6&-4&4&2&2\end{aamatrix} \quad \arref \quad
\begin{aamatrix}{3}1&0&-3&4&-2&3\\0&1&-2/3&2/3&1/3&1/3\\0&0&0&0&0&0\end{aamatrix}.
\]
In this case, we \textit{are} able to solve for each column of $X$, but in each case there are infinitely many possibilities: we find $X = \bbm \vec{x}_1&\vec{x}_2&\vec{x}_3\ebm$, where
\[
\vec{x}_1 = \bbm 4+3r\\ \frac{2}{3}+\frac{2}{3}r\\r\ebm, \vec{x}_2 = \bbm -2+3s\\ \frac{1}{3}+\frac{2}{3}s\\s\ebm, \vec{x}_3 = \bbm -3+3t\\ \frac{1}{3}+\frac{2}{3}t\\ t\ebm,
\]
for parameters $r, s, t$. Any choice of values for each of these parameters provides us with a solution. For simple example, we can set all three parameters equal to zero, giving us
\[
X = \bbm 4&-2&-3\\2/3&1/3&1/3\\0&0&0\ebm.
\]
It's easy to check that indeed, $AX=C$ in this case.}

\medskip

In the previous example, we saw that the equation $AX=B$ had no solution, while we were able to solve $AX=C$.
How do we know which of these will be the case? Let's go back to Equation \eqref{eq:AXB} above. From this equation, we can see that each column of $B$ is of the form $\vec{b}_i = A\vec{x}_i$ for some vector $\vec{x}_i$. Now recall from Section \ref{sec:Rn} that the set of vectors of the form $A\vx$ is precisely the column space of $A$. In the case of the matrix $C$, we can check that if we write
\[
A = \bbm \vec{a}_1&\vec{a}_2&\vec{a}_3\ebm \text{ and } C = \bbm \vec{c}_1 & \vec{c}_2 & \vec{c}_3\ebm,
\]
then $\vec{c}_1 = \vec{a}_1-\vec{a}_2$, $\vec{c}_2 = \vec{a}_1+\vec{a}_2+\vec{a}_3$, and $\vec{c}_3 = \vec{a}_2+\vec{a}_3$. Thus, all three columns of $C$ are linear combinations of the columns of $A$, which is what allowed us to find a solution to $AX=C$ even though the rank of $A$ was less than $n$.

Now that we've justified the technique we've been using in this section to solve $\tta\ttx=\ttb$ for \ttx, we reinforce its importance by restating it as a Key Idea.

\smallskip

\keyidea{idea:solve_AXB}{Solving $\tta\ttx=\ttb$}{
Let \tta\ be an $n\times n$ matrix, where the \rref\ of \tta\ is \tti. To solve the matrix equation $\tta\ttx=\ttb$ for \ttx,
	\begin{enumerate}
	\item		Form the augmented matrix $\bmx{c|c}\tta&\ttb\emx$.
	\item		Put this matrix into \rref. 
	\begin{enumerate}
	\item If it is of the form $\bmx{c|c} \tti & C \emx$, for some matrix $C$ that appears in the columns where \ttb\ once was, then $C=X$.
    \item If it is of the form $\bmx{c|c} R & C\emx$, where the matrix $R$ (the \rref\ of $A$) has one or more rows of zeros, then there will be either no solution or infinitely many solutions, depending on whether or not the columns of $B$ belong to the column space of $A$.
    \end{enumerate}
	\end{enumerate}
}
	
\smallskip

These simple steps cause us to ask certain questions. One of these we asked (and answered) above: What if \tta\ does not have maximum rank, so that the \rref\ of $A$ is not equal to $I$? Second, we specify above that \tta\ should be a square matrix. What happens if \tta\ isn't square? Is a solution still possible? If you study what happens in Example \ref{ex_AXB_subrank} carefully, you can probably guess that a similar argument applies, by applying the ideas of Section \ref{sec:vector_solutions} to each column of $B$ individually.

%what if  we found the \rref\ of $\bmx{cc}\tta&\ttb\emx$ and we did not get the identity matrix \tti\ where \tta\ used to be? What would that mean?
%\footnote{Recall that solutions to linear systems come in one of three forms: exactly one solution, infinite solutions and no solution. We have focused here on exactly one solution to $\tta\ttx=\ttb$. What would the \rref\ of $\bmx{cc} \tta & \ttb\emx$ look like if there were no solution? Infinite solutions? How would we represent the solution if there were infinite solutions?}


These questions are good to ask, and we leave it to the reader to discover their answers. Instead of tackling these questions, we instead tackle the problem of ``Why do we care about solving $\tta\ttx=\ttb$?'' The simple answer is that, for now, we only care about the special case when $\ttb=\tti$. By solving $\tta\ttx = \tti$ for \ttx, we find a matrix \ttx\ that, when multiplied by \tta, gives the identity \tti. That will be very useful.\\

%revisit an old algebra problem: solve $ax=b$ for $x$. 

%We know that $x = \frac{b}{a}$; we can write this several different ways. One formal way is to write $x = a^{-1}b$, where $a^{-1}$ is that special number where $a^{-1}\cdot a = 1$. 

\printexercises{exercises/02_04_exercises}
